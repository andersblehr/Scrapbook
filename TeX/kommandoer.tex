%%%%%%%%%%%%%%%%%%%%%%%%%%%% -*- Mode: Plain-Tex -*- %%%%%%%%%%%%%%%%%%%%%%%%%%
%%% kommandoer.tex --- 
%%% Author          : Anders Blehr
%%% Created On      : Sat Apr 25 17:54:59 1992
%%% Last Modified By: Anders Blehr
%%% RCS revision    : $Revision: 1.1 $ $Locker:  $
%%% Status          : Unknown, Use with caution!
%%%%%%%%%%%%%%%%%%%%%%%%%%%%%%%%%%%%%%%%%%%%%%%%%%%%%%%%%%%%%%%%%%%%%%%%%%%%%%

\input{psfig}

\newcommand{\hsql}{{\it HSQL\/}}
\newcommand{\nal}{{\it SNaL\/}}
\newcommand{\nash}{{\it NaSh\/}}
\newcommand{\niks}{{\it NIKS\/}}
\newcommand{\nit}{NTH}
\newcommand{\nlp}{NLP}
\newcommand{\SICStus}{{\it SICStus Prolog\/}}
\newcommand{\solon}{{\it SOLON\/}}
\newcommand{\tuc}{{\it TUC\/}}

% Et tastetrykk, eller en kombinasjon av to taster.
\newcommand{\tast}[1]{\fbox{\rule{0mm}{1.5ex}{\tt #1}}}
\newcommand{\totast}[2]{\tast{#1}-\tast{#2}}

% Binder en taste-sekvens mot en kommando.
\newcommand{\keybinding}[2]{#1 executes {\em #2}. See Section~\ref{#2-func}.}

% Kommando som definerer ny funksjon.
% Parameter 1 : Funksjonsnavn, Kommer til venstre under en strek.
%           2 : Parametre. Kommer til h|yre for navnet i kursiv
%           3 : Command|Function|Variable    Kommer til hoyre.
%           4 : Forklaringen. Den quote's.
\newcommand{\descfunc}[3]{
OBSOBSOBSOBSOBSOBSOBS DESCFUNC er ikke tilgjengelig mer!!! \\
\typeout{Ikke bruk descfunc!!!!! (#1)}}

% \newenvironment{params}{{\large Parameters:}\begin{description}}{\end{description}}

\newcommand{\returns}[1]{
        \begin{description}
          \item [Returns : ]#1
        \end{description}}

% Dette er environmentet som brukes ved forklaring av funksjoner,
% programmer eller variable.
\newenvironment{func}[3]{\pagebreak[3] \label{#1-func} %
  {\large\it  #1\hspace{1cm} #2 \hfill {\bf #3}}
\begin{quote}}{\end{quote}}


% Hente PS-bilder. Bredde = 0.8\textwidth
%
% Bruksm}te :
%   \hentps{mintegning}{Forklarende tekst til figuren}
%
\newcommand{\hentps}[2] {\hentpsbredde{#1}{#2}{0.8}}

% Bruksm}te:
%  \hentpsbredde{mintegning}{Forklarende tekst \label{labref}}{tall}
%
%  'tall' er et tall mellom 0 og 1 som angir br|kdel av tekstbredden.
%
\newcommand{\hentpsbredde}[3]
{
   \begin{figure}[htb]
   \begin{centering}
      \mbox{\psfig{figure=/home/loke/a/ingvald/fag/pspros/rapport/figurer/#1.ps,width=#3\textwidth}}
      \caption{#2}
    \end{centering}
   \end{figure}
}

