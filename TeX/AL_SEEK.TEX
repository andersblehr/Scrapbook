%%%%%%%%%%%%%%%%%%%%%%%%%%%%%% -*- Mode: Latex -*- %%%%%%%%%%%%%%%%%%%%%%%%%%%%
%%% al_seek.tex --- 
%%% Author          : blehr
%%% Created On      : Sun Mar 28 02:25:13 1993
%%% Last Modified By: blehr
%%% Last Modified On: Tue Mar 30 05:55:00 1993
%%% RCS revision    : $Revision$ $Locker$
%%% Status          : In writing....
%%%%%%%%%%%%%%%%%%%%%%%%%%%%%%%%%%%%%%%%%%%%%%%%%%%%%%%%%%%%%%%%%%%%%%%%%%%%%%

\section{Finding a Pupil Point -- The Swimming Octopus}
\label{algo:seek}

In this section, the {\em swimming octopus\/} search algorithm is
described.  As pointed out in the previous section, the aim of the
algorithm is to locate a point in a given image that unambigously can
be classified as a pupil point.  The reason for the name is that the
performance of the algorithm can be compared to an octopus looking for
the sea, as will become clear below.  The basic idea behind the
algorithm is described in Section~\ref{algo:seek:idea}, whereas
Section~\ref{algo:seek:filter} suggests a pixel filter to be used by
it.  In Section~\ref{algo:seek:IQ}, the swimming octopus itself is
presented, and the overall search strategy is described in
Section~\ref{algo:seek:strategy}.  Throughout this section, {\em the
  algorithm\/} will be used to refer to the swimming octopus, and not
to the overall algorithm, {\octopus}.

\subsection{The Basic Idea}
\label{algo:seek:idea}

\inserteps{landscap}{\label{fig:landscape}An image of the eye depicted
  as a three-dimensional landscape.}

The foundation of the algorithm is a variant of thresholding in which
all pixels with values (gray levels) below or equal to the threshold
value $T_{l}$ have their old values replaced by $T_{l}$, and all
pixels with values above $T_{l}$ retain their old values.  The effect
of applying thresholding of this sort to an image can be illustrated
by picturing the image as a three-dimensional landscape, as depicted
in Fig.~\ref{fig:landscape}.  The negative peaks correspond to the
abovementioned spots, whereas the positive peak in the left of the
image represents an actual bright point (compare
Fig.~\ref{fig:testimages}(b)).  When the image is thresholded, $T_{l}$
can be viewed as representing the surface level of a mostly
subterranean lake underneath this landscape.  Wherever the landscape
descends below $T_{l}$, this lake comes to the surface, forming
``ponds'' whose sizes depend on the extent of the corresponding
sub-lake portion of the landscape.  This is illustrated in
Fig.~\ref{fig:lake}.  Note that, although the pupil is hardly
discernable in Fig~\ref{fig:landscape}, it is clearly visible as
forming the largest pond, or {\em lake\/}, after thresholding.  Note
also that a number of relatively large lakes are formed by non-pupil
portions of the eye.

The principle idea behind the search algorithm is that, bearing the
above notation in mind, every pixel in a given image can be classified
as being either {\em wet\/} or {\em dry\/}, wet pixels having values
below or equal to $T_{l}$, and dry pixels having values above $T_{l}$.
The advantage of this approach is that the darkest regions in the
image, of which the pupil is assumed to constitute the largest, are
assigned a unique value, $T_{l}$, which becomes {\em the\/} darkest
gray level of the thresholded image, whereas non-lake regions in the
image are left untouched.  Thus, sub-lake landscape features, which
characterize regions of which one is primarily interested in knowing
that they belong to the set of the darkest regions, are eliminated,
whereas the features of non-lake regions are preserved.  Using a wet
pixel as origin of operation (see Section~\ref{eval:approach:edge}),
for the moment not heeding whether or not this pixel is a pupil point,
it is clear that, in all directions, the first edge which is
encountered corresponds to either the {\em shore\/} of the lake of
which the pixel is a part, or to the shore of an {\em island\/} in
this lake.  Assuming that $T_{l}$ has been chosen so that the shores
of the {\em pupil lake\/} in an image correspond as closely as
possible to the actual pupil contour, and that the contrast of the
image is such that the pupil lake is relatively {\em deep\/}, thus
minimizing the probability of having islands in it, this approach
ensures minimum (zero) image noise between the origin of operation and
the pupil contour.

With images having as low contrast as the given test images,
particular care has to be taken when choosing $T_{l}$.  For the images
I have used during my thesis work, I found that $T=18$ ensures
``optimal'' pupil lakes and sufficiently small non-pupil lakes.
However, due to the low contrast, the probability of encountering
islands in the pupil lake is by no means ignorable.  $T=17$ tends to
make the image completely dry, whereas $T=19$ tends to ``drown'' large
portions of it.  It is evident that, if the final system were to
employ equally low-contrasted images, only a small change in light
conditions would destroy this extremely frail balance.  However,
employing a frame grabber supplying images whose gray levels occupy a
large portion of the desired gray level range (256) would by and large
eliminate this problem.

\inserteps{lake18}{\label{fig:lake}The image in
  Fig.~\protect\ref{fig:landscape} after thresholding with $T=18$.}

\subsection{A Possible Pixel Filter}
\label{algo:seek:filter}

From the above, it is clear that the first requirement that an actual
pupil point has to satisfy in order to be recognized as such, is that
it be wet.  However, as is seen from Fig.~\ref{fig:lake}, the number
of wet non-pupil points in an image cannot be assumed to be zero.
Thus, to avoid classifying a non-pupil point as a pupil point, some
sort of neighbourhood filtering has to be done.  In
Section~\ref{eval:eval:segment} it was stated that, by means of
carefully thresholding the value resulting from some sort of
neighbourhood averaging, it is possible to design a pixel filter which
only lets points through that are actual pupil points.  In the
following, the pixel filter employed by the search algorithm is
described.

The pupil lake is assumed to constitute the largest lake in an image
(if this is not the case, $T_{l}$ has been badly chosen).  Thus a point
can safely be classified as a pupil point if it is known to be part of
a lake whose size is in the order of the average pupil size.  More
precisely, the neighbourhood about the candidate pixel over which the
filtering is made must have a radius slightly smaller than the minimum
radius of the pupil.  

The pixel filter, hereafter referred to as the {\em octopus\/} (not to
be confused with {\octopus}, the name of the entire eye-tracking
algorithm), has the shape of a four-armed octopus viewed from above,
with its {\em arms\/} stretched in the {\em north\/}, {\em east\/},
{\em south\/} and {\em west\/} directions, respectively.  North is
defined to be upwards in the image.  See Fig.~\ref{fig:octopus}.  The
radius $r_{o}$ of the octopus (actually the length of its arms) is in
the current implementation set to be 80\% of the minimum pupil radius,
which, in the case of the given test images, is estimated to be
approximately 10\% of the horizontal image size.  The arms of the
octopus are viewed as consisting of {\em sensors\/}.  When applied at
a given location $(x,y)$ in an image, each sensor registers and
reports whether or not the pixel it covers is wet.  The {\em first
  swim-requirement\/} that has to be satisfied for the octopus to
report that it is (hence the name) {\em swimming\/}, that is, that it
has found the pupil lake, is that its central sensor, covering the
pixel at $(x,y)$, as well as its hands, that is, the outermost sensors
of each arm, report wetness.  If this requirement is satisfied, the
octopus starts to count how many of its sensors that report wetness.
If this number is higher than a given fraction $T_{s}$ of the total
number of sensors, the octopus is swimming and consequently reports
that the pixel at location $(x,y)$ represents a pupil point.  This
last requirement will be referred to as the {\em second
  swim-requirement\/}.

\insertepswidth{octopus}{\label{fig:octopus}The {\em octopus\/} pixel
  filter.  $r_{o}$ is its radius i pixels.}{0.33}

The probability that the point which the octopus reports to be a pupil
point actually is a non-pupil point depends on the threshold value
$T_{s}$.  With the current test images, which contain large and dark
non-pupil regions, and where the pupil is hardly discernable, I found
that the octopus returned no non-pupil points if $T_{s}$ was set to
0.8.  With better images, having higher contrast and not so large and
dark non-pupil regions, it ought to be possible to choose a lower
value for $T_{s}$ and still have the octopus return only pupil points,
particularly since it then would be hard finding a non-pupil point
satisfying the first swim-requirement, that the centre and extremities
of the octopus report wetness.

A last point which requires some elaboration, is that, with the given
test images, lower values for $T_{s}$ made the search algorithm run
faster.  This can be attributed to the poor image quality, and in
particular to the low contrast.  The average gray level of the pupil
lakes in the images was found to be 17, making the average depth of
any pupil lake 1 (given that $T_{l}=18$).  Thus local variations in
the pupil with positive amplitudes of 2 and more cause islands to
appear in the pupil lake.  Hence, if the octopus actually {\em is\/}
in the pupil sea, the probability that one or more of its arms cross
at least one island, thus reducing the number of sensors reporting
wetness, is relatively large.  Accordingly, the probability $p$ that
the second swim-requirement will not be satisfied at a given location
is higher than it would be if there were none or few islands in the
pupil lake.  Evidently, $p$ is inversly proportional to $T_{s}$.
Since, in addition, the number $n$ of points in the image that have to
be filtered in order to land at a point satisfying the
swim-requirements is inversely proportional to $p$, it follows that
$n$ is directly proportional to $T_{s}$.  Thus, when $T_{s}$ is
reduced, the time required to find a point satisfying the
swim-requirements is also reduced.  If the image quality were better,
however, there would seldom be islands in the pupil lake, and
accordingly a pupil point would be recognized as such more or less
independently of $T_{s}$.

\subsection{The ``Intelligent'' Octopus}
\label{algo:seek:IQ}

One approach to finding a pupil point would be to apply the octopus as
described above to every pixel in the image, say, starting in the
upper left corner (the reference point), until a pupil point had been
detected.  This would, however, not be a very time-efficient solution.
Another approach would be to define a grid covering the image and only
apply the octopus to the grid points, using some strategy as to the
order in which the points are subjected to filtering.  With this
approach, the time needed to detect a pupil point would depend on two
factors: The grid spacing, and the application strategy.  An even
better approach would be to incorporate into the octopus some sort of
mechanism indicating, from its current location, the most probable
direction in which to find the pupil lake.  This is the approach taken
in {\octopus}.

\subsubsection{Algorithmic Overview}

The principle of operation for the ``intelligent'' octopus can be
summarized as follows:
\begin{enumerate}
\item Define a subset of all the pixels in the image to form a set
  {\SS} of {\em starting pixels\/}.  The pixels belonging this set
  ought to be evenly distributed over the entire image, and their
  number should depend on the fraction of the image occupied by the
  pupil, not on the image resolution.  Mark all members of {\SS} {\sl
    not visited\/}.
\item Pick an unvisited starting pixel from {\SS} and mark its entry
  in {\SS} {\sl visited\/}.
\end{enumerate}

\subsubsection{Detailed Description}

\subsection{The Search Strategy}
\label{algo:seek:strategy}

\subsection{Number of Operations}
\label{algo:seek:O}

