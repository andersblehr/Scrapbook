%%%%%%%%%%%%%%%%%%%%%%%%%%%%%% -*- Mode: Latex -*- %%%%%%%%%%%%%%%%%%%%%%%%%%%%
%%% al_seek.tex --- 
%%% Author          : blehr
%%% Created On      : Sun Mar 28 02:25:13 1993
%%% Last Modified By: blehr
%%% Last Modified On: Mon Mar 29 06:41:11 1993
%%% RCS revision    : $Revision$ $Locker$
%%% Status          : In writing....
%%%%%%%%%%%%%%%%%%%%%%%%%%%%%%%%%%%%%%%%%%%%%%%%%%%%%%%%%%%%%%%%%%%%%%%%%%%%%%

\section{Finding a Pupil Point -- The Swimming Octopus}
\label{algo:seek}

In this section, the {\em swimming octopus\/} search algorithm is
described.  As pointed out in the previous section, the aim of the
algorithm is to locate a point in a given image that unambigously can
be classified as a pupil point.  The reason for the name is that the
performance of the algorithm can be compared to an octopus looking for
the sea, as will become clear below.  The basic idea behind the
algorithm is described in Section~\ref{algo:seek:idea}, whereas
Section~\ref{algo:seek:filter} suggests a pixel filter to be used by
it.  In Section~\ref{algo:seek:IQ}, the swimming octopus itself is
presented, and the overall search strategy is described in
Section~\ref{algo:seek:strategy}.  Throughout this section, {\em the
  algorithm\/} will be used to refer to the swimming octopus, and not
to the overall algorithm, {\octopus}.

\subsection{The Basic Idea}
\label{algo:seek:idea}

\inserteps{landscap}{\label{fig:landscape}An image of the eye depicted
  as a three-dimensional landscape.}

The foundation of the algorithm is a variant of thresholding in which
all pixels with values (gray levels) below or equal to the threshold
value $T$ have their old values replaced by $T$, and all pixels with
values above $T$ retain their old values.  The effect of applying
thresholding of this sort to an image can be illustrated by picturing
the image as a three-dimensional landscape, as depicted in
Fig.~\ref{fig:landscape} (compare Fig.~\ref{fig:testimages}(b)).  When
the image is thresholded, $T$ can be viewed as representing the
surface level of a mostly subterranean lake under this landscape.
Wherever the landscape descends below $T$, this lake comes to the
surface, forming ponds whose sizes depend on the extent of the
corresponding sub-lake portion of the landscape.  This is illustrated
in Fig.~\ref{fig:lake}.  Note that, although the pupil is hardly
discernable in Fig~\ref{fig:landscape}, it is clearly visible as
forming the largest pond, or {\em lake\/}, after thresholding.  Note
also that a number of relatively large lakes are formed by non-pupil
portions of the eye.  Thus the pixel filter has to be designed so that
only lakes having an extent approximately of the same order as the
average pixel radius are recognized as constituting pupil lakes.

The principle

With images having as low contrast as these, much care has to be taken
when choosing $T$.  For the images which I have used during my thesis
work, it was found that $T=18$ ensures relatively large pupil lakes
and sufficiently small non-pupil lakes.  $T=17$ tends to make the
image completely ``dry'', whereas $T=19$ tends to ``drown'' large
portions of it.  It is evident that, if the final system were to
employ equally low-contrast images, only a small change in light
conditions would destroy this extremely frail balance.  However,
employing a frame grabber supplying images whose gray levels occupy a
large portion of the desired gray level range (256) would by far
eliminate this problem.

\inserteps{lake18}{\label{fig:lake}The image in
  Fig.~\protect\ref{fig:landscape} after thresholding with $T=18$.}

\subsection{A Possible Pixel Filter}
\label{algo:seek:filter}

\subsection{The ``Intelligent'' Octopus}
\label{algo:seek:IQ}

\subsection{The Search Strategy}
\label{algo:seek:strategy}

