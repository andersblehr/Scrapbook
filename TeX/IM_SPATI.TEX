%%%%%%%%%%%%%%%%%%%%%%%%%%%%%% -*- Mode: Latex -*- %%%%%%%%%%%%%%%%%%%%%%%%%%%%
%%% im_spati.tex --- 
%%% Author          : blehr
%%% Created On      : Sun Mar 14 01:24:49 1993
%%% Last Modified By: blehr
%%% Last Modified On: Sun Apr 11 18:34:04 1993
%%% RCS revision    : $Revision$ $Locker$
%%% Status          : In writing....
%%%%%%%%%%%%%%%%%%%%%%%%%%%%%%%%%%%%%%%%%%%%%%%%%%%%%%%%%%%%%%%%%%%%%%%%%%%%%%

\section{The Spatial Domain}
\label{image:spatial}

The term {\em spatial domain\/} refers to the entire aggregate of
pixels making up an image.  Spatial domain methods are methods that
operate directly on these pixels.

\subsection{General Overview}
\label{image:spatial:overview}

An image in the spatial domain can be described by the two-dimensional
light-intensity function $f(x,y)$ whose value at location $(x,y)$
corresponds to the intensity (or {\em gray level\/}) of the image at
that point.  Image-processing functions in the spatial domain may be
expressed as
\begin{equation}
\label{eq:spatial_operator}
g(x,y)=T[f(x,y)]\mbox{,}
\end{equation}
where $g(x,y)$ is the processed image and $T$ is an operator on the
input image $f$, defined over some neighbourhood of $(x,y)$.  In other
words, the value of $g$ at $(x,y)$ is determined by the values of $f$
in this neighbourhood.

\insertepswidth{neighbor}{\label{fig:neighbour}A $3\times 3$
  neighbourhood about a point $(x,y)$ in an image.}{0.4}

The principle approach used in defining a neighbourhood about $(x,y)$
is to use an $n\times n$ subimage of $f$ centred at $(x,y)$, as shown
in Fig.~\ref{fig:neighbour}.  Other shapes are also possible, but
square arrays are by far the most predominant, due to their ease of
implementation.  The centre of the subimage is moved from pixel to
pixel, applying the operator at each location $(x,y)$ in $f$.  Thus it
is seen that applying a spatial operator to an $N\times N$ image
generally requires $O(N^{2})$ operations.

When the size $n$ of the square neighbourhood is 1, $T$ reduces to a
gray-level {\em transformation function\/} where the value of $g$ at
$(x,y)$ only depends on the value of $f$ at this location.  One widely
used gray-level transformation function is
\begin{equation}
\label{eq:threshold:1}
  g(x,y)=\left\{ \begin{array}{ll} 
                   1 & \mbox{if $f(x,y)\in Z$} \\ 
                   0 & \mbox{otherwise,}
                 \end{array} \right.
\end{equation}
where $Z$ is a defined set of distinct gray levels and 0 and 1 denote
black and white, respectively (typically represented by gray levels 0
and 255).  This is a general formulation of what is called {\em
  thresholding\/}, elaborated further upon in
Section~\ref{image:segment:threshold}.

\subsection{Spatial Masks}
\label{image:spatial:mask}

One of the principle approaches to the formulation in
Eq.~(\ref{eq:spatial_operator}) is the use of so-called {\em spatial
  masks\/}.  A spatial mask, or just {\em mask\/}, is a small
two-dimensional array ($m\times n$ or, more commonly, $n\times n$)
whose elements $w_{i}$ are coefficients used to compute $g$ at
$(x,y)$.  In Fig.~\ref{fig:mask}, a $3\times 3$ sub-image centred at
$(x,y)$ is shown, together with a $3\times 3$ mask with general
coefficients $w_{i}$.  When {\em convolving\/} the original image $f$
with this mask (cf.\ Section~\ref{image:frequency:convolution},
p.~\pageref{pg:convolutionmasks}), the value at location $(x,y)$ of
the resulting image $g$ is given by the relation
\begin{equation}
\label{eq:mask}
g(x,y)=T[f(x)]=w_{1}f(x-1,y-1)+w_{2}f(x,y-1)+\cdots+w_{9}f(x+1,y+1)\mbox{.}
\end{equation}
To compute the transformed image $g$, the mask is moved from pixel to
pixel in the same manner as for the neighbourhood operator shown in
Fig.~\ref{fig:neighbour}.  As pointed out above, if $f$ is of size
$N\times N$, the number of operations required to compute $g$ is
$O(N^{2})$.

The coefficients $w_{i}$ of the mask are chosen to detect given
properties in an image.  For example, if all $w_{i}$ are chosen to be
1/9, the gray level of each pixel in the resulting image $g$ would be
the average of the the gray levels of the corresponding pixel in $f$
and its 8 immediate neighbours ({\em 8-neighbours\/}).  This is known
as {\em neighbourhood averaging\/} (cf.\
Section~\ref{image:noise:averaging}), and the effect is one of {\em
  blurring\/} the original image.

Neighbourhood averaging is only one of several possible mask
operations.  By properly selecting the coefficients, it is possible to
perform a variety of useful image operations, among which are {\em
  noise reduction\/} (Section~\ref{image:noise}) and {\em edge
  detection\/} (Section~\ref{image:edge}).  It is worth noting,
however, that applying a mask at each pixel location in an image is a
computationally expensive task.  From Eq.~(\ref{eq:mask}), it is seen
that for $3\times 3$ masks, nine multiplications and eight additions
are required to compute $g$ at a given location $(x,y)$.  Thus,
convolving a $512\times 512$ image with a $3\times 3$ mask like the
one in Fig.~\ref{fig:mask}(b) requires a total of 2,359,296
multiplications and 2,097,152 additions.

\insertepswidth{mask}{\label{fig:mask}(a) Sub-area of image. (b) A
  $3\times 3$ mask with general coefficients.
  (From~\protect\cite{digim})}{0.8}
