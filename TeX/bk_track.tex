%%%%%%%%%%%%%%%%%%%%%%%%%%%%%% -*- Mode: Latex -*- %%%%%%%%%%%%%%%%%%%%%%%%%%%%
%%% bk_track.tex --- 
%%% Author          : blehr
%%% Created On      : Wed Mar  3 01:08:28 1993
%%% Last Modified By: blehr
%%% Last Modified On: Sun Apr 11 12:36:07 1993
%%% RCS revision    : $Revision: 1.7 $ $Locker:  $
%%% Status          : In writing
%%%%%%%%%%%%%%%%%%%%%%%%%%%%%%%%%%%%%%%%%%%%%%%%%%%%%%%%%%%%%%%%%%%%%%%%%%%%%%

\section{Methods for Measuring Eye Movement}
\label{back:track}

Methods for measuring eye movements fall into four main
categories~(\cite{monitor}): {\em Magnetic induction methods\/}, {\em
  photoelectric methods\/}, {\em video techniques\/} and methods
making use of the {\em optical properties\/} of the eye.

In this section, these categories are described, along with four
different systems, each falling into one of the categories.
Section~\ref{back:track:coil} presents the {\em search coil\/} and
{\em double magnetic induction\/} methods; in
Section~\ref{back:track:oculometer}, a system making use of
photoelectric principles is described;
Section~\ref{back:track:monitor} introduces the video-based {\em eye
  monitor\/}, and in Section~\ref{back:track:tv} an optical properties
method employing an infrared TV-camera is described.

All systems described require the head of the subject to be fixed.
With monkeys, this is less of a problem, since the head can be fixed
using a crown attached to the skull.  With human subjects, this
requirement is harder to satisfy completely, but good results have
been achieved using a bite board combined with a tightly fastened
crown on the head.

\subsection{Magnetic Induction Methods}
\label{back:track:coil}

The {\em search coil\/} technique was first proposed by D.\ A.\ 
Robinson in~\cite{robinson}.  In Robinson's setup, the subject wears a
short-circuited {\em induction coil\/} on a scleral contact lens, and
is placed in two perpendicular, rapidly alternating magnetic fields,
one horizontal and one vertical.  The voltage induced in the coil is
related to the position of the eye relative to the horizontal and
vertical fields, and can be measured using phase-sensitive detection
methods.  To facilitate the registration of the induced voltage, the
coil is equipped with a set of wires connected to the registering
equipment, thus making it rather cumbersome for the subject to wear.

A development of Robinson's method, the so-called {\em double magnetic
  induction method\/}, first proposed by J.\ Allik {\em et al.\/}
in~\cite{allik} and further developed by J.\ P.\ H.\ Reulen and L.\ 
Bakker in~\cite{reulen}, requires no physical connection between the
eye coil and the registering equipment.  The idea is to detect the eye
position indirectly by determining the strength of the induced
magnetic field of the coil through a {\em detection coil\/} placed in
front of the eye.

In~\cite{coil}, L.\ J.\ Bour {\em et al.\/} describe a further
improvement of the double magnetic induction method.  In their setup,
the amount of noise present is reduced through the use of a second
external coil, the {\em compensation coil\/}.  This coil is placed in
front of the second eye, and is used to cancel out remaining primary
induction voltages, since it is the secondary, induced voltage that is
of interest.  This is accomplished through differential amplification
of the signals from the detection and compensation coils, thus
sufficiently cancelling out the primary magnetic field.

\subsection{Photoelectric Methods}
\label{back:track:oculometer}

Photoelectric eye position measuring methods take advantage of the
fact that an image of the eye, when focused on a position sensitive
photodiode, induces photocurrents in the diode, which through
connectors at different points can be measured, and whose values
depend on the image intensities at points surrounding the connectors
(i.e., on the position of the pupil within the image).

In~\cite{oculometer}, M.\ Bach {\em et al.\/} describe a system which
measures gaze direction relative to head coordinates by analyzing the
centre of gravity (COG) of the first Purkinje image as well as of the
entire image of the eye viewed in infrared light.  In their setup, the
eye is illuminated by a number of infrared LEDs.  The LEDs are
arranged in an annulus around a lens which focuses the infrared image
of the eye at a magnification of about two on a position sensitive
photodiode.  The position is then measured via the COG of the image
falling on the diode.  The diode has 5 connections: 2 at opposing
edges ($x_{1},x_{2}$), 2 at the remaining edges ($y_{1},y_{2}$), and a
central one at the backside ($s$).  The $x$- and $y$-coordinates of
the COG are given by the partition of the photocurrents $I$ between
the electrodes:
\[x=\frac{I_{x_{1}}-I_{x_{2}}}{I_{s}}\] 
\[y=\frac{I_{y_{1}}-I_{y_{2}}}{I_{s}}\mbox{.}\] Although the position
of the COG depends in a complex way on gaze direction, it turns out
that the resultant voltage output of the system is a linear function
of gaze direction.

\subsection{Video-Based Methods}
\label{back:track:monitor}

Video-based eye tracking systems analyze the video signals from a
camera directly in order to determine the position of the pupil within
the image represented by the signals.  Video-based systems are widely
applicable and easy to use with untrained subjects.

In~\cite{monitor}, G.\ A.\ Myers, K.\ R.\ Sherman and L.\ Stark
present a system for real-time recording of eye movement, called the
{\em eye monitor\/}.  This system takes advantage of the ability of
video-based systems to measure the pupil (e.g., area, radius) as part
of the basic measurement.

The eye monitor is implemented almost entirely in hardware.  It
employs an infrared video camera whose pixel clock directly drives a
number of counters.  Each time the video signal is darker than an
adjustable threshold value, one of these counters, C1, is incremented,
and thus at the end of each video half frame, the contents of C1
correspond to the pupil area in pixels.  This value is divided by two
by shifting it one place to the right.  It is then loaded into another
counter, C2, and during the subsequent half frame, C2 is decremented
each time C1 is incremented, so that its contents equal zero when the
current video line passes through the equator of the thresholded
pupil.  The vertical centre of the pupil then corresponds to the
number of horizontal lines processed.  The horizontal centre of the
pupil as well as the horizontal and vertical centres of the first
Purkinje image are found in conceptually similar manners.  The centre
of the first Purkinje image is needed to make corrections for small
head movements, as the difference between the pupil centre and that of
the first Purkinje image is a first-order invariant with respect to
translational head movements.

Since the edges of the video field tend to be dark, they could be
mistaken to be parts of the pupil.  To avoid this, a circular {\em
  tracking window\/} centred on the current estimate of the pupil
position disables the pupil-detection hardware outside the window.
The position estimate is updated every half frame, and the area
(radius) of the window is also updated to correspond to that of the
pupil.

All in all, the eye monitor can be said to employ an internal model of
the eye, comprising most of what is known about what the eye looks
like to a video camera.  The shape of the tracking window (circular)
corresponds to the shape of the pupil, the location and size of the
window is continuously updated to conform with the last known location
and size of the pupil, and the pupil is detected using {\em area
  detection\/}, emphasizing the low-frequency characteristics of the
eye's video image, which are less sensitive to noise.

\subsection{Optical Properties Methods}
\label{back:track:tv}

Although the aforementioned methods and systems all take advantage of
the optical properties of the eye in determining the centre of the
pupil, they have been associated with other categories of systems,
corresponding to the manner in which these properties are detected.

This last category really denotes a collection of techniques which all
differ from the previous ones in that they perform the analysis in two
completely separate steps, of which the first is common for all.  The
first step consists of acquiring and possibly digitizing and storing
({\em grabbing\/}) a complete image of the eye as it appears visually.
During the next step, the acquired image is analyzed and the position
of the pupil is determined using methods which differ from one system
to another.  By far the most widely used method during the second step
is {\em image processing\/}.  In Chapter~\ref{image}, some well-known
image processing techniques are presented.  Here another approach is
described, reported by S.\ Nagao in~\cite{tv}.

Nagao describes a system for real-time eye-tracking making use of an
infrared TV-camera.  The eye is illuminated by infrared LEDs, and the
camera is used to deliver an image of the eye to a high resolution TV
monitor.  Two narrow windows, one horizontal and one vertical, are set
on the TV screen, covering the margins of the moving pupil
(Fig.~\ref{fig:back:track:tv}).  Positions of the left-right and
upper-lower margins of the pupil within the two windows are determined
by the brightness contrast between the pupil and the iris by means of
a slicer device.  Due to the camera delivering only half frames, the
average left and right margins ($x_{1}$ and $x_{2}$, respectively) of
the pupil within the horizontal window are measured twice as often as
the upper and lower margins ($y_{1}$ and $y_{2}$, respectively) within
the vertical window.  The centre position of the pupil within the TV
screen is given by the relations
\[x=\frac{1}{2}(x_{1}+x_{2})\] \[y=\frac{1}{2}(y_{1}+y_{2})\mbox{.}\]

\insertpdfwidth{tv}{\label{fig:back:track:tv}The principle of the
  infrared TV-camera method.  {\sf v\/} and {\sf h\/} designate the
  vertical and horizontal windows, respectively, and $(x_{1},x_{2})$
  and $(y_{1},y_{2})$ denote the detected margins of the pupil.}{0.45}

\subsection{Summary}
\label{back:track:summary}

The systems discussed in this section each belong to one of the four
categories of eye position measuring systems listed in the preamble.
Their advantages and disadvantages correspond in large to those of the
categories they represent.  In this section, an overview is given of
the positive and negative aspects of the different approaches in terms
of the requirements specified in Section~\ref{intro:motivation}, and a
conclusion is drawn as to which category seems to suit the present
needs at best.

\subsubsection{Magnetic Induction Methods}

The two magnetic induction methods described, the search coil
method~(\cite{robinson}) and double magnetic
induction~(\cite{allik},~\cite{reulen},~\cite{coil}), display
approximately equal properties when it comes to accuracy and real-time
abilities.  L.\ J.\ Bour {\em et al.\/} conclude in~\cite{coil} that
the double induction method, as developed by them, is {\em ``useful
  for the accurate measurement of eye movement in both man and
  monkey''\/}.  The linear range of this system is $\pm 10^{\circ}$,
and the total measuring range is $\pm 35^{\circ}$.  The calculated
error of the system is $\pm 0.1^{\circ}$, and the eye position is
measured with a rate of 500 samples/s, both of which satisfy the
given requirements.

Of the two systems, the latter, especially as developed by L.\ J.\ 
Bour {\em et al.\/} in~\cite{coil}, is the least discomforting for the
subject, in that there are no connecting wires between the search coil
and the registering equipment.  Still, the subject is required to wear
a metallic ring on the eye, causing discomfort and stress.  Another
negative aspect, and the most important one in our case, is that the
alternating magnetic fields would induce unwanted electric currents in
the microelectrodes used to perform recordings from the visual cortex
of the monkeys and in the EEG electrodes.  A last problem with these
systems is their price, being in the order of DM 50,000, which by far
exceeds the available means.

\subsubsection{Photoelectric Methods}

The resolution of the infrared oculometer of M.\ Bach {\em et
  al.\/}~(\cite{oculometer}) is indicated to be $\pm 0.3^{\circ}$ in
the worst case.  The highest measuring rate of the oculometer is not
reported explicitly, but rates of 85 Hz are mentioned.  Both
resolution and measuring rate satisfy the requirements.  The linear
range of the system is reported to be greater than $\pm 20^{\circ}$,
which is twice the range of the double magnetic induction method.

An advantage of photoelectric methods as compared to magnetic
induction methods is that they require no surgery, and thus do not
interfere directly with the eye.  The main disadvantage of the method
is that it has to be implemented in hardware, thus leaving little room
for adaption to specific needs.  Another problem is that most
photoelectric systems require time-consuming calibrating, which is
not desirable in our case.

As for the magnetic induction systems, the cost of the systems
employing photoelectric techniques is prohibitive, being in the order
of DM 40,000.

\subsubsection{Video-Based Systems}

The resolution of the eye monitor of G.\ A.\ Myers {\em et
  al.\/}~(\cite{monitor}) is $\pm 0.2^{\circ}$ within a range of $\pm
20^{\circ}$.  The measuring rate of the system, as of any video system,
is bound to the video refresh rate of the camera.  Good cameras
operate with refresh rates in the order of 50--60 half frames per
second (thus reducing by a factor of 2 the vertical resolution of the
system as compared to a system employing a camera delivering entire
frames), the camera employed in~\cite{monitor} delivering
approximately 60 half frames per second.  Thus both resolution and
measuring rate satisfy the stated requirements.

The main disadvantages with video-based systems are, as was the case
for photoelectric methods, for one their price, and secondly that they
are completely hardware dependent, thus preventing customization to
specific needs.

\subsubsection{Optical Properties Methods}

The measuring rate of eye tracking systems belonging to this category
is limited by (1) the refresh rate of the camera used, (2) the
grabbing-time of an eventual frame grabber, and (3) the computing time
of the position-analyzing algorithm.

The system described by S.\ Nagao in~\cite{tv} employs a camera
delivering one half frame every 33.3 ms, needs no frame grabbing, as
the image is analyzed directly as it appears on the TV screen, and the
computing time of the position-analyzing device is constant in time.
The measuring rate of the system is thus limited to 30 Hz, as given by
the refresh rate of the camera.  The spatial resolution of the system
depends on the power of the TV-camera employed, and is reported to be
$\pm 0.025^{\circ}$ in the given setup, which adheres to the specified
requirements.  The measuring rate of the setup, on the other hand, is
too slow for our purpose, but constitutes no real problem, in that it
can be improved by choosing a camera with higher refresh rate.

The primary advantage of this system is the superfluousness of time
consuming frame grabbing, making the entire time interval between half
frames available for the position analyzing device.  Thus its only
time limiting aspect is the refresh rate of the camera, which can be
chosen more or less freely.  The need to perform the position-analysis
directly on the TV screen, on the other hand, does not give an
impression of elegance, and the rigidness of the position-analyzing
device constitutes another drawback of the system, making it
unsuitable for our purpose.

\subsubsection{Conclusion}

As is evident from the above paragraphs, none of the presented systems
fulfil all the stated requirements, their costs being the most
prohibitive factor.  Magnetic induction methods are unsuited because
the magnetic fields they employ induce unwanted currents in the
experimental registering equipment.  Photoelectric methods as well as
video-based methods were excluded from the set of possible candidates
because of their strict hardware nature, violating the requirement of
configurability.  The only category which has not been not completely
excluded, although the one system belonging to it was, is the category
of optical properties methods.

The big advantage of image processing based optical properties systems
as opposed to the approach used by Nagao in his system system is that
they, being implemented mainly in software, offer easy customization
and adaption to specific needs.  In fact, the more that is implemented
in software, the easier the customization of the system.  One
disadvantage of this kind of system, is that the video frames have to
be grabbed before they can be analyzed, this constituting a
competitive factor on valuable processor time.  Having two or more
processors running in parallel (e.g., Transputers), one of which being
responsible for grabbing the frames, would pretty much eliminate this
problem.

There already exists at least one commercially available and good eye
tracking system based on a video camera delivering video frames to a
computer performing the position-analysis~(\cite{antrag}).  Its price,
though, is prohibitive ($\sim$DM 150,000), due to old and therefore
expensive technology being employed.  The simplicity of this system,
on the other hand, suggests that it ought to be possible to develop a
similar system for only a fraction of the costs, employing new and
cheaper technology.

Based on this, it was decided that an entirely new image processing
and software based eye tracking system should be developed, adhering
as closely as possible to the requirements listed in
Section~\ref{intro:motivation}. 
