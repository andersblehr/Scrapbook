% >>> Missing from original .ZIP, reconstructed Dec 2016 <<<

\section{Synchronization and Cognition}
\label{intro:synch}

{\em Synchronization and Cognition} (SC) is a cooperation project
initiated by R. Eckhorn and R. Bauer of the Biophysics Group and
F. R{\"o}sler of the Physiological Psychology Group at the
Philipps-Universit{\"a}t Marburg, Germany.

The basis of the current work is observations done by R. Eckhorn and
R. Bauer (\cite{cat}), as well as by the group of W. Singer (e.g.,
\cite{singer2} and \cite{singer1}), that when stimulated, distributed
neural assemblies organize rhythmic activities in peripheral visual
areas in the brain of anesthetized cats.  The phase coupling among
these assemblies, but not their frequencies, depend on the applied
visual stimulus.

The goal of the SC project is, through the use of microelectrodes
(monkeys; {\em makaba mulatta}) and EEG recordings (humans), to show
that coherent neurological activations ({\em synchronizations}) in the
visual cortex, relating to coherent visual stimuli, is a requirement
for the coherent perception of these stimuli.  In addition, the
time-space {\em synchronization dynamics} in peripheral and central as
well as in intermediate cortex areas are to be investigated as to how
they relate to the cognitive tasks being performed.

For a further description of the SC project, see \cite{antrag}.  Some
of the publications by R. Eckhorn {\em et al.} concerning the
discovery and implications of synchronizations in the visual cortex of
cats and monkeys are \cite{eckhorn1}, \cite{eckhorn2}, \cite{cat} and
\cite{eckhorn3}.

\subsubsection{Note}

During my thesis work, I have only interacted with the group of
R. Eckhorn and R. Bauer.  Accordingly, I will in the following refer
to the {\em monkey} and the {\em subject} interchangeably.  The
general principles, however, when referring to the monkey, can be
assumed also to apply to the work done by the group of F. R{\"o}sler.
