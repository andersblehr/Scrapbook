%%%%%%%%%%%%%%%%%%%%%%%%%%%%%% -*- Mode: Latex -*- %%%%%%%%%%%%%%%%%%%%%%%%%%%%
%%% nash_shell.tex --- 
%%% Author          : Anders Blehr
%%% Created On      : Mon Apr 27 17:29:40 1992
%%% Last Modified By: Anders Blehr
%%% Last Modified On: Mon May  4 06:18:02 1992
%%% RCS revision    : $Revision: 1.1 $ $Locker:  $
%%% Status          : Unknown, Use with caution!
%%%%%%%%%%%%%%%%%%%%%%%%%%%%%%%%%%%%%%%%%%%%%%%%%%%%%%%%%%%%%%%%%%%%%%%%%%%%%%

\section{The Expert System Shell}
\label{shell}

As mentioned in Section~\ref{aims}, the aim of {\nash} has been to
make a working prototype of a system containing the basic features of
an expert system shell, and of which the dialogue is entirely based on
{\nal}. {\nal}, both the language and the analyzer, have been
thoroughly treated in Section~\ref{def}. In this section, the
reasoning system employed in {\nash} is described.

\subsection{Asserting Facts and Rules}
\label{assert}

The internal database of {\nash} contains two categories of knowledge:
{\em facts\/}, that is, declarative statements about the world, and
{\em rules\/}.

After a sentence supplied by the user has been transformed into the
corresponding {\niks}-representation by the analyzer, this
representation is passed on to the {\em evaluator\/}, which, depending
on the category the sentence belongs to (statement, rule or query)
decides what to do next. In this section we look at how statements and
rules are treated by the evaluator, whereas the treatment of queries
is described separately in Section~\ref{answer}.

\subsubsection{Asserting Rules}

Asserting rules is straight-forward in {\nash}: they are explicitly
stored in the knowledge base in their given form without any further
treatment.

\subsubsection{Asserting Facts}

Asserting facts could have been treated similarly to the assertion of
rules, that is, to be stored explicitly without further delay. But,
since facts as supplied by the analyzer may contain large amounts of
implicit knowledge which may be cumbersome to extract later as part of
the reasoning process, I found it convenient to try to extract as much
of this knowledge as possible at an as early stage as possible, and to
store each piece of knowledge thus extracted separately.

To illustrate the approach, consider the sentence

\begin{tabbing}
  xxx\= xxx\= \kill
  \>\>{\em John lives in a beautiful house in France\/}
\end{tabbing}
As is evident, this sentence indeed conveys that John lives in a
beautiful house in France, but it also conveys a lot more. For
instance that John lives in France, or that there is at least one
beautiful house in France. To extract this knowledge, we can use the
fact that it is contained in {\em nested prepositional phrases\/} to
recursively go through the semantic structure representing the
sentence, collecting noun phrases (like, in this case, John and the
beautiful house) and combining them with the prepositional phrases
from whose scope they are visible. Thus it is extracted that John
lives in France, but not that France is in a beautiful house. To
complete the example, these are the pieces of knowledge extracted from
the given sentence by {\nash}, presented in paraphrased form:

\begin{tabbing}
  xxx\= xxx\= \kill
  \>\>{\em John lives in a beautiful house in France.\/}\\
  \>\>{\em John lives in a beautiful house.\/}\\
  \>\>{\em A beautiful house is in France.\/}\\
  \>\>{\em John lives in France.\/}\\
  \>\>{\em John lives.\/}
\end{tabbing}
From a mere syntactic point of view, as that of {\nash}, this is about
all that is possible to deduce about John and his beautiful house in
France. Knowledge made up from nouns and their qualifiers is not
extracted. This is because the combinations of nouns and qualifiers
are considered to be separate entities, not variants of the same
entity.  Thus, in the above example, it is not deduced that ``{\em A
house is beautiful\/}''.

It has to be pointed out, though, that the algorithm
is not robust, in that it misses pieces of knowledge contained in
certain complex nested constructs. 

\subsection{Answering to Queries}
\label{answer}

In the following discussion, the term {\em literal\/} means a
constituent part of a conjugation, and the verb to {\em solve\/} is
used to denote the process of proving a given literal.

The reasoning mechanism of {\nash} works analogously to the resolution
principle described in Section~\ref{resolution}. Whereas resolution
requires that the clause to be solved be a disjunction of literals,
the mechanism employed in {\nash} requires that the clause be a
conjunction of literals. Moreover, implications (rules, that is) need
not be converted to the analogue of clause form.

To supply an answer to a query, it is decomposed into a set of
literals, where each literal has to be solved separately.  Whenever
one of the literals in the set has been solved, it is removed from the
set, and thus the entire query being solved corresponds to each of its
constituent literals being solved (this is analogous to a conjunction
being true only if all of its constituents are true).

Simple literals may be solved in a number of ways:

\begin{itemize}
\item If the literal contains no variables and is contained in the
  knowledge base, or a literal is contained in the knowledge base
  which can be unified with the given literal, this literal has been
  solved.
\item If a rule is found whose right side can be unified with the
  given literal, the rule is fired, which means that the left side of
  the rule is substituted for the literal, and thus has to be solved
  in order for the literal to be solved.
\item If a rule is found whose right side constitutes is a
  conjugation of literals, one of which can be unified with the given
  literal, the rule is fired, and the process continues as
  above\footnote{This is logically justified as follows. The given
  rule is expressed as \(P\rightarrow Q_1\wedge Q_2\wedge\cdots\wedge
  Q_{i}\wedge\cdots\wedge Q_{n}\mbox{,}\) which is equivalent to
  \((\neg P\vee Q_1)\wedge(\neg P\vee Q_2)\wedge\cdots\wedge(\neg
  P\vee Q_{i})\wedge\cdots\wedge(\neg P\vee Q_{n})\). In other words,
  \(\neg P\vee Q_{i}\), which is equivalent to \(P\rightarrow Q_{i}\),
  logically follows from the initial rule.}.
\item If the leftmost argument of a prepositional literal which
  cannot be solved as it is, is constituted by a non-atomic literal,
  an attempt is made at solving this non-atomic literal. If the
  attempt succeeds, the literal is recombined with the enclosing
  expression, and a new attempt is made at solving it. 
\item If neither of the above stages have caused the literal to be
  solved, the user is prompted for the information needed for a
  solution to be arrived at. If the information thus acquired is
  insufficient, failure is reported and the search process terminates.
\end{itemize}
If the query has been solved, the answer it supplies depends on the
nature of the query. That is, a confirmative query (see
Section~\ref{naldef}) supplies a ``{\em yes\/}'' or ``{\em no\/}''
answer, whereas all other queries (excluding ``{\em how\/}''-queries)
corresponds to the user wanting to know the {\em name\/} of a certain
individual (instance) of a class of world objects, and thus this name
constitutes the answer given.

\subsection{Collecting Proofs}
\label{proof}

Proofs are collected during the reasoning process. Each time a literal
has been solved, or a rule has been fired, it (the solved literal or
the rule) is added to a list which is passed along through each step
of the reasoning process. When the initial query has been solved, this
list constitutes a proof as to how the conclusion was arrived at, and
will be presented to the user on request (that is, if the query was
prefixed with the keyword ``{\em how\/}'', see Section~\ref{naldef}).

\subsection{Cooperativity Issues (Not Implemented)}
\label{notcoop}

As mentioned in Section~\ref{coop}, Grice' Cooperation Principle has
become widely accepted as a model on which to base intelligent
(cooperative) interfaces to {\nlp}-systems. Several approaches have
been proposed on how to implement cooperative interfaces. In this
section, two of these approaches, {\em integrity constraints\/} and
{\em scripts analysis\/}, are briefly touched upon. In {\nash}, no
measures have been taken in order to make its interface cooperative.

\subsubsection{Integrity Constraints}

In~\cite{gal}, an approach is presented which is based on detecting
disparities between the knowledge base (or database) and the
assumptions the user makes about it. 

Different users of the database may have different perceptions (or
{\em viewpoints\/}) of it, which can be described by a set of {\em
intensional\/} laws (that is, laws used to derive new facts from what
is explicitly stored in the database, such as ``{\em 1848 is before
1989\/}''. These laws can also be used to express semantic
restrictions, or {\em integrity constraints\/} associated with the
database. An example of an integrity constraint would be that one
person cannot at the same time both be a mother and a father.

The {\em presupposition\/} of a query can be loosely defined as all
the more general statements resulting from the query. For example, the
query ``{\em How many people took the train to Haugesund in 1991?\/}''
presupposes that there exists a railway connection to Haugesund. When
a very general presupposition is false, all the presuppositions less
general than it are false also.

A {\em misconception\/} occurs between a user and a database when no
answer to the user's query exists because either a presupposition of
the query is found to be false, or one ore more integrity constraints
are violated. Several misconception may occur in connection with the
same query. In~\cite{gal}, several rules are listed to guide the
system in the selection of which information to present to the user as
a response to a query containing one or more misconceptions.

For a thorough description of the approach, as well as a Prolog
implementation of parts of it, see~\cite{gal}.

\subsubsection{Scripts Analysis}

A {\em script\/} may be defined as a stereotyped description of a
situation which may occur within a certain domain, say, the act of
going to a restaurant, where each entity is assigned a particular
r\^{o}le, the waiters serving food, the guests entering, ordering,
eating, paying and leaving. A script consists of a set of {\em
slots\/}, which represent people and objects that are involved in the
events described in the script. A script has associated with it a {\em
header\/}, which describes the prerequisites for its being activated,
as well as its preferred locations, r\^{o}les and events. One of the
most important features of scripts, is their ability to predict events
which have not been explicitly observed.

The main use of scripts has been in the field of {\em plan
recognition\/}, but they have inherent features, such as their ability
to detect the most likely course of events, which make them well
suited as a basis for cooperative interfaces. For a more thorough
treatment of scripts, see for instance~\cite{torulf} and~\cite{rich}.

\subsection{Limitations}
\label{limsys}

When evaluating {\nash}, it is important to keep in mind that it, like
{\nal}, is no more than a prototype, and should thus be expected
neither to be complete nor sound. Many of its main limitations are of
an implementational nature, in that the system as of yet is not fully
implemented. Neither has it undergone thorough and error-seeking
testing. Rather, its test set has been a limited set of rules and
declarative statements, and errors have been corrected which were
discovered using this set. Some of its limitations can be summarized
as follows (see also Section~\ref{limnal}):

\begin{itemize}
\item Each time the system is started, its knowledge base is empty.
  This means that the user has to, by hand, supply it with the
  necessary rules and facts. This is not satisfactory under any
  circumstances, and particularly not when the set of rules and facts
  constituting the given domain becomes large.
\item The user may not ask {\em why\/} he is asked a question. It is
  desirable that he be able to do so, as a means to establish his
  confidence in the system.
\item When the system has arrived at an answer to a query, it is
  presented as a single-word response, not as a complete sentence.
  This is not a major issue, though, but it would be desirable from
  the user's point of view that full-sentence responses be given.
\item The set of separate pieces of knowledge derived from a
  declarative statement is not complete (see Section~\ref{assert}).
\item When trying to prove a query on the basis of a set of rules,
  the system fires a rule and then loops forever if it is not able to
  prove the query given that particular rule. This is caused by the
  reasoning mechanism not being equipped with a counter ensuring that
  failure eventually is reported if the set of unsolved literals does
  not become empty (see Section~\ref{answer}). A minor bug to fix,
  though.
\item The system is not reliable, neither is its behaviour
  predictable. Depending on the structure of the constructs
  representing the query to be solved, the system may or may not
  arrive at a solution. Most of the time it does, though (arrive at a
  solution, that is).
\end{itemize}
There are other limitations as well, mostly due to implementational
incompleteness, which are not elaborated upon here.
