%%%%%%%%%%%%%%%%%%%%%%%%%%%%%% -*- Mode: Latex -*- %%%%%%%%%%%%%%%%%%%%%%%%%%%%
%%% introduction.tex --- 
%%% Author          : Anders Blehr
%%% Created On      : Thu Apr 23 14:50:59 1992
%%% Last Modified By: Anders Blehr
%%% Last Modified On: Mon May  4 09:12:43 1992
%%% RCS revision    : $Revision: 1.1 $ $Locker:  $
%%% Status          : In progress
%%%%%%%%%%%%%%%%%%%%%%%%%%%%%%%%%%%%%%%%%%%%%%%%%%%%%%%%%%%%%%%%%%%%%%%%%%%%%%

\chapter{Introduction}
\pagenumbering{arabic}
\label{intro}

\section{Motivation}
\label{tucc}

{\em The Understanding Computer\/} ({\tuc}) is a project initiated by
Tore Amble at the Knowledge Systems Group of The Division of Computer
Science \& Telematics (IDT) at The Norwegian Institute of Technology
({\nit}), Trondheim, Norway.

{\tuc} is a successor of the Nordic cooperation project {\hsql}, which
was a prototype for answering queries to a relational database, given
in any of the Scandinavian languages.

The goal of the project is to investigate {\em ``how a naturally
readable language can be a unifying framework for knowledge
representation and natural language interfaces, {\em [that is,
whether]} natural language is able to express knowledge in a way that,
apart from its understandability, also is sufficient for an automated
reasoning system to draw the right conclusions''} (\cite{solon}).

In connection with {\tuc}, Tore Amble has developed a logical language
{\solon} ({\em Second Order Logic for Natural Language\/}), which
allows flexibility with respect to the number and types of arguments
and which eliminates as many variables as possible from expressions.
For a further description of {\tuc} and its foundations,
see~\cite{solon}.

Initially, it was the intention that I should work directly with
{\tuc}, but given its complexity and the limited time I had available,
it was decided that I should develop an independent system, and that
Tore Amble, who has implemented the current version of {\tuc}, and who
doubtlessly has the most in-depth knowledge of it, incorporate those
parts of my work he sees fit into it.

\section{Structure of the Paper}

In Chapter~\ref{background}, an introduction to natural languages and
natural language systems is given, as well as a description of rule
based systems and brief introduction to predicate logic. In
Chapter~\ref{nash}, my system, {\nash}, a natural language based
expert system shell, is described. In Chapter~\ref{concl} the
conclusions I have drawn while working with the project are presented.

\section{The Project Environment}

All computational work was done using the {\em SPARC\/} and {\em HP\/}
workstations at IDT.

{\nash} was implemented solely in {\SICStus}, a {\em WAM\/}-based
implementation of Prolog made at The Swedish Institute of Computer
Science, Kista, Sweden (see~\cite{sicstus}).

This document was prepared with Leslie Lamport's {\LaTeX} document
preparation system (see~\cite{lamport}) and Donald Knuth's {\TeX}
typesetting system on which {\LaTeX} is based.

Lastly, I would like to thank my supervisor Tore Amble for his
enthusiasm and inspiring guidance.
