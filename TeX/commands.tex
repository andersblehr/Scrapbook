%%%%%%%%%%%%%%%%%%%%%%%%%%%% -*- Mode: Latex -*- %%%%%%%%%%%%%%%%%%%%%%%%%%%%%%
%%% commands.tex --- 
%%% Author          : Anders Blehr
%%% Created On      : Sat Apr 25 17:54:59 1992
%%% Last Modified By: blehr
%%% Last Modified On: Sun Apr  4 18:42:33 1993
%%% RCS revision    : $Revision: 1.12 $ $Locker:  $
%%% Status          : Subjected to continuous change....
%%%%%%%%%%%%%%%%%%%%%%%%%%%%%%%%%%%%%%%%%%%%%%%%%%%%%%%%%%%%%%%%%%%%%%%%%%%%%%
%%% History 		
%%% 20-Mar-1993		blehr	
%%%    La til \inserttwoeps, som setter inn to EPS-filer ved siden av
%%%    hverandre, kaller dem (a) hhv. (b), og gir dem en felles
%%%    \caption.  Skiller ogsaa mellom mine og Anders' saker - Anders'
%%%    er plassert nederst i filen.
%%% 23-Feb-1993		blehr	
%%%    Tok bort \hentps og \henpsbredde, siden dvips ikke forstaar
%%%    psfig.  La til \inserteps og \insertepswidth istedet.  Begge
%%%    disse krever Encapsulated PostScript (.eps).

%%
%%  Mine saker
%%

\renewcommand{\SS}{\bf S}
\newcommand{\octopus}{OCTOPUS}
\newcommand{\odd}{\mbox{\scriptsize odd}}
\newcommand{\even}{\mbox{\scriptsize even}}
\newcommand{\fft}{FFT}
\newcommand{\nit}{NTH}

%
% Kommandoer for importering av EPS-filer (Encapsulated PostScript).
%

%
% \insertpdf[2]
%
% Bruksmaate:
%   \insertpdf{filnavn}{\label{fig:labref}figurtekst}
%
% X-utstrekningen til figuren settes til 0.7 ganger tekstbredden.
%

\newcommand{\insertpdf}[2]{\insertpdfwidth{#1}{#2}{0.7}}

%
% \insertpdfwidth[3]
%
% Bruksmaate:
%   \insertpdfwidth{filnavn}{\label{fig:labref}figurtekst}{bredde}
%
% `bredde' er et tall mellom 0 og 1 som angir broekdel av
% tekstbredden som figuren skal ha.
%

\newcommand{\insertpdfwidth}[3] {
  \begin{figure}[tb]
    \centering
    \includegraphics[width=#3\textwidth]{figurer/#1.pdf}
    \caption{#2}
  \end{figure}
}

%%
%%  Torp's saker
%%

% Et tastetrykk, eller en kombinasjon av to taster.
\newcommand{\tast}[1]{\fbox{\rule{0mm}{1.5ex}{\tt #1}}}
\newcommand{\totast}[2]{\tast{#1}-\tast{#2}}

% Binder en taste-sekvens mot en kommando.
\newcommand{\keybinding}[2]{#1 executes {\em #2}. See Section~\ref{#2-func}.}

% Kommando som definerer ny funksjon.
% Parameter 1 : Funksjonsnavn, Kommer til venstre under en strek.
%           2 : Parametre. Kommer til h|yre for navnet i kursiv
%           3 : Command|Function|Variable    Kommer til hoyre.
%           4 : Forklaringen. Den quote's.

% \newenvironment{params}{{\large Parameters:}\begin{description}}{\end{description}}

\newcommand{\returns}[1]{
        \begin{description}
          \item [Returns : ]#1
        \end{description}}

% Dette er environmentet som brukes ved forklaring av funksjoner,
% programmer eller variable.
\newenvironment{func}[3]{\pagebreak[3] \label{#1-func} %
  {\large\it  #1\hspace{1cm} #2 \hfill {\bf #3}}
\begin{quote}}{\end{quote}}
