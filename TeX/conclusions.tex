%%%%%%%%%%%%%%%%%%%%%%%%%%%%%% -*- Mode: Latex -*- %%%%%%%%%%%%%%%%%%%%%%%%%%%%
%%% conclusions.tex --- 
%%% Author          : Anders Blehr
%%% Created On      : Mon May  4 03:03:46 1992
%%% Last Modified By: Anders Blehr
%%% Last Modified On: Mon May  4 09:14:01 1992
%%% RCS revision    : $Revision: 1.1 $ $Locker:  $
%%% Status          : Unknown, Use with caution!
%%%%%%%%%%%%%%%%%%%%%%%%%%%%%%%%%%%%%%%%%%%%%%%%%%%%%%%%%%%%%%%%%%%%%%%%%%%%%%

\chapter{Conclusion}
\label{concl}

In working with this project, my aim has been to investigate some of
the problems pertaining to the field of natural language processing,
and to develop a suitable platform on which to base the implementation
of a natural language based expert system shell. 

\section{Evaluation}

The implementation, as of today, is by no means complete, as discussed
in previous chapters. As one of the prerequisites of my work has been
that it should take place within the framework of {\tuc}, it is of
importance that the system serves a purpose within this framework.
Whereas one of the goals with {\tuc} is that it shall demonstrate
semantic as well as temporal features, my implementation contains no
such features. This is obviously a drawback. On the other hand, {\tuc}
is very restrictive with respect to the {\em meanings\/} assigned to
particular words, in that the notion of {\em situations\/} is defined
in terms of them. This does not correspond very well with natural
languages as used in daily life, where different words assume
different meanings depending on the context in which they appear. In
my implementation, words have no meanings associated with them, and
may thus be used differently in different contexts. A desirable
solution would be a combination of these two approaches, featuring
both the semantic powers of {\tuc} as well as the expressive powers
resulting from flexible use of words. Thus I hope that my system will
serve a purpose as a contributor to the continued development of
{\tuc}.

\section{Summary}

In working with this project, I have had the opportunity to delve
deeper into what I find to be one of the most interesting as well as
challenging fields of computer science. I have found out that my {\em
a priori\/} interest in the field of natural languages and natural
language processing indeed was more than a mere superficial fancy, and
I hope that I in the future will get the opportunity to continue
working with some of the questions I found interesting during the
work.
