%%%%%%%%%%%%%%%%%%%%%%%%%%%%%% -*- Mode: Latex -*- %%%%%%%%%%%%%%%%%%%%%%%%%%%%
%%% back_rbs.tex --- 
%%% Author          : Anders Blehr
%%% Created On      : Sat Apr 25 20:00:15 1992
%%% Last Modified By: Anders Blehr
%%% Last Modified On: Mon May  4 06:01:54 1992
%%% RCS revision    : $Revision: 1.1 $ $Locker:  $
%%% Status          : Unknown, Use with caution!
%%%%%%%%%%%%%%%%%%%%%%%%%%%%%%%%%%%%%%%%%%%%%%%%%%%%%%%%%%%%%%%%%%%%%%%%%%%%%%

\section{Rule Based Systems}
\label{rbs}

When we start out on the task of solving a particular problem, it is
important that we know what the initial situation is (that is, the
situation defining the problem), as well as what final situations
constitute acceptable solutions to the problem. Along the way from the
initial situation to an acceptable final situation, we pass through a
set of intermediate steps, each leading from one situation to another.
It is convenient that all of these intermediate situations, as well as
the initial and final situation(s), be classified as {\em states\/},
and that the process of moving from the initial to a final state be
viewed as moving through a {\em state space\/}. Further, we need a set
of {\em rules\/} describing the legal moves, given the current state.
In any state, choosing one rule over another may bring us closer to or
farther away from a solution, depending on the specific rule. In other
words we need a control strategy to help us choose an appropriate
rule. When no direct method for choosing a rule is known, as is
usually the case, {\em search\/} has to be applied.  Search also
provides a framework into which more direct methods for solving
subparts of a problem can be embedded. Thus the process of finding a
solution to a given problem can be classified as searching through a
state space.

\subsection{Production Systems}
\label{prodsys}

{\em Production systems\/} provide a means for describing and
performing the search process described above. A production system
consists of (from~\cite{rich}):

\begin{itemize}
  \item A set of rules, each consisting of a left side (a pattern)
    that determines the applicability of the rule and a right side
    that describes the operation to be performed if the rule is
    applied.
  \item One or more knowledge/databases that contain whatever
    information is appropriate for the particular task.
  \item A control strategy that specifies the order in which the
    rules will be compared to the database and a way of resolving the
    conflicts that arise when several rules match at once.
  \item A rule applier.
\end{itemize}

This definition is very general, and it encompasses a great number of
systems, including a family of complex, often hybrid, systems known as
{\em expert systems\/} (see Section~\ref{expert}).

\subsubsection{Control Strategies}

As mentioned above, we need a control strategy to help us decide which
rule to apply, given the current state of a search for a solution of a
problem. How these decisions are made has a great impact on how
quickly, if at all, a solution is reached. Two requirements the
control strategy has to fulfill, are that it cause {\em motion\/} and
that it be {\em systematic\/}. The first requirement ensures that a
solution sooner or later will be arrived at, given that it exists, and
the second requirement ensures that a certain path in the search space
not be explored more than once.

Several search strategies have been suggested, the most well-known
being

\begin{description}
  \item[Depth-first search] Construct a tree with the initial
    state as its root. Generate a child by applying one of the
    applicable rules to the initial state. Pursue a single branch of
    the tree by repeating this process at the lastly generated leaf
    node until a goal is produced or it is decided to terminate the
    branch. If the branch is terminated, backtrack to the first
    predecessor which has not had all of its offspring examined and
    continue the search from there.
  \item[Breadth-first search] Generate all the offspring of the
    root by applying each of the applicable rules to the initial
    state.  Repeat this process for all the leaf nodes until some rule
    produces a goal.
  \item[Heuristic search techniques] Sometimes it is required that
    one compromise the requirements of mobility and systematicity in
    order to efficiently arrive at a solution which no longer is
    guaranteed to be the best one, but which almost always is a very
    good one. A {\em heuristic\/} is a search technique which
    accomplishes this, employing what is called a {\em heuristic
    function\/} to decide which rule to choose. The value returned by
    the heuristic function is a measure of the desirability of
    exploring the given path. ``{\em The purpose of a heuristic
    function is to guide the search process in the most profitable
    direction by suggesting which path to follow first when more than
    one are available}'' (from~\cite{rich}).
\end{description}

Most complex problems, and that means most problems of practical
interest, need to employ heuristic search techniques in order to be
solved efficiently, and the process of problem solving can thus more
accurately be described as a heuristic search process rather than
merely a search process.

\subsection{Expert Systems}
\label{expert}

As mentioned in Section~\ref{prodsys}, expert systems constitute a
subset of the family of production systems.  An {\em expert system\/}
can be labeled as a system which guides the user in solving problems
which normally requires the intervention of a human expert in the
field.

To solve an expert-level problem, an expert system needs to employ a
powerful reasoning system ({\em inference engine;\/}~\cite{amble}) as
well as to have access to a substantial domain-specific {\em knowledge
base\/}. To be convincing as to its conclusions, it is also important
that the user of the expert system be able to interact with it easily,
and that he, when it has arrived at a conclusion, can be given an
explanation of how the conclusion was arrived at, that is, that the
system be able to explain its reasoning.  Moreover, the system has to
be able to acquire new knowledge as well as modifications of old
knowledge.

\subsubsection{Expert System Shells}

Expert systems are employed in a wide variety of domains, including
medical diagnosis, legislation and decision making, just to mention
some. But even though the domains of different expert systems have
little or nothing in common, they are generally based on the same
principles. Thus it is common to separate the domain specific parts
(typically the knowledge base) of an expert system from the non-domain
specific ``core'', which usually is implemented separately as a
general purpose expert system {\em shell\/}. The tasks performed by
the expert system shell are typically (see~\cite{amble}) those of the
inference engine, search processing and dialogue handling (user
interface).

From an {\nlp}-point of view, the ideal expert system shell would be
one in which all interaction with the user as well as knowledge
acquisition and explanation are done in natural language, and which
still contains all the features of present day shells. Whether this is
possible, is the question which the {\tuc}-project seeks to find some
of the answers to.
