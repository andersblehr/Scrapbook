% >>> Missing from original .ZIP, reconstructed Dec 2016 <<<

\cleardoublepage
\chapter*{Preface}
\label{preface}

This paper constitutes my diploma thesis for the title
``sivilingeni\o r'' at the Norwegian Institute of Technology.  As a
participant in the EC ERASMUS exchange programme, I have had the
opportunity to carry out my thesis work at the Institute of Applied
Physics and Biophysics, Philipps-Universit{\"a}t Marburg, Germany.

My task has been to develop and implement an algorithm for determining
the location of the pupil in a given image of the eye.  The algorithm
is at a later stage intended for form the core of a real-time
eye-tracking system, and accordingly it has been of vital importance
during the development process that the algorithm be capable of
performing in a real-time environment.  A second important requirement
has been that the positions returned by the algorithm be as a accurate
as possible.

The structure of the paper is as follows.  Chapter~\ref{intro} is
intended to give an overall view of the background and motivation for
my thesis work.  The requirements that a future eye-tracking system
will have to satisfy are presented, and in relation to these, the
problem definition to which my work has been related is given.
Chapter~\ref{back} supplies some general background material relating
to the eye and its structure.  Also, some aspects of the optical
properties of the eye are discussed.  Lastly, some already existing
eye-tracking systems are described and evaluated with respect to the
given requirements.  In Chapter~\ref{image}, a relatively broad
presentation of different image processing techniques is given.  These
techniques all constitute possible approaches to the given problem and
accordingly are given a relatively thorough treatment.  In
Chapter~\ref{eval}, the techniques presented are evaluated with
resepct to their applicability to the problem at hand and a decision
is made as which combination of techniques appears to constitute the
most promising approach to designing an algorithm to solve the
problem.  In Chapter~\ref{algo}, the proposed algorithm is presented
in detail whereupon it is evaluated in terms of the given
requirements, based on a set of tests performed.  In
Section~\ref{concl} some conclusions that can be drawn from my thesis
work are presented and finally, Appendix~\ref{implem} discusses some
implementation issues and lists the sources constituting the current
implementation of the algorithm.

I would like to thank my supervisor at Philipps-Universit{\"a}t Marburg,
Reinhard Eckhorn, for giving me the opportunity to stay here in
Marburg, and also my supervisor at the Norwegian Institute of
Technology, Jan Komorowski, for consenting to me staying here.  A very
special thanks goes to Uwe Thomas for his continued support and help
during all stages of my work, and particularly for being a friend to
turn to.  Thanks also to the staff here at the institute for lighting
up the gray days of programming.  Lastly, I would like to say a very
big thanks to my family for making it all possible and to all my
friends in Trondheim, presently dispersed all over Norway and Europe,
for making the four years we spent together in in Trondheim a
memorable time.
