%%%%%%%%%%%%%%%%%%%%%%%%%%%%%% -*- Mode: Latex -*- %%%%%%%%%%%%%%%%%%%%%%%%%%%%
%%% bk_eye.tex --- 
%%% Author          : blehr
%%% Created On      : Tue Mar  9 03:58:22 1993
%%% Last Modified By: blehr
%%% Last Modified On: Sun Apr 11 13:26:13 1993
%%% RCS revision    : $Revision: 1.6 $ $Locker:  $
%%% Status          : In writing....
%%%%%%%%%%%%%%%%%%%%%%%%%%%%%%%%%%%%%%%%%%%%%%%%%%%%%%%%%%%%%%%%%%%%%%%%%%%%%%

\section{The Eye}
\label{back:eye}

In this section, an introduction is given to some of the terms
relating to the eye, as well as to its structure.  Although the
subjects may be human as well as monkeys, only the human eye is
described, given that the terms in general also apply to the monkey's
eye.  Also, some, from an image processing point of view, useful
visual properties of the eye are described.

\subsection{Structure of the Human Eye}
\label{back:eye:structure}

In Fig.~\ref{fig:back:eye:human}, a horizontal cross section of the
human eye is shown.  The shape of the eye is nearly spherical, and its
average diameter is approximately 20 mm.  The eye ball ({\em optical
  globe\/}) is made up of three enclosing membranes: the {\em
  cornea\/} and {\em sclera\/}, which together form the outer cover of
the eye, the middle {\em choroid\/}, and the inner {\em retina\/}.

The cornea covers the {\em iris\/} and the {\em pupil\/}, and consists
of a tough, transparent tissue, allowing light to pass through.  The
rest of the eye is covered by the white sclera, which is continuous
with the cornea.

The choroid lies directly below the sclera.  This membrane contains a
network of blood vessels that serves as a major source of nutrition to
the eye.  Its coat is heavily pigmented to reduce the amount of
extraneous light entering the eye.  The part of the choroid belonging
to the foremost portion of the eye (its {\em anterior extreme\/}) is
divided into the iris and the {\em ciliary body\/}.  The {\em ciliary
  muscles\/} allow the iris to contract or expand to control the
amount of light entering the eye through the pupil.  The pupil is
variable in diameter from approximately 2 mm up to 8 mm, corresponding
to an area range of 3--50 mm$^{2}$.  The front of the iris contains
the visible pigment of the eye, whereas the back contains a black
pigment.

\insertpdfwidth{eye}{\label{fig:back:eye:human}A simplified diagram of
  a cross section of the human eye.
  (From~\protect\cite{digim})}{0.6}

The {\em lens\/}, which lies behind the iris, is made up of concentric
layers of fibrous cells and is suspended by {\em ciliary fibres\/},
attaching it to the ciliary body.  It contains between 60 and 70 per
cent water, about 6 per cent fat, and more protein than any other
tissue in the eye.  The shape of the lens is controlled by the tension
in the ciliary fibres.  To focus on distant objects, the controlling
muscles cause the lens to be relatively flattened, whereas they cause
the lens to thicken to focus on objects near the eye.  Both
ultraviolet and infrared light are absorbed well by the lens and, in
large amounts, may cause damage to the eye.

The innermost membrane of the eye is the retina, which covers the
inside of the backmost portion of the eye.  The surface of the retina
is covered by light receptors on which an image of the outside world
(an object, a scene, etc.) is focused by the lens.  There are two
classes of light receptors: {\em cones\/} and {\em rods\/}.  The
cones, which number between 6 and 7 million and are highly sensitive
to colour, are primarily located in the central portion of the
retina, the {\em fovea\/}.  The eye is rotated until the image of the
object of interest falls on the fovea.  Each cone is connected to its
own nerve end, allowing very fine visual resolution.  Cone vision is
known as {\em photopic\/} or bright-light vision.

The rods, which are distributed over the entire retinal surface,
number between 75 and 150 million, and serve to give a general,
overall picture of the field of view.  The larger area of distribution
and the fact that several rods are connected to a single nerve end
reduce the amount of detail discernible by them.  They are insensitive
to colour, but much more sensitive to low levels of illumination than
the cones.  Rod vision is known as {\em scotopic\/} or dim-light
vision.

\subsection{Optical Properties of the Eye}
\label{back:eye:visual}

In Fig.~\ref{fig:scaneye} is shown a relatively high-quality image of
a monkey's eye.  From an eye-tracking point of view, the most
interesting optical property noted is the relatively sharp transition
in brightness between the dark pupil and the comparably bright iris.
It is noted, however, that the iris of the monkey appears to be darker
than that of man.  Note also that the pupil does not constitute the
only dark region in the image.  In fact, in the given image there
appears to be large regions outside the pupil with gray levels in the
proximity of those of the pupil.  Evidently, an eye-tracking system
has to take this into consideration in order to be able to
differentiate between the pupil and other dark regions in an image.

Another thing to be noted is that the pupil appears to approximate a
circle relatively well.  However, although it is known to be circular
physically, the appearance of the pupil is circular only when the
subject looks straight into the camera.  If it looks to the side, the
pupil appears as an ellipse whose major axis is vertical; if it looks
up or down, the major axis of the pupil ellipse is horizontal; if it
looks up or down as well as to the side, the major axis of the pupil
ellipse has either a positive or negative angle with respect to the
horizontal plane.  The deviation in the pupil's appearance from the
circular shape depends on the gaze angle relative to the camera.  When
the angle is relatively small, i.e., in the approximate range $\pm
15^{\circ}$, the deviation is less than 4\%, but when the angle
increases, the elliptic appearance becomes more dominant and is apt to
cause errors in eye-tracking systems assuming that the pupil always
constitutes a circle in an image.

Note also the bright spot at the upper extremity of the pupil.  This
is called to as the {\em first Purkinje image\/}, and constitutes the
reflection of the illuminating source from the anterior surface of the
cornea.  

A last point of interest is that the iris appears brighter under
infrared illumination than in visible light (\cite{template}), which
is an additional incentive for choosing infrared illumination of the
eye for the final eye-tracking system (cf.\ 
Section~\ref{intro:problem}).

\insertpdfwidth{scaneye}{\label{fig:scaneye}An image of a monkey's
  eye.}{0.4}
