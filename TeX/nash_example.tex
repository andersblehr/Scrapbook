%%%%%%%%%%%%%%%%%%%%%%%%%%%%%% -*- Mode: Latex -*- %%%%%%%%%%%%%%%%%%%%%%%%%%%%
%%% nash_example.tex --- 
%%% Author          : Anders Blehr
%%% Created On      : Mon Apr 27 17:27:20 1992
%%% Last Modified By: Anders Blehr
%%% Last Modified On: Mon May  4 06:22:36 1992
%%% RCS revision    : $Revision: 1.1 $ $Locker:  $
%%% Status          : Unknown, Use with caution!
%%%%%%%%%%%%%%%%%%%%%%%%%%%%%%%%%%%%%%%%%%%%%%%%%%%%%%%%%%%%%%%%%%%%%%%%%%%%%%

\section{An Example Domain - The Norwegian Nationality Act}
\label{example}

As an example domain for {\nash}, I have chosen the Norwegian
Nationality Act. This is due to the fact that legislation, being
basically definitional in nature, is particularly suited for
formalization. This example is presented as though {\nash} is working
completely in accordance with its specification. For the most part
this is true, at least for the given domain, but no guarantee is given
that the system will behave well in all circumstances (see
Section~\ref{limsys}).

\subsection{User Commands}

In order for the user to be able to ask the system to perform specific
tasks not directly connected to the reasoning process, {\nash} is
equipped with a (small) set of user commands. These are

\begin{itemize}
\item {\em summary\/}: This command causes the system to print the
  contents of the knowledge base, paraphrased into {\nal}. Facts are
  preceded with the keyword ``{\sl known:\/}'', and rules with the
  keyword ``{\sl rule:\/}''.
\item {\em resk\/}: Retract all facts from the knowledge base.
\item {\em resr\/}: Retract all rules from the knowledge base.
\item {\em reset\/}: Retract everything from the knowledge base, both
  facts and rules.
\end{itemize}
In addition the two built-in predicates {\em debug\/} and {\em
nodebug\/} may be invoked from {\nash}.

\subsection{Judicial Expert Systems}

In judicial expert systems, the legislation is the primary source of
the knowledge base. In addition, it may be influenced by other
legislative factors, such as {\em case law\/} (law established by
judicial decision in cases) and judicial literature. An expert system
with the Norwegian Nationality Act as its domain will be able to
answer questions like ``{\em Can Eigil Skallagrimsson become a
Norwegian citizen?\/}''.

It is important to have in mind that judicial expert systems never can
replace a judicial system. Legislation often contains {\em vague\/}
statements, such as ``{\em being of good character\/}'' and ``{\em
having sufficient knowledge of Norwegian\/}'', as well as other
ambiguities, and sometimes even contradictions. This demands careful
consideration in each case to which the particular piece of
legislation is applied, consideration which an automated reasoning
system such as an expert system will not be able to support, at least
not in the foreseeable future. The main task of judicial expert
systems is thus a supervisory one; they may act as a guide in finding
one's way through intricate law texts, and may point to areas
demanding extra consideration.

The pioneer project in the field of legislative expert systems was a
system based on the British Nationality Act, done at the Imperial
College, U.\ K., and presented in~\cite{british}. The Norwegian
Nationality Act has been used previously as domain for judicial
expert systems in connection with the {\hsql}-project at {\nit};
see for instance~\cite{foredrag} and~\cite{maehle}.

\subsection{Creating the Knowledge Base}

In Appendix~\ref{app_lov}, the extract from the Norwegian Nationality
Act on which the knowledge base of the example is based, is given.

In order for the system to be able to understand the text, it has to
be translated into {\nal}, in the form of formal rules. That is, the
paragraphs have to be transformed into {\em if-then\/}-constructs,
where the premises are conjugations of simple declarative statements,
and the conclusions are single declarative statements. The given
paragraphs have already been logically formalized in~\cite{eggen}, and
are thus easily translated into {\nal}. In this example we will
concentrate on \S 1, which gets the following representation:

\begin{tabbing}
  xxx\= xxx\= If \= \kill
  \>\>{\em If any woman is a citizen in any year and\/}\\
  \>\>\>{\em this woman is the mother of any person and\/}\\
  \>\>\>{\em this person is born in this year\/}\\
  \>\>{\em then\/}\\
  \>\>\>{\em this person is a citizen by birth.\/}
\end{tabbing}

\begin{tabbing}
  xxx\= xxx\= If \= \kill
  \>\>{\em If any man is a citizen in any year and\/}\\
  \>\>\>{\em this man is the father of any person and\/}\\
  \>\>\>{\em any woman is the mother of this person and\/}\\
  \>\>\>{\em this man marries this woman in any other year and\/}\\
  \>\>\>{\em this other year is before this year and\/}\\
  \>\>\>{\em this person is born in this year\/}\\
  \>\>{\em then\/}\\
  \>\>\>{\em this person is a citizen by birth.\/}
\end{tabbing}

\begin{tabbing}
  xxx\= xxx\= If \= \kill
  \>\>{\em If any man is the father of any person and\/}\\
  \>\>\>{\em any woman is the mother of this person and\/}\\
  \>\>\>{\em this man dies in any year and\/}\\
  \>\>\>{\em this man is a citizen in this year and\/}\\
  \>\>\>{\em this man is married to this woman in this year\/}\\
  \>\>{\em then\/}\\
  \>\>\>{\em this person is a citizen by birth.\/}
\end{tabbing}

\begin{tabbing}
  xxx\= xxx\= If \= \kill
  \>\>{\em If any person is an abandoned orphan\/}\\
  \>\>{\em then\/}\\
  \>\>\>{\em this person is a citizen.\/}
\end{tabbing}
``{\em Citizen\/}'' in this context is to be interpreted as ``{\em
Norwegian citizen\/}''. These formalized rules can be fed directly to
{\nash}, which stores them as formal rules in the knowledge base.

\subsection{A Sample Dialogue}

If we want to find out whether a certain person, e.g., Kristin, is a
Norwegian citizen, we ask the question ``{\em Is Kristin a
citizen?\/}'' to the system. In order for the system to know that any
citizen by birth also is a citizen, we have to supply it with the
additional rule (here given as an implicit rule):

\begin{tabbing}
  xxx\= xxx\= \kill
  \>\>{\em Every citizen by birth is a citizen.\/}
\end{tabbing}

The system will now try to prove that Kristin indeed {\em is\/} a
citizen, first by investigating whether this is an already known fact
(which we assume it is not in our case), then by applying the above
rules one by one until either a proof has been found or failure is
reported. When applying a certain rule, the system may have to query
the user for information in order to arrive at a proof.

\subsubsection{Interacting with the User}

The first assertion that the system has to prove (the initial
question, that is), is whether Kristin is a citizen. There exist two
rules in the knowledge base describing how she can be classified as a
citizen. One says that she is a citizen if she is an abandoned orphan,
the other that she is a citizen if she is a citizen by birth. The
rules are applied in order of appearance in the knowledge base, and
thus the first question asked to the user is

\begin{tabbing}
  xxx\= xxx\= \kill
  \>\>{\em Is Kristin an abandoned orphan?\/}
\end{tabbing}
If the answer given is ``{\em yes\/}'', the assertion has been proved,
and the search process terminates. If the answer is ``{\em no\/}'',
the system tries to prove that Kristin is a citizen by birth, to which
it can apply either of the three first rules above. The questions
asked in connection with each rule, as well as answers causing a
solution for the particular rule, are:

\begin{tabbing}
  xxx\= xxx\= \kill
  \>\>{\em When is Kristin born?\/}\ \ {\sl Kristin is born in 1966.}\\
  \>\>{\em Who is a mother of Kristin?\/}\ \ {\sl Sigrid.}\\
  \>\>{\em Is Mary a citizen in 1966?\/}\ \ {\sl Yes.}
\end{tabbing}

\begin{tabbing}
  xxx\= xxx\= \kill
  \>\>{\em When is Kristin born?\/}\ \ {\sl 1966.}\\
  \>\>{\em Who is a mother of Kristin?\/}\ \ {\sl Sigrid is the mother
  of Kristin.}\\
  \>\>{\em Who is a father of Kristin?\/}\ \ {\sl Jamm\ae lt.}\\
  \>\>{\em Is Jamm\ae lt a citizen in 1966?\/}\ \ {\sl Yes.}\\
  \>\>{\em When does Jamm\ae lt marry Sigrid?\/}\ \ {\sl Jamm\ae lt
  marries Sigrid in 1964.}
\end{tabbing}

\begin{tabbing}
  xxx\= xxx\= \kill
  \>\>{\em Who is a mother of Kristin?\/}\ \ {\sl Sigrid.}\\
  \>\>{\em Who is a father of Kristin?\/}\ \ {\sl Jamm\ae lt.}\\
  \>\>{\em When does Jamm\ae lt die?\/}\ \ {\sl 1964.}\\
  \>\>{\em Is Jamm\ae lt a citizen in 1964?\/}\ \ {\sl Yes.}\\
  \>\>{\em Is Jamm\ae lt married to Sigrid in 1964?\/}\ \ {\sl Yes.}
\end{tabbing}
If the information supplied by the user does not suffice in proving
that Kristin is a citizen by any rule in the knowledge base, the user
is notified that failure has occurred.

Note that it is possible to give full-sentence answers as well as one
word answers. In fact, it is possible to supply answers which may have
no relevance whatsoever with the question. The information contained
in these answers is added to the knowledge base and may be used to
answer later questions, which thus do not have to be presented to the
user.  For instance, if given the question ``{\em Who is a mother of
Kristin?\/}'', the answer may well be ``{\em Jamm\ae lt dies in
1964\/}''. The system still does not know who Kristin's mother is, and
will have to ask the question again, but it {\em does\/} know that
Jamm\ae lt dies in 1964.  In other words, all questions given in
connection with a specific rule have to be answered in order to arrive
at a conclusion, but the order in which the answers are given does not
have to correspond with the order of the questions.

\subsubsection{Supplying Proofs}

If the user wants a proof as to why Kristin is a citizen, he asks the
question ``{\em How is Kristin a citizen?\/}'' to the system. For
instance, if the system is able to prove that Kristin is a citizen by
the second rule above, the proof supplied to the user may be (how
proofs are collected is described in section~\ref{proof}):

\begin{tabbing}
  xxx\= xxx\= xxx\= xxx\= xxx\= xxx\= xxx\= \kill
  \>\>{\em Kristin is a citizen if Kristin is a citizen by birth.}\\
  \>\>{\em Kristin is a citizen by birth if}\\
  \>\>\>{\em A is a man and B is a year and A is a citizen in B and}\\
  \>\>\>{\em A is a father of Kristin and}\\
  \>\>\>{\em C is a woman and C is a mother of Kristin and}\\
  \>\>\>{\em D is a year and A marries C in D and D is before B}\\
  \>\>\>{\em and Kristin is born in B.}\\
  \\
  \>\>{\em 1964 is before 1966.\/}\\
  \>\>{\em Jamm\ae lt is a citizen in 1966.\/}\\
  \>\>{\em Jamm\ae lt is a father of Kristin.\/}\\
  \>\>{\em Sigrid is a mother of Kristin.\/}\\
  \>\>{\em Jamm\ae lt marries Sigrid in 1964.\/}\\
  \>\>{\em Kristin is born in 1966.\/}
\end{tabbing}

\subsubsection{Summaries}

In addition to query the system and ask for proofs, the user may ask
to be given a summary of the contents of the knowledge base. The
command {\em summary\/} is used for this purpose. For instance, if the
knowledge base has the same contents as that of the previous example,
the summary given will look like:

\begin{tabbing}
  xxx\= xxx\= {\sl Rule: } \= {\em C} \= xxx\= xxx\= xxx\= \kill
  \>\>{\sl Known: }{\em Kristin is born in 1966.\/}\\
  \>\>{\sl Known: }{\em Kristin is born.\/}\\
  \>\>{\sl Known: }{\em Sigrid is a mother of Kristin.\/}\\
  \>\>{\sl Known: }{\em Sigrid is a mother.\/}\\
  \>\>{\sl Known: }{\em Jamm\ae lt is a father of Kristin.\/}\\
  \>\>{\sl Known: }{\em Jamm\ae lt is a father.\/}\\
  \>\>{\sl Known: }{\em Jamm\ae lt is a citizen in 1966.\/}\\
  \>\>{\sl Known: }{\em Jamm\ae lt is a citizen.\/}\\
  \>\>{\sl Known: }{\em Jamm\ae lt marries Sigrid in 1964.\/}\\
  \>\>{\sl Known: }{\em Jamm\ae lt marries Sigrid.\/}\\
  \>\>{\sl Known: }{\em Sigrid is in 1964.\/}\\
  \>\>{\sl Known: }{\em Jamm\ae lt is in 1964.\/}\\
  \\
  \>\>{\sl Rule: }{\em C is a citizen by birth if}\\
  \>\>\>\>{\em A is a man and B is a year and A is a citizen in B and}\\
  \>\>\>\>{\em C is a person and A is a father of C and}\\
  \>\>\>\>{\em D is a woman and D is a mother of C and}\\
  \>\>\>\>{\em E is a year and A marries D in E and E is before B and}\\
  \>\>\>\>{\em C is born in B.}\\
  \>\>{\sl Rule: }{\em A is a citizen if}\\
  \>\>\>\>{\em A is a person and A is a citizen by birth }
\end{tabbing}

Note that some of the information contained in the knowledge base is
inconsistent with common sense knowledge about the world. For
instance, it is not sound to deduce from the fact that Jamm\ae lt is
(or rather, {\em was\/}) a citizen in 1966 that he still is a citizen,
which is how the average user would interpret the sentence ``{\em
Jamm\ae lt is a citizen\/}''. This kind of inconsistence is a
consequence of {\nash}' lack of semantic analysis (see
Sections~\ref{anagen} and~\ref{limnal}), and may cause the knowledge
base to contain some rather awkward pieces of knowledge.

