%%%%%%%%%%%%%%%%%%%%%%%%%%%%%% -*- Mode: Latex -*- %%%%%%%%%%%%%%%%%%%%%%%%%%%%
%%% nash_def.tex --- 
%%% Author          : Anders Blehr
%%% Created On      : Mon Apr 27 17:24:52 1992
%%% Last Modified By: Anders Blehr
%%% Last Modified On: Mon May  4 06:12:49 1992
%%% RCS revision    : $Revision: 1.1 $ $Locker:  $
%%% Status          : Unknown, Use with caution!
%%%%%%%%%%%%%%%%%%%%%%%%%%%%%%%%%%%%%%%%%%%%%%%%%%%%%%%%%%%%%%%%%%%%%%%%%%%%%%

\section{The Language}
\label{def}

In this section the semi-natural language employed in {\nash}, {\nal},
is described. As mentioned in the previous section, the aim with
{\nal} has been that it be as close to standard English as possible,
and that it be sufficiently formal. The emphasis has been on accepting
legal constructs rather than rejecting illegal ones.

\subsection{Linguistic Background}
\label{ling}

Most of the linguistic material in this section is gathered
from~\cite{gal}. 

The basic element of any (natural) language is the {\em word\/}, which
may have associated with it different {\em inflections\/} (number,
tense, degree, etc.), depending on its function in the language. A
{\em sentence\/} in a language is a sequence of words, and the
particular language defines, by means of abstract rule systems, which
sequences constitute legal sentences.  Within each sentence, words
make up syntactic {\em constituents\/}, which in turn can be part of
more complex syntactic constituents. The constituents are determined
by how strongly the individual words of the sentence are linked to
each other (e.g., an adjective will have stronger links with the noun
it qualifies than with a preposition). These links are made explicit
by the internal structure of the sentence. Simple sentences can be
combined or their internal structures can be rearranged in order to
form more complex constructions. Basic constituents are built from
(centered on) the basic syntactic {\em categories\/} (i.e., word
classes: nouns, verbs, adjectives, adverbs, prepositions, pronouns,
etc.), and are defined in terms of them. Thus we have {\em noun\/}
phrases, {\em verb\/} phrases, {\em adjective\/} phrases, {\em
adverbial\/} phrases, {\em prepositional\/} phrases, and so on (the
abbreviations {\sl NP\/}, {\sl VP\/}, {\sl AdjP\/}, {\sl AdvP\/}, {\sl
PP\/} and the like are commonly used). The structure of the
constituents is described using {\em phrase-structure rules\/}.

As an example, consider English, where a simple sentence can be built
from a noun phrase and a verb phrase (using BNF-notation for the
phrase-structure rules):
\begin{tabbing}
  xxx\= xxx\= \kill \>\>\begin{tabular}{lcl} {\sl S\/} &
  $\rightarrow$ & {\sl NP\/}, {\sl VP\/}.
\end{tabular}  
\end{tabbing}
Similarly, a noun phrase can be made up from a determiner, an adjective
phrase and a noun, a verb phrase from a (transitive) verb and a noun
phrase (representing the direct object), and an adjective phrase from
either an adjective or the empty string (denoted $\epsilon$):
\begin{tabbing}
  xxx\= xxx\= \kill \>\>\begin{tabular}{lcl} {\sl NP\/} &
  $\rightarrow$ & {\sl Det\/}, {\sl AdjP\/}, {\sl N\/} \\ {\sl VP\/} &
  $\rightarrow$ & {\sl V\/}, {\sl NP\/} \\ {\sl AdjP\/} &
  $\rightarrow$ & {\sl Adj\/} \\ & $|$ & $\epsilon$
\end{tabular}
\end{tabbing}
Using these simple rules, we can construct sentences like ``The old man
walked the dog'' and ``Every man loves a woman''.

The incorporation of words is achieved by {\em lexical insertion
rules\/}, e.g.,
\begin{tabbing}
  xxx\= xxx\= \kill \>\>\begin{tabular}{lcl} {\sl N\/} &
  $\rightarrow$ & [{\sl Noun\/}],
\end{tabular}  
\end{tabbing}
where {\sl Noun\/} is any word in the {\em dictionary\/} classified as a
noun. The dictionary (or {\em lexicon\/}) contains all the words of the
given language, along with categorical information and morphological
characteristics.

A sentence is said to be {\em well-formed\/} if there exists at least
one set or sequence of rules which define a complete description of
the sentence. If there exist more than one such set or sequence, the
sentence is said to be {\em structural ambiguous\/} (this corresponds
to the inherent ambiguity of any natural language, see for instance
the examples in Section~\ref{nla}).

\subsection{Definition}
\label{naldef}

For a BNF-description of {\nal}, see Appendix~\ref{app_def}.

The largest unit of {\nal} is the {\em sentence\/}. There are two
types of sentences: {\em statements\/}, representing facts (knowledge)
and rules, and {\em queries\/}, representing questions, either to the
system or to the user. The only form of punctuation marks allowed in
{\nal}, are ``.'', used to end statements, and ``?'', used to end
queries. For clarity, other punctuation marks may occur in the
examples, which all constitute legal sentences in {\nal}, given that
the words they contain are found in the dictionary.

\subsubsection{Statements}

The statements of {\nal} can be divided into two distinct categories,
{\em declarative statements\/}, whereof the set of {\em implicit
rules\/} constitute a subset, and {\em formal rules\/}.

{\em Declarative statements\/} convey some sort of information about
the world, and may be conjugated with other declarative statements
when different aspects of information about a particular world object
(or several world objects) need to be contained in one sentence. The
following sentences constitute declarative statements:

\begin{tabbing}
  xxx\= xxx\= \kill
  \>\>{\em John is an old man.\/}\\
  \>\>{\em Peter lives with Mary in London.\/}\\
  \>\>{\em The woman that Robert loves, loves Frank.\/}\\
  \>\>{\em Maurice builds a large house on an open field outside
  Marseilles\/.}\\
  \>\>{\em John looks at Mary and Mary talks to Peter and Peter
  admires Janice.\/}
\end{tabbing}

A subset of the set of declarative statements is the set of {\em
implicit rules\/}, which can be used to convey knowledge about
specific classes of world objects. An implicit rule is defined by the
determiner ``every''. The justification for this supposition is found
in predicate calculus (see Section~\ref{predicate} and~\cite{logic}),
where it has been pointed out that it is a common, albeit not
universal pattern of predicate logic that the universal quantifier be
followed by an implication. Examples of implicit rules are:

\begin{tabbing}
  xxx\= xxx\= \kill
  \>\>{\em Every man loves a woman.\/}\\
  \>\>{\em Rosette likes every picture that is painted by
  Cezanne.\/}
\end{tabbing}

{\em Formal rules\/} are constructs of the form {\em ``If $<${\sl
premise\/}$>$ then $<${\sl conclusion\/}$>$''}, and are used for
defining more complex relations between specific world objects, and
particularly for defining general rules to be used for reasoning
within a certain domain. Both the {\sl premise\/} and the {\sl
conclusion\/} have to be declarative statements. The sets of implicit
and formal rules are not disjunct, that is, it is often possible to
express knowledge about a specific relation either as an implicit or
formal rule. It is recommended, though, for the sake of determinism
(complex implicit rules may be transformed into logical expressions
containing more than one implication, which in {\nash} represent an
unhandled case), that complex relations be expressed by means of
formal rules. The following constitute examples of formal rules:
\label{specrules}

\begin{tabbing}
  xxx\= xxx\= \kill
  \>\>{\em If John is married to Mary, then Mary is married to
  John.\/}\\
  \>\>{\em If Maurice loves Michelle and Michelle is married to Paul,
  then Maurice is unhappy.\/}
\end{tabbing}

\subsubsection{Queries}

{\em Queries\/} are used for asking questions to be answered either by
the system or the user. What makes queries different from other
sentences, (statements, that is), is that they have a different
syntactic structure, and that they demand some sort of action to be
taken. Queries may belong to one of the following categories:

\begin{description}
  \item[Aux-prefixed] The first word of the query is either ``{\em
    is\/}'' or ``{\em does\/}'' (plurals are not handled, see
    Section~\ref{limnal}). Queries belonging to this category may be 
    classified as {\em confirmative\/}, in that the answers they
    demand be either negative (``{\em no\/}''-answers) or affirmative
    (``{\em yes\/}''-answers).
  \item[Wh-prefixed] The first word of the query is a so-called
    {\em wh-pronoun\/} (``{\em what\/}'', ``{\em who\/}'', ``{\em
    where\/}'', ``{\em when\/}'', etc.). Queries belonging to this
    category correspond to the user (or the system) during the
    reasoning process having come across an unbound variable which
    is required to be bound in order that a solution be found. The
    response given is required to cause a binding of this variable,
    usually to a term supplied as part of the response.
\end{description}
In addition, aux-prefixed queries (to the system) may be prefixed with
the keyword ``{\em how\/}'' in order that a proof be supplied (see
Section~\ref{proof}).  Examples of queries are (the corresponding
declarative statements are given above):

\begin{tabbing}
  xxx\= xxx\= \kill
  \>\>{\em Is John an old man?\/}\\
  \>\>{\em Does Peter live with Mary in London?\/}\\
  \>\>{\em Who does the woman that Robert loves, love?\/}\\
  \>\>{\em What does Maurice build on an open field outside
  Marseilles?\/}
\end{tabbing}

\subsubsection{Using Variables}

To attribute knowledge to a specific class of world objects as a whole
(e.g., persons, birds, houses, etc.), it would be rather cumbersome
explicitly to attribute this knowledge to every known individual
belonging to the class. Rather, it would be desirable that it be
possible to refer to the class as a whole. Specifically, we need to be
able to assert {\em general rules\/}, applying to entire classes of
world objects, not only to specific individuals of the class. To
achieve this, specific syntactic constructs (defined by Tore Amble,
and also employed in {\tuc}) have been introduced to represent
uninstantiated individuals of a class. 

Look at the two formal rules listed in the example on
page~\pageref{specrules}. They apply only to John and Mary, and
Maurice, Michelle and Paul, respectively, even though they define
universal relations, namely that marriage is a commutative relation,
and that a man who loves a married woman (who is not his wife) is
unhappy (whether this last relation is always true in the {\em real\/}
world is not of concern here). If we restate these rules as
\begin{tabbing}
  xxx\= xxx\= \kill
  \>\>{\em If any man is married to any woman then this woman is
  married to this man.\/}
\end{tabbing}
and
\begin{tabbing}
  xxx\= xxx\= xxx\= \kill
  \>\>{\em If any man loves any woman and this woman is married to any
  other man,\/}\\
  \>\>\>{\em then this man is unhappy.\/},
\end{tabbing}
they, if they are universally quantified, still apply to John and Mary
and the rest, given that it is known that John is married to Mary, and
that Maurice loves Michelle who is married to Paul. In addition, they
apply to every other set of individuals fulfilling the same premises.
This can be formalized as follows:

The {\em key structures\/} ``{\em any\/}'', ``{\em any other\/}'' and
``{\em any third\/}'' define the immediately following noun to
quantify a variable. Thus a specific noun may quantify up to three
different variables (by defining more key structures, it would be
possible that it quantify accordingly many more variables). If a set
of variables has been defined by these constructs, they may be
referenced at a later time (within the same rule) by applying the key
structures ``{\em this\/}'', ``{\em this other\/}'' and ``{\em this
third\/}'', together with the quantifying noun. Thus the variables
defined by the constructs ``{\em any man\/}'', ``{\em any other
piece\/}'' and ``{\em any third river\/}'' may be referenced using the
constructs ``{\em this man\/}'', ``{\em this other piece\/}'' and
``{\em this third river\/}''.

It is evident that restricting the words ``{\em any\/}'' and ``{\em
this\/}'' to this use only, imposes restrictions on the language we
are defining. On the other hand, the advantages of employing a
mechanism like this one by far exceed the disadvantages.

\subsection{Analysis and Generation}
\label{anagen}

{\nal} is treated at a purely syntactic level. That is, the semantic
analysis step described in Section~\ref{nlpr} is not employed; words
are treated as pure lexical units, that is, they have no {\em
meaning\/} associated with them, and sentences are treated as pure
syntactic constructs. The semantic representations that are generated
represent only the syntactic categories of the basic constituents.

\subsubsection{Definite Clause Grammars}

{\em Definite clause grammars\/}, or DCG's, were developed by Fernando
C.\ N.\ Pereira and David H.\ D.\ Warren and presented in~\cite{dcg}.
DCG's constitute a subset of the set of so-called {\em logic
grammars\/}, which are directly inspired by the formalisms and
techniques of predicate logic: the grammar rules of a DCG have the
form of predicates which allow arguments, and thus context dependent
information, to be passed between them.

\subsubsection{Analysis}

The analyzer of {\nash} is based on a definite clause grammar for
language analysis presented by Pereira and Warren in~\cite{dcg}.

The basic idea behind the approach taken, is the application of the
principle of {\em compositionality\/}, that is, the semantic
representation of any constituent is composed from the semantic
representation of its component parts. The computation of the semantic
representation of a sentence therefore relies on

\begin{itemize}
  \item The representation of individual words, given in the
    dictionary. At this level, variables are introduced as well.
  \item The rules describing how the semantic representation of a
    constituent is composed from the semantic representations of its
    constituent parts.
\end{itemize}
For a detailed description of the mechanisms behind the analyzer, see
for instance~\cite{gal},~\cite{pereira} or~\cite{dcg}. The logical
representation scheme employed in {\nash}, {\niks}, is described in
Section~\ref{predicate}.

\subsubsection{Generation}

As mentioned in Section~\ref{aims}, the main goal with {\nash} has
been that its user interface should be exclusively based on {\nal}.
That is, knowledge passing, both from the user to the system and from
the system to the user, should be done in {\nal}. Since {\nash} uses
{\niks}, not {\nal}, when it reasons, it is necessary that the
knowledge be paraphrased from {\niks} into {\nal} before it is
presented to the user. This task is of nature not trivial, and is
known as {\em automatic text generation\/} (see~\cite{gal}).

The approach taken in {\nash} is a rather na\"{\i}ve one, and his
heavily dependent on the structure of {\niks}, but it works fairly
well, and is easy to understand, and thus also to modify. The
cornerstone of the approach is a DCG for paraphrasing basic
{\niks}-structures into simple sentences. Deeply nested structures are
treated recursively in a depth-first manner, and other complex
structures are decomposed into their components, which are then
treated separately. The simple sentences which result from this
process may be conjugated and in other ways combined to form the final
{\nal}-representation. Also, depending on whether the structure being
paraphrased represents a statement or a query, different approaches
are taken. The approach used in making a statement out of the
resulting sentences is fairly straight-forward, based on conjugation
and nesting, whereas the approach for making queries is more complex,
as queries have an inherently different structure from statements.

Some examples of {\niks}-structures and the {\nal}-representations
resulting from the paraphrasing are shown below.

\begin{center}
\begin{tabular}{lcl}
  {\tt man(john)} [query] & $\Rightarrow$ & {\em Is John a man?\/} \\
  {\tt man(john)}         & $\Rightarrow$ & {\em John is a man.\/} \\
  {\tt in(with(live(john),mary),london)} [query] & $\Rightarrow$ &
  {\em Does John live with Mary in London?\/} \\
  {\tt in(with(live(john),mary),london)} & $\Rightarrow$ &
  {\em John lives with Mary in London.\/} \\
  {\tt which(X):(place(X)\&in(live(john),X))} & $\Rightarrow$ &
  {\em Where does John live?\/} \\
  {\tt which(X):(person(X)\&of(mother(X),mary))} & $\Rightarrow$ &
  {\em Who is a mother of Mary?\/} \\
  {\tt exists(X):((ship(X)\&old(A))\&see(john,A))} & $\Rightarrow$ &
  {\em John sees an old ship.\/}
\end{tabular}
\end{center}

\subsection{Limitations}
\label{limnal}

When considering {\nal}, it is important to keep in mind, that it is
no more than a prototype, and can thus be expected neither to be sound
nor complete. Some of the most evident limitations of {\nal} are the
following:

\begin{itemize}
\item No semantic analysis is done. This means that for instance the
  sentences ``{\em Peter is a student in 1992\/}'' and ``{\em Gerard
  is the mayor of Auvers\/}'' are identical as far as their semantical
  representations are concerned, in that the semantical representation
  of a sentence is determined exclusively by the basic constituents of
  which it is composed. As a consequence, no temporal information can
  be represented (as is evident from the example), thus seriously
  reducing the practical usefulness of the system.
\item Of the language, only the present tense is handled. This only
  confirms what was stated above, that no temporal information can be
  represented.
\item Only the third person singular is handled. This means that it is
  impossible to refer to groups of individuals other than by referring
  directly to the class to which they belong, which is hardly useful
  in general applications.
\item In English, compound nouns are not formed by concatenating
  basic words; rather, they are represented as sequences of separate
  words. In {\nal}, a noun has to be a single word, thus imposing
  serious restrictions on the set of legal nouns.
\item Possessives are not handled, neither through possessive pronouns
  nor possessive suffixes. This imposes further restrictions on the
  usefulness of the system.
\item Queries containing dangling prepositions, like in ``{\em Who
  does Sarah play with?\/}'', are not handled at present. This is not
  a major problem, though, in that it can be incorporated rather
  easily.
\item As stated in Section~\ref{naldef}, complex implicit rules may
  result in an unhandled case, that is, a logical representation
  containing more than one implication. This imposes a factor of
  indeterminism on the system, which ought not to be there.
\end{itemize}
