%%%%%%%%%%%%%%%%%%%%%%%%%%%%%% -*- Mode: Latex -*- %%%%%%%%%%%%%%%%%%%%%%%%%%%%
%%% im_noise.tex --- 
%%% Author          : blehr
%%% Created On      : Thu Mar 11 05:38:23 1993
%%% Last Modified By: blehr
%%% Last Modified On: Mon Mar 15 07:12:38 1993
%%% RCS revision    : $Revision$ $Locker$
%%% Status          : In writing....
%%%%%%%%%%%%%%%%%%%%%%%%%%%%%%%%%%%%%%%%%%%%%%%%%%%%%%%%%%%%%%%%%%%%%%%%%%%%%%

\section{Noise Reduction}
\label{image:noise}

Noise reduction, or {\em smoothing\/}, is the process of reducing or,
if possible, removing unwanted spurious effects (noise) that may be
present in an image.  Noise in a digital image may stem from several
sources, such as a bad transmition channel or a poor sampling system.
In this section, noise reduction techniques both in the spatial and
frequency domains are considered.

\subsection{Averaging}
\label{image:noise:averaging}

{\em Averaging\/} is a general technique with which the value at a
given pixel location in an image is replaced by the average value of a
set of pixels, either from the same image, or from a set of images.

\subsubsection{Neighbourhood Averaging}

In {\em neighbourhood averaging\/}, the value at a given pixel
location is replaced by the average value of the pixel in a defined
neighbourhood of the location.  The spatial mask proposed on
page~\pageref{pg:image:spatial:mask} corresponds to neighbourhood
averaging where the neighbourhood is defined to be the 8-neighbours of
the given location plus the location itself.  The general formulation
of neighbourhood averaging of an $N\times N$ image $f$ is
\begin{equation}
\label{eq:averaging:neighbourhood}
  g(x,y)=\frac{1}{M}\sum_{(n,m)\in S}f(n,m)
\end{equation}
for $x,y=0,1,2,\ldots,N-1$, where $S$ is the set of coordinates
defining the neighbourhood of $(x,y)$, and $M$ is the number of points
in the neighbourhood.

The obvious effect of applying Eq.~\ref{eq:averaging:neighbourhood} to
an image, is one of {\em blurring\/} the image.

\subsection{Median Filtering}
\label{image:noise:median}

\subsection{Frequency Domain Methods}
\label{image:noise:frequency}

\subsubsection{Lowpass Filtering}

\subsubsection{Low Frequency Emphasis}

\subsubsection{Removal of Frequency Peaks}
