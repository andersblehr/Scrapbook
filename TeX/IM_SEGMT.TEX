%%%%%%%%%%%%%%%%%%%%%%%%%%%%%% -*- Mode: Latex -*- %%%%%%%%%%%%%%%%%%%%%%%%%%%%
%%% im_segmt.tex --- 
%%% Author          : blehr
%%% Created On      : Thu Mar 11 05:40:23 1993
%%% Last Modified By: blehr
%%% Last Modified On: Tue Mar 30 00:16:02 1993
%%% RCS revision    : $Revision$ $Locker$
%%% Status          : In writing....
%%%%%%%%%%%%%%%%%%%%%%%%%%%%%%%%%%%%%%%%%%%%%%%%%%%%%%%%%%%%%%%%%%%%%%%%%%%%%%

\section{Image Segmentation}
\label{image:segment}

Image segmentation is the process of dividing an image into its
constituent parts or objects.  One may be interested just in dividing
the image into contiguous {\em regions\/}, without paying attention to
what sort of objects the individual regions represent, or one may want
to be able to identify and classify one or several parts of the image
as belonging to specific classes of objects.  This section is only
concerned with the problem of recognizing parts of images as belonging
to continuous regions with certain characteristics.  A branch of image
segmentation which is not touched upon here, is {\em discontinuty
  detection\/}, which aims at detecting points, lines and edges in an
image.  However, in Section~\ref{image:edge}, the problem of {\em edge
  detection\/} is given separate treatment.

\subsection{Thresholding}
\label{image:segment:threshold}

The foundation of the segmentation technique known as {\em
  thresholding\/} is the presumption that objects in an image can be 
segmented out on the basis of their gray levels.  For example, if the 
image consists of a bright object on a dark background, the foreground 
object may be segmented out by classifying all pixels in the image 
whose gray levels are above a certain {\em threshold value\/} $T$ as 
belonging to it, and all other pixels (i.e., those whose gray levels 
are below the threshold) as belonging to the background.  The 
threshold is said to divide the gray scale into two {\em bands\/}.

Generally, thresholding may be viewed as an image operation that
involves tests against a {\em thresholding function\/} $T$ of the form
\begin{equation}
\label{eq:threshold:operator}
  T=T[x,y,p(x,y),f(x,y)]\mbox{,}
\end{equation}
where $f(x,y)$ is the gray level at point $(x,y)$, and $p(x,y)$
denotes some local property of this point.  A thresholded image
$g(x,y)$ is then created by defining (compare
Eq.~(\ref{eq:threshold:1}) on page~\pageref{eq:threshold:1})
\begin{equation}
\label{eq:threshold:2}
  g(x,y)=\left\{
    \begin{array}{ll}
      1 & \mbox{if $f(x,y)>T$} \\
      0 & \mbox{if $f(x,y)\leq T$.}
    \end{array}\right.
\end{equation}

When $T$ depends only on $f(x,y)$, the threshold is called {\em
  global\/}; if it depends on both $f(x,y)$ and $p(x,y)$, it is called
{\em local\/}; and if it, in addition, depends on the spatial
coordinates $x$ and $y$, it is called {\em dynamic\/}.  In this
section, only global thresholding is considered.  That is,
Eq.~(\ref{eq:threshold:operator}) is in the form
\begin{equation}
  T=T[f(x,y)]\mbox{.}
\end{equation}

If a given image $f$ consists of objects that are, say, darker than
their background, global thresholding is a natural approach to
extracting them.  In this case, the gray-level histogram of the image
would have characteristics similar to the one given in
Fig.~\ref{fig:threshold:histogram}.  In the figure, the object and
background pixels are grouped into two dominant modes, one at the dark
end of the spectrum (the object), and one on the bright end of the
spectrum (the background).
\insertepswidth{treshist}{\label{fig:threshold:histogram}Gray-level
  histogram of an image containing a dark object on a bright
  background.}{0.5}
By choosing the threshold value $T$ as indicated in the figure, the
object of interest is easily separated from its background.

By applying Eq.~(\ref{eq:threshold:2}) on the image, the resulting
image will be a binary one, the object being black and its background
white (assuming that 0 in the equation is chosen to correspond to
black, and 1 to correspond to white).

The problem with thresholding is to choose a value for $T$ which
separates the object of interest from other parts of the image well
enough; only in ideal cases are the histogram characteristics of the
image similar to those of Fig.~\ref{fig:threshold:histogram}.
Normally, the gray levels of the image assemble in lumps whose dual
characteristics are hardly discernable.  Consequenctly, care has to be
taken when choosing the threshold level $T$.

\subsection{Region Growing}
\label{image:segment:region}

{\em Region growing\/} is an algorithm which groups pixels or
subregions into larger regions according to criteria of {\em
  similarity\/} and {\em connectivity\/}.  The criterion of similarity
says that only pixels or subregions having a specific set of
properties can be added to the growing region, and the criterion of
connectivity says that only pixels or subregions that constitute
immediate neighbours of the growing region can be added to it.

The simplest approach to region growing is called {\em pixel
  aggregation\/}, where the starting point is a so-called {\em seed
  point\/} from which a region is grown by appending to the seed point
neighbouring pixels having similar properties to it.  It is of course
also possible to apply more than one seed point, each at a different
location in the image, thus having several regions grow
simultaneously.  This is referred to as {\em multiple\/} region
growing, as opposed to {\em single\/} region growing, applying only
one seed point.  In the following, only single region growing is
discussed.  The principles presented are, however, easily extendable
to multiple region growing.

Two immediate problems with region growing are the selection of a seed
point which properly represents the region of interest, and the
selection of a suitable set of criteria for including new points
during the growing process.

If one has {\em a priori\/} knowledge of the properties of the region
of interest, one approach in selecting the seed point is to apply a
search strategy on the image, searching for the first pixel which
satisfies these properties, and then using this pixel as seed point.
With this approach there is of course the danger that the pixel found
does not belong to the region of interest, since single pixels outside
the region of interest may share properties with it.  Thus it is
important to equip the search strategy with a very tight stop
criterion, in order to minimize the chance of erroneously choosing a
non-region seed point (cf.\ Section~\ref{eval:eval:segment}).

A usual similarity criterion for the inclusion of new points during
the growing process is that the gray level of a pixel to be included
not differ from that of the seed point by more than a specified
amount.  The obvious stop criterion for the region growing procedure
is to stop when no more pixels satisfy the criteria for inclusion in
the region.

The advantage of region growing is its degree of automation.  It can
be compared to filling an unevenly shaped pool with water: Once you
start pouring water into it, you can be certain that the water will
protrude into every far off corner of the pool, thus defining a
continuous surface entirely covering the ``region of interest'', the
floor of the pool.  The problem is that you have to know where to
start pouring: The seed point.

\subsection{Template Matching}
\label{image:segment:template}

So far, the focus has been on methods for segmenting out objects in an
image based on a set of characteristics that differentiates the
objects from their background.  These characteristics, however,
include neither {\em geometrical\/} (e.g., shape) nor {\em textural
  properties\/} (e.g., dark and bright regions) of the objects to be
segmented out.  Thus, if one wants to be able to recognize an object
in an image having an arbitrary pattern of gray levels, other
techniques have to be found.

{\em Template matching\/} is, perhaps, one of the most straightforward
methods for locating the presence of an object in an image, based on
specific geometrical and textural properties.  The idea is that an
object with given properties can be detected in an image by matching
the image at different locations against a {\em template\/} having the
sought properties, as illustrated in Fig.~\ref{fig:template}.  This
must be done at every pixel location in the image, and, if the sought
object is allowed to rotate, at every possible orientation.

\inserteps{template}{\label{fig:template}An image (a), and a template
  designed to locate the bull's head (b).
  (From~\protect\cite{template})}

Matching of this kind would be a trivial task if an exact copy of the
template could be expected to be present in the image.  A match would
then be detected when each pixel in the template has the same gray
level as the one it is matched against.  In practice, however, the
image is normally noisy, and thus the sought object has to be found by
maximizing some measure of the degree of match between the template
and the picture.  Some standard measures of the degree of match
between an image $f(x,y)$ and a template $T$ whose gray level at
$(x,y)$ is $t(x,y)$ are (for the sake of generality, integrals are
used; in the digital case these are replaced by summations):
\begin{equation}
\label{eq:max}
  \max_{(x,y)\in T}|f(x,y)-t(x,y)|
\end{equation}
\begin{equation}
\label{eq:savd}
  {\int\int}_{T}|f(x,y)-t(x,y)|dxdy
\end{equation}
\begin{equation}
\label{eq:correlation}
  {\int\int}_{T}[f(x,y)-t(x,y)]^{2}dxdy\mbox{.}
\end{equation}
These measures are all zero for a perfect match, and have high values
for poor matches.  They differ, however, in the type of errors to
which they are sensitive (to see how they differ, see~\cite{digpat}).

A useful and popular match measure that can be derived from
Eq.~(\ref{eq:correlation}) is the {\em correlation coefficient\/}
$\rho$, defined by
\begin{equation}
\label{eq:correlation:coefficient}
  \rho(x,y)=\frac{{\int\int}_{T}f(x,y)t(x,y)dxdy}
  {\sqrt{{\int\int}_{T}f^{2}(x,y)dxdy{\int\int}_{T}t^{2}(x,y)dxdy}}\mbox{.}
\end{equation}
It can be shown that $0\leq \rho(x,y)\leq 1$ for all (x,y) in the
image, with value $1$ being achieved if and only if $f(x,y)\equiv
ct(x,y)$ for some positive constant $c$.  Also, $\rho$ is unaffected
if the gray scale of $f$ is multiplied by a constant, unlike the
measures in Eqs.~(\ref{eq:max}) through~(\ref{eq:correlation}).

As is evident from the above, template matching is rather expensive
computationally.  Correlation may be carried out more efficiently in
the frequency domain, but only if the image and the template are of
the same size, which is seldom the case.  The results obtained also
vary depending on the similarity measure employed.
In~\cite{template}, three measures of match are evaluated as to their
performance in measuring eye position.  The measures evaluated are the
correlation coefficient (Eq.~(\ref{eq:correlation:coefficient})), the
{\em sum of absolute valued differences\/} (SAVD,
Eq.~(\ref{eq:savd})), and a relatively new technique called the {\em
  stochastic sign change criterion\/} (SSC).  In this paper it is
concluded that, due to ``gross misregistrations'', the correlation
coefficient $\rho$ is ``inapplicable as a similarity measure''.  For
the SAVD and SSC measures, the results as to their accuracy are
satisfactory.
