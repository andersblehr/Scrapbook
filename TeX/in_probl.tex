%%%%%%%%%%%%%%%%%%%%%%%%%%%%%% -*- Mode: Latex -*- %%%%%%%%%%%%%%%%%%%%%%%%%%%%
%%% in_probl.tex --- 
%%% Author          : blehr
%%% Created On      : Thu Mar 11 23:03:15 1993
%%% Last Modified By: blehr
%%% Last Modified On: Sun Apr 11 13:57:10 1993
%%% RCS revision    : $Revision: 1.9 $ $Locker:  $
%%% Status          : In writing....
%%%%%%%%%%%%%%%%%%%%%%%%%%%%%%%%%%%%%%%%%%%%%%%%%%%%%%%%%%%%%%%%%%%%%%%%%%%%%%

\section{Problem Definition and Initial Work}
\label{intro:problem}

Originally, it was the intent that I should concern myself with the
design of a complete eye-tracking system.  My task was defined by the
following problem definition, hereafter referred to as the {\em
  overall problem definition\/}:

\paragraph{Overall problem definition:} 
Design and test a complete system for measuring the eye movements of
the subject during neurophysiological recordings from the visual
cortex (monkey) or EEG recordings (man), satisfying the requirements
listed in Section~\ref{intro:motivation}.  The different parts of the
system to be designed are:
\begin{itemize}
\item An eye-tracking program satisfying
  requirements~\ref{req:first}--\ref{req:last} of
  Section~\ref{intro:motivation}.  The program shall run within the
  framework of an IBM-compatible PC, either in the CPU of the PC
  itself or on a special-purpose image processing board subjected to
  it.
\item An interface between the eye-tracking program and the
  registering equipment of the setup, to enable the simultaneous and
  correlated registration of brain activity and gaze direction.
\item A setup to ensure optimal illumination of the subject's eye.  In
  order to minimize the induced activity in the visual system of the
  subject unrelated to the visual task, the illumination is to be in
  the infrared area.
\item A setup supplying the camera with an optimal image of the eye.
  This is to be done with an infrared mirror placed in front of the
  eye, reflecting an infrared image of it to the camera.  The camera
  is to be placed in front of and above the eye, perpendicular to the
  gaze direction.
\end{itemize}
In addition to the design, the following items have to be purchased:
\begin{itemize}
\item A highly IR sensitive camera.  The camera must conform to the
  CCD video standard, and must be able to deliver at least 50,
  preferably 100, half frames per second.
\item All items required to build the illumination and camera setups
  described above.
\item A frame grabbing PC board.  The video input of the grabber has
  to conform to the CCD video standard, and it must be able to receive
  and digitize as many half frames per second as the camera delivers.
  The digitized images are to have a resolution of $512\times 512$
  pixels with 256 gray levels.  Alternatively,
\item a special-purpose image processing board, in case of deciding on
  letting the eye-tracking program run on such a board instead of in
  the CPU of the host PC.  The board has to fulfil the requirements
  listed above, and its processor(s) must be easily programmable.
\end{itemize}
The total cost of the above items may not exceed the amount given in
requirement~\ref{req:last} in Section~\ref{intro:motivation}.
\vspace*{0.1cm}

\noindent As the resolution of the system to be designed depends on
the resolution of the digitized images, it would have been desirable
with an image resolution of at least $1024\times 1024\times 256$.
However, the number of pixels in an image with this resolution is too
high from a real-time image processing point of view to be
sustainable.  Thus it was decided on a lower resolution of $512\times
512\times 256$, as indicated above.

\subsubsection{Working with the Initial Problem Definition}

During the first months of my thesis work, I spent a lot of time
trying to find reasonable approaches to problems pertaining to the
non-algorithmic, fundamental aspects of this definition.  The most
important of these can be summarized as
\begin{itemize}
\item Does the requirement of speed imply that the final system has to
  employ a special-purpose image processing board?  In that case,
  which requirements does the image processing board have to satisfy?
  Which image processing boards are available, and what can they do?
  Do they satisfy the technical and low-cost requirements?  From which
  vendors are they available?  How long is the delivery time for the
  different boards?  How easily programmable are they?
\item In case an algorithm is found that is fast enough to allow the
  program to run in CPU of the host PC, which requirements does the
  frame grabbing board have to satisfy?  Which frame grabbing boards
  are available, and what do they offer?  Do they satisfy the
  technical and low-cost requirements?  From which vendors are they
  available?  How long is the delivery time for the different boards?
\item Which are the requirements that the camera has to satisfy, in
  addition to the requirements of sensitivity, speed, and low cost?
  Which highly infrared sensitive cameras complying to the requirement
  of speed are available?  And from which vendors?  How do the video
  signals from the different cameras conform with the CCD standard,
  alternatively with the video input channels of the different frame
  grabbing and image processing boards that are available?
\item How shall the problem of eye illumination be approached?  Are
  infrared light-emitting diodes sufficient?  Which is the best
  positioning of the diodes relative to the eye to ensure optimal
  illumination?  How many diodes are necessary to illuminate the eye
  sufficiently?  Does the illumination constitute a danger to the eye?
\end{itemize}

Much time was spent in order to arrive at answers to some of these
questions, partly on the telephone with different companies offering
various cameras, frame grabbing and image processing boards, inquiring
about how well the different cameras and boards would suit the given
problem, ordering information material and inquiring about other
vendors offering similar products; and partly browsing through a
considerable amount of information material, trying to land at a
conclusion as to which vendor offered the most cost-efficient
solution.  Also, different approaches to the illumination problem were
investigated, diodes were acquired and tested, and simple illumination
setups were built.

After some months of working with the overall problem definition, it
became clear that the time available would not suffice to design a
complete eye-tracking system, because neither a suitable image
processing board complying to the low-cost requirement, nor a
corresponding frame grabbing board or camera had been found.
Particularly, as suitable image processing boards turned out to be far
too costly, it was decided that an attempt should be made at
developing an eye-tracking algorithm that was fast enough to run in
the CPU of a host PC without assistance from an external processor.
To supply the necessary test images, a small frame-grabber was
purchased that received images from an ordinary video camera (cf.\ 
Section~\ref{intro:env} below).  Thus a revised problem definition,
hereafter referred to as the {\em algorithmic problem definition\/},
or simply the {\em problem definition\/}, was found to be necessary:

\paragraph{Algorithmic problem definition:} 
Design, implement and, as extensively as possible, test an algorithm
to, on the basis of a given image of the eye, determine the position
of the (centre of the) pupil, relative to some given reference point.
The algorithm must to be fast enough to run in the CPU of the
computer, and has to satisfy the following requirements:
\begin{itemize}
\item The position of the pupil is to be determined with an accuracy
  corresponding to 1 pixel in the digitized image.
\item Whenever the pupil cannot be found (due to blinking,
  disturbances, etc.), an error signal is to be given.
\item The total running time of the algorithm may not exceed 20 ms in
  the worst case.
\end{itemize}
Whether the accuracy requirement above complies with
requirement~\ref{req:2} in Section~\ref{intro:motivation} depends on
the resolution of the image and on the fraction of the image occupied
by the pupil.  Preferably, the image should have a resolution of
$512\times 512$ pixels with 256 gray levels, and the eye should fill
the entire image.  Also, it has to be determined whether the final
algorithm actually is to run in the CPU of the host PC, since data
transfer rates may make it necessary to have it run in a processor
near to the storage device to which the digitized images are written.
