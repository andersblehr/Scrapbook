%%%%%%%%%%%%%%%%%%%%%%%%%%%% -*- Mode: Latex -*- %%%%%%%%%%%%%%%%%%%%%%%%%%%%%%
%%% commands.tex --- 
%%% Author          : Anders Blehr
%%% Created On      : Sat Apr 25 17:54:59 1992
%%% Last Modified By: blehr
%%% Last Modified On: Sun Apr  4 18:42:33 1993
%%% RCS revision    : $Revision$ $Locker$
%%% Status          : Subjected to continuous change....
%%%%%%%%%%%%%%%%%%%%%%%%%%%%%%%%%%%%%%%%%%%%%%%%%%%%%%%%%%%%%%%%%%%%%%%%%%%%%%
%%% History 		
%%% 20-Mar-1993		blehr	
%%%    La til \inserttwoeps, som setter inn to EPS-filer ved siden av
%%%    hverandre, kaller dem (a) hhv. (b), og gir dem en felles
%%%    \caption.  Skiller ogsaa mellom mine og Anders' saker - Anders'
%%%    er plassert nederst i filen.
%%% 23-Feb-1993		blehr	
%%%    Tok bort \hentps og \henpsbredde, siden dvips ikke forstaar
%%%    psfig.  La til \inserteps og \insertepswidth istedet.  Begge
%%%    disse krever Encapsulated PostScript (.eps).

%%
%%  Mine saker
%%

\input{epsf}

\newcommand{\SS}{\bf S}
\newcommand{\octopus}{OCTOPUS}
\newcommand{\odd}{\mbox{\scriptsize odd}}
\newcommand{\even}{\mbox{\scriptsize even}}
\newcommand{\fft}{FFT}
\newcommand{\nit}{NTH}

%
% Kommandoer for importering av EPS-filer (Encapsulated PostScript).
%

%
% \inserteps[2]
%
% Bruksmaate:
%   \inserteps{filnavn}{\label{fig:labref}figurtekst}
%
% X-utstrekningen til figuren settes til 0.7 ganger tekstbredden.
%

\newcommand{\inserteps}[2]{\insertepswidth{#1}{#2}{0.7}}

%
% \insertepswidth[3]
%
% Bruksmaate:
%   \insertepswidth{filnavn}{\label{fig:labref}figurtekst}{bredde}
%
% `bredde' er et tall mellom 0 og 1 som angir broekdel av
% tekstbredden som figuren skal ha.
%

\newcommand{\insertepswidth}[3] {
  \begin{figure}[tb]
    \leavevmode
    \epsfxsize #3\textwidth
    \begin{centering}
      \epsfbox{c:/user/blehr/fag/diplom/rapport/figurer/eps/#1.eps}
      \caption{#2}
    \end{centering}
  \end{figure}
  }

%
% \inserttwoeps[3]
%
% Bruksmaate:
%   \inserttwoeps{filnavn1}{filnavn2}{\label{fig:labref}figurtekst}
%
% Setter inn to EPS-filer ved siden av hverandre.  Hver av dem faar
% bredde lik 0.3 ganger tekstbredden, og figurteksten kommer sentrert
% under begge.
%

\newcommand{\inserttwoeps}[3] {
  \begin{figure}[tb]
    \epsfxsize 0.3\textwidth
    \makebox[0.45\textwidth][r]{
      \epsfbox{c:/user/blehr/fag/diplom/rapport/figurer/eps/#1.eps}}
    \makebox[0.4\textwidth][r]{
      \epsfbox{c:/user/blehr/fag/diplom/rapport/figurer/eps/#2.eps}}\\
    \hspace*{0.28\textwidth}(a)\hspace*{0.38\textwidth}(b)
    \caption{#3}
  \end{figure}
  }

%%
%%  Torp's saker
%%

% Et tastetrykk, eller en kombinasjon av to taster.
\newcommand{\tast}[1]{\fbox{\rule{0mm}{1.5ex}{\tt #1}}}
\newcommand{\totast}[2]{\tast{#1}-\tast{#2}}

% Binder en taste-sekvens mot en kommando.
\newcommand{\keybinding}[2]{#1 executes {\em #2}. See Section~\ref{#2-func}.}

% Kommando som definerer ny funksjon.
% Parameter 1 : Funksjonsnavn, Kommer til venstre under en strek.
%           2 : Parametre. Kommer til h|yre for navnet i kursiv
%           3 : Command|Function|Variable    Kommer til hoyre.
%           4 : Forklaringen. Den quote's.

% \newenvironment{params}{{\large Parameters:}\begin{description}}{\end{description}}

\newcommand{\returns}[1]{
        \begin{description}
          \item [Returns : ]#1
        \end{description}}

% Dette er environmentet som brukes ved forklaring av funksjoner,
% programmer eller variable.
\newenvironment{func}[3]{\pagebreak[3] \label{#1-func} %
  {\large\it  #1\hspace{1cm} #2 \hfill {\bf #3}}
\begin{quote}}{\end{quote}}
