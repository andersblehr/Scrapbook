%%%%%%%%%%%%%%%%%%%%%%%%%%%%%% -*- Mode: Latex -*- %%%%%%%%%%%%%%%%%%%%%%%%%%%%
%%% im_concl.tex --- 
%%% Author          : blehr
%%% Created On      : Thu Mar 11 05:41:23 1993
%%% Last Modified By: blehr
%%% Last Modified On: Mon Mar 22 06:06:33 1993
%%% RCS revision    : $Revision$ $Locker$
%%% Status          : In writing....
%%%%%%%%%%%%%%%%%%%%%%%%%%%%%%%%%%%%%%%%%%%%%%%%%%%%%%%%%%%%%%%%%%%%%%%%%%%%%%

\section{Conclusion}
\label{image:concl}

As mentioned in Section~\ref{image:intro:structure}, the purpose of
this section is to compare the different techniques discussed in this
chapter as to their individual applicability to the problem at hand.
This comparison is found in Section~\ref{image:concl:comp} below.  The
key criterion for applicability is {\em low computational cost\/}.  In
Section~\ref{image:concl:approach}, a conclusion is drawn as to which
approach to take in developing an algorithm to solve the problem,
based on the foregoing comparison.

\subsection{Applicability of the Presented Techniques}
\label{image:concl:comp}

In the following, when comparing the computational cost of the
different techniques, it is assumed that the given image is of size
$N\times N$, and that the technique in question is carried out on the
{\em entire\/} image.  The cost is measured in terms of $N$.

\subsubsection{Noise Reduction}

As pointed out in Section~\ref{image:spatial}, applying spatial
operators on an image generally requires $O(N^{2})$ operations.  Thus
all the spatial noise reduction schemes presented in
Section~\ref{image:noise}, neighbourhood averaging, averaging of
multiple images, and median filtering, perform in $O(N^{2})$ time.

Of the two averaging techniques, averaging of multiple images can be
discarded without delay, owing to its requiring several input images
of the same scene, a requirement which is unsustainable in our case.
Neighbourhood averaging has the drawback of attenuating the edges in
the image, thus making location of the pupil based on edge detecting
schemes difficult.

As pointed out in Section~\ref{image:noise:median}, median filtering
turns out to be clearly superior to averaging techniques, both with
respect to removing noise from an image, and to maintaining edges in
the image.  In addition, median filtering is, when carefully
implemented (e.g., by using binary trees to sort the values),
computationally cheaper than averaging schemes, in that it requires
mostly comparisons and no multiplications.

From Section~\ref{image:frequency:fourier}, p.~\pageref{pg:fft:O}, we
have that performing an {\fft} algorithm on an image requires
$O(N^{2}\log_{2}N)$ complex operations.  Thus image processing
techniques operating in the frequency plane generally perform in
minimum $O(N^{2}\log_{2}N)$ time, assuming that the image first has to
be Fourier transformed.  This pertains to both the highpass filtering
noise reduction schemes described in
Section~\ref{image:noise:frequency}, simple highpass filtering and
selective filtering.

Simple highpass filtering has a similar effect to neighbourhood
averaging, namely blurring of the image, and consequently attenuating
of edges.  The selective filtering scheme would be better suited, but
then it is required to know the Fourier transform of the superimposed
noise function aforehand.  This is seldom the case in real-time
applications, where the noise normally has a random distribution.

The main drawback of these frequency domain schemes (and of any
frequency domain scheme), however, is their immense computational
cost, eleborated upon in Section~\ref{image:frequency:fourier},
p.~\pageref{pg:fft:O}.  Thus frequency domain noise reduction schemes
implemented in software can be discarded as an alternative in
real-time applications (hardware implementations may be fast enough
to be sustainable).

\subsubsection{Detecting the Pupil}

\subsection{Chosen Approach}
\label{image:concl:approach}

