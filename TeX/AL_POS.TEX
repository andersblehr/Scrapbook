%%%%%%%%%%%%%%%%%%%%%%%%%%%%%% -*- Mode: Latex -*- %%%%%%%%%%%%%%%%%%%%%%%%%%%%
%%% al_pos.tex --- 
%%% Author          : blehr
%%% Created On      : Sun Mar 28 02:27:59 1993
%%% Last Modified By: blehr
%%% Last Modified On: Thu Apr  1 06:51:10 1993
%%% RCS revision    : $Revision$ $Locker$
%%% Status          : In writing....
%%%%%%%%%%%%%%%%%%%%%%%%%%%%%%%%%%%%%%%%%%%%%%%%%%%%%%%%%%%%%%%%%%%%%%%%%%%%%%

\section{Determining the Position of the Pupil}
\label{algo:pos}

The principle behind the line oriented edge detection algorithm
employed in {\octopus} was given a fairly thorough presentation in
Section~\ref{eval:approach:edge}.  Thus the focus in this section is
on the different application specific aspects of the general
algorithm.  In Section~\ref{algo:pos:operation}, an operational
description of how line oriented edge detection has been tailored to
suit the specific needs of {\octopus} is given, and in
Section~\ref{algo:pos:operators}, the gradient operators employed to
detect the pupil contour are presented.  In Section~\ref{algo:pos:O},
the number of operations required to determine the position of the
pupil is derived.

\subsection{Operational Description}
\label{algo:pos:operation}

The original formulation of line oriented edge detection, as stated in
Section~\ref{eval:approach:edge}, says that the algorithm operates by
proceeding, from the origin of operation, in different directions,
applying some edge detecting operator successively on every pixel
encountered in each direction, and, if an edge point is recognized,
registering the coordinates of this point.  By registering the
coordinates of the oppositely positioned edge points of two
perpendicular lines intercepting at the origin of operation, an
estimate for the position of the pupil can be computed using the
procedure illustrated in Fig.~\ref{fig:compute}.  By taking the
average of several estimates thus computed, an improved estimate of
the position is obtained.  Accordingly, as stated in the basic
formulation of the {\octopus} algorithm in
Section~\ref{eval:approach:algo}, line oriented edge detection must be
applied to as high a number of line pairs as allowed by the time
available, in order to obtain as accurate an estimate of the centre
position of the pupil as possible.

\subsubsection{Computational Limitations}

\insertepswidth{lines}{\label{fig:lines}Horizontal, vertical and
  diagonal lines in an image, as opposed to lines having other
  inclinations.  $(x_{o},y_{o})$ designates the origin of operation.
  Note that $x_{1}=y_{1}$, whereas $x_{2}\neq y_{2}$.}{0.35}

From a computational point of view, there are only four directions in
an image in which lines are easily traced, namely horizontally,
vertically, and in the two diagonal directions, as illustrated in
Fig.~\ref{fig:lines}.  Horizontally, only the $x$ variable is varied,
and vertically only the $y$ variable, whereas diagonally, both the $x$
and $y$ variables are varied, either with equal signs (southeastwards,
assuming that the origin is in the upper left corner of the image) or
with opposite signs (northeastwards).  In all other directions, the
two variables are varied unequally, and thus require overhead
computations.

\subsection{Operators Employed}
\label{algo:pos:operators}

\subsection{Number of Operations}
\label{algo:pos:O}

