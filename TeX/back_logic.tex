%%%%%%%%%%%%%%%%%%%%%%%%%%%%%% -*- Mode: Latex -*- %%%%%%%%%%%%%%%%%%%%%%%%%%%%
%%% back_logic.tex --- 
%%% Author          : Anders Blehr
%%% Created On      : Tue Apr 28 23:01:04 1992
%%% Last Modified By: Anders Blehr
%%% Last Modified On: Mon May  4 06:03:37 1992
%%% RCS revision    : $Revision: 1.1 $ $Locker:  $
%%% Status          : Unknown, Use with caution!
%%%%%%%%%%%%%%%%%%%%%%%%%%%%%%%%%%%%%%%%%%%%%%%%%%%%%%%%%%%%%%%%%%%%%%%%%%%%%%

\section{Logic Foundations}
\label{logic}

For a formal definition of the terms used in this section, see for
instance~\cite{logic} or~\cite{nilsson}.

\subsection{Predicate Logic}
\label{predicate}

{\em First order predicate logic\/}, or simply {\em predicate
logic\/}, is a branch of the general science of logic, and offers a
powerful way of deriving new knowledge from old by means of what is
called {\em mathematical deduction\/}.

The counterpart in predicate logic of sentences in natural language is
{\em well-formed formulas\/}, or simply {\em formulas\/}. Well-formed
formulas are built from {\em constants\/}, {\em variables\/}, {\em
functors\/}, {\em predicate symbols\/} and recursively from other
well-formed formulas.  Variables can be either {\em free\/} or {\em
bound\/}. If they are bound, they are either {\em existentially\/} or
{\em universally quantified\/}. The symbols denoting existential and
universal quantification are $\exists$ and $\forall$, respectively.
Other symbols used in predicate logic, are the logical {\em
connectives\/}: {\em conjunction\/} ($\wedge$), {\em disjunction\/}
($\vee$), {\em negation\/} ($\neg$) and {\em implication\/}
($\rightarrow$).

Given the above, the natural language sentence ``Socrates is a man''
can be represented by the well-formed formula

\begin{center}
  {\em man\/}({\em socrates\/}).
\end{center}

\noindent Similarly, the following well-formed formulas represent,
from top to bottom, the sentences ``all men are mortal'', ``some pigs
have wings'', ``every man loves a woman'' and ``not all birds can
fly'':

\begin{center}
  $(\forall x)(\mbox{\em man\/}(x)\rightarrow\mbox{\em
  mortal\/}(x))$\\ $(\exists x)(\mbox{\em pig\/}(x)\wedge\mbox{\em
  have\/}(x,\mbox{\em wings\/}))$\\ $(\forall x)(\mbox{\em
  man\/}(x)\rightarrow(\exists y)(\mbox{\em woman\/}(y)\wedge\mbox{\em
  love\/}(x,y)))$\\ $(\exists x)(\mbox{\em
  bird\/}(x)\wedge\neg\mbox{\em fly\/}(x))\mbox{\ or\ }\neg(\forall
  x)(\mbox{\em bird\/}(x)\rightarrow\mbox{\em fly\/}(x))$.
\end{center}

\noindent The two representations given for the last sentence are of
course equivalent. A common, but not universal, pattern of predicate
logic (\cite{logic}) is recognized here, that the universal quantifier
very often is followed by an implication and that the existential
quantifier very often is followed by a conjunction. This last property
has been utilized in {\nash} (see Section~\ref{naldef}).

\subsubsection{The Internal Knowledge Representation Scheme of {\nash}}

The internal knowledge representation scheme of {\nash} ({\niks}) is
based on a scheme presented in~\cite{dcg}. The syntax of {\niks} is
very similar to that of predicate logic. In addition to the universal
and existential quantifiers, the connectives conjunction, implication
and are employed, with a, for obvious reasons, slightly different
notation: {\tt forall(X)}, {\tt exists(X)}, {\tt \&}, {\tt =>} and
{\tt not} for $(\forall x)$, $(\exists x)$, $\wedge$, $\rightarrow$
and $\neg$, respectively. In addition, {\niks} defines a
non-associative binary infix operator {\tt :} for associating
quantifiers with the logical expressions they quantify. Thus the
examples given in above will look as follows in {\niks}:

\begin{center}
  {\tt forall(X):(man(X)=>mortal(X))\\
  exists(X):(pig(X)\&have(X,wings))\\
  forall(X):(man(X)=>exists(Y):(woman(Y)\&love(X,Y)))\\
  exists(X):(bird(X)\&not fly(X))} or {\tt not
  forall(X):(bird(X)=>fly(X))}
\end{center}

All the connectives employed in {\niks} are defined as operators. In
order of ascending precedence, they are defined as follows: {\tt =>}
is non-associative binary infix, {\tt \&} is right-associative binary
infix and {\tt not} is prefix. The operator of highest precedence is
{\tt :}.

\subsection{Resolution}
\label{resolution}

The variant of mathematical deduction commonly employed in connection
with predicate logic is called {\em resolution\/}, and is based on an
inference rule called the {\em resolution principle\/}. Resolution
produces proofs by {\em refutation\/}, that is, to prove that a
statement is valid an attempt is made to show that the negation of the
statement produces a contradiction with the known statements. For a
thorough description of how the resolution procedure works,
see~\cite{nilsson} and~\cite{rich}. Here we just mention that it
requires that all statements be in {\em clause form\/}, that is, they
should contain no quantifiers and only one connective, the disjunction
$\vee$. An algorithm for transforming statements into clause form is
given in~\cite{rich}.

To give an informal presentation of how resolution works, imagine that
we want to prove that Socrates is mortal given that we know that he is
a man and that all men are mortal. I.e., our knowledge base contains
the two clauses (which are also given above, although not in clause
form)

\begin{center}
  {\em man\/}({\em socrates\/})\\ $\neg\mbox{\em
  man\/}(x)\vee\mbox{\em mortal\/}(x)$,
\end{center}

\noindent and we want to prove the clause

\begin{center}
  {\em mortal\/}({\em socrates\/}).
\end{center}

\noindent To accomplish this, we try to refute the clause (or rather,
the {\em goal\/}) $\neg\mbox{\em mortal\/}(\mbox{\em socrates\/})$
(i.e., prove that it causes a contradiction with what we already
know). When comparing the goal with the knowledge base, we see that it
can be resolved with the literal {\em mortal\/}({\em x\/}), unifying
$x$ with {\em socrates\/} and leaving us with the contradiction

\begin{center}
  {\em man\/}({\em socrates\/})\\ $\neg${\em man\/}({\em
  socrates\/}).
\end{center}

\noindent In the next iteration of the the resolution procedure, the
empty clause ($\Box$) will be produced, which means that we have
succeeded in refuting the goal and thus in proving that Socrates
indeed {\em is\/} mortal.
