%%%%%%%%%%%%%%%%%%%%%%%%%%%%%% -*- Mode: Latex -*- %%%%%%%%%%%%%%%%%%%%%%%%%%%%
%%% nash_limit.tex --- 
%%% Author          : Anders Blehr
%%% Created On      : Mon Apr 27 17:28:24 1992
%%% Last Modified By: Anders Blehr
%%% Last Modified On: Mon May  4 06:23:58 1992
%%% RCS revision    : $Revision: 1.1 $ $Locker:  $
%%% Status          : Unknown, Use with caution!
%%%%%%%%%%%%%%%%%%%%%%%%%%%%%%%%%%%%%%%%%%%%%%%%%%%%%%%%%%%%%%%%%%%%%%%%%%%%%%

\section{Future Extensions}
\label{future}

The limitations of {\nal} and {\nash} are elaborated upon in
Sections~\ref{limnal} and~\ref{limsys}. Some possible future
extensions for the system as a whole are the following:

\begin{itemize}
\item Increase the reasoning powers of the system, by including
  semantic analysis. In addition, it would be desirable to equip it
  with mechanisms supporting temporal reasoning, which is a major
  requirement that has to be fulfilled if the aim of the system is
  that it be of practical use.
\item Extend the definition of {\nal} also to handle the past tense,
  other inflections than the third person singular, compound nouns,
  possessives and dangling prepositions.
\item Provide the system with an automated input mechanism for
  asserting rules and facts when it is initialized. For instance, the
  system may be initialized with a file as input parameter, containing
  the required rules and facts.
\item The bug causing the system to loop forever when the chosen rule
  does not apply, has to be fixed. How this can be done, is described
  in Section~\ref{limsys}.
\end{itemize}
Generally, it would be desirable to improve both the definition of the
language as well as the reasoning mechanisms involved, in order to
make the system of practical as well as theoretical interest. An
approach to improving the reasoning mechanism, would be to base it on
Tore Amble's powerful logical representation scheme {\solon}, rather
than {\niks}.
