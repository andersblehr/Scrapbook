%%%%%%%%%%%%%%%%%%%%%%%%%%%%%% -*- Mode: Latex -*- %%%%%%%%%%%%%%%%%%%%%%%%%%%%
%%% image.tex --- 
%%% Author          : blehr
%%% Created On      : Thu Mar 11 06:09:05 1993
%%% Last Modified By: blehr
%%% Last Modified On: Thu Mar 11 06:14:11 1993
%%% RCS revision    : $Revision$ $Locker$
%%% Status          : Unknown, Use with caution!
%%%%%%%%%%%%%%%%%%%%%%%%%%%%%%%%%%%%%%%%%%%%%%%%%%%%%%%%%%%%%%%%%%%%%%%%%%%%%%

\chapter{Digital Image Processing}
\label{image}

In this chapter, an introduction to the vast field of digital image
processing is given, with an inclination towards the problem at hand,
namely recognition and location of the pupil within a digital image of
the eye.  A thorough problem definition is given in
Section~\ref{image:intro}, along with a presentation of the general
fields of digital image processing and an introduction to spatial and
frequency plane analysis.  In Section~\ref{image:noise}, the problem
of noise reduction is addressed; Section~\ref{image:enhance} discusses
image enhancement techniques; in Section~\ref{image:pattern}, the
problem of pattern recognition is presented, and in
Section~\ref{image:edge}, some edge detection approaches are
discussed.  In Section~\ref{image:concl}, a conclusion is drawn as to
which technique to choose for the problem at hand.

%%%%%%%%%%%%%%%%%%%%%%%%%%%%%% -*- Mode: Latex -*- %%%%%%%%%%%%%%%%%%%%%%%%%%%%
%%% im_intro.tex --- 
%%% Author          : blehr
%%% Created On      : Thu Mar 11 05:37:29 1993
%%% Last Modified By: blehr
%%% Last Modified On: Sun Apr 11 12:42:53 1993
%%% RCS revision    : $Revision: 1.14 $ $Locker:  $
%%% Status          : In writing....
%%%%%%%%%%%%%%%%%%%%%%%%%%%%%%%%%%%%%%%%%%%%%%%%%%%%%%%%%%%%%%%%%%%%%%%%%%%%%%

\section{Introduction}
\label{image:intro}

\subsection{Intent}
\label{image:intro:intent}

The intent of this chapter is to give a relatively broad presentation
of different techniques that may be applied to solve the problem
defined by the algorithmic problem definition in
Section~\ref{intro:problem}.  The presentation is intended to indicate
some possible approaches and thus to form a theoretical foundation on
which to base the development of a solution as complete and
satisfactory as possible.

Because of the requirement of speed, the techniques presented have
been selected on the basis of being relatively cheap computationally.
Most of them are also well-established, in that they are thoroughly
described in the literature.  Many reliable, but, due to relatively
complex mathematics, computational expensive methods, such as the
Canny/Deriche approach to edge
detection~(\cite{canny},~\cite{deriche}), are not discussed.

In order to maintain an overall view, the presentation is kept on a
relatively fundamental level.  Still, it has been aimed at making it
thorough enough to suffice as a basis on which to draw a conclusion
as to a suitable approach to the actual problem.

\subsection{Categories in Image Processing}
\label{image:intro:categories}

The general field of automated image processing may be divided into
four main categories.  These categories are~(\cite{digim}):

\begin{description}
\item[Image digitization:] In order to be in a form suitable for
  processing in a computer, an image has to be discretized (digitized)
  both spatially and in amplitude.  {\em Image digitization\/} is the
  process of converting continuous (real) images into digitized form.
\item[Image enhancement and restoration:] Often an image is unsuited
  for machine and/or human perception, due to its being blurred,
  noisy, or in some other way degraded.  The field of improving
  degraded images for machine or human perception is generally
  referred to as {\em image enhancement and restoration\/}.
\item[Image encoding:] Techniques for representing an image or the
  information contained in the image with fewer bits than required
  for the ``raw'' digitized image are categorized as {\em image
    encoding\/} techniques.
\item[Image segmentation, representation, and description:] An image
  is normally composed of contiguous regions, each region being
  characterized by a set of properties that the pixels belonging to
  the region share.  These regions correspond to the objects that make
  up the image.  Often, one is interested in segmenting out one or
  more of the objects that are present in an image.  This can be done
  by applying techniques classified as {\em image segmentation\/}
  techniques.  In addition, one may be interested in representing the
  constituent parts of an image in forms suitable for further computer
  processing, and to describe them in terms of parts and properties.
  This is referred to as {\em image representation\/} and {\em
    description\/}, respectively.
\end{description}

The field of {\em digital\/} image processing is concerned with the
latter three of these categories, in that already digitized images are
required as input to a computer.  In this chapter, the focus is on the
fields of {\em noise reduction\/}, and {\em image segmentation\/}, as
described in the next section.

\subsection{Structure of the Chapter}
\label{image:intro:structure}

In this chapter, both spatial and frequency domain techniques are
presented.  A theoretical introduction to the spatial domain and terms
pertaining to it is given in Section~\ref{image:spatial}.  The
theoretical foundations of frequency domain techniques, the Fourier
transform and the convolution theorem, are given a relatively
extensive presentation in Section~\ref{image:frequency}, along with a
basic description of their use in image processing.

A usual problem when processing digital images, is that they often
have been subjected to some sort of noise.  In
Section~\ref{image:noise}, some approaches to the problem of reducing
the amount of noise present in a given image are presented.  

Since the problem at hand is one of recognizing and locating a given
object in an image (the pupil), the problem of decomposing an image
into constituent regions representing different objects is elaborated
upon in Section~\ref{image:segment}.  

A branch of the field of image segmentation is the field that is
concerned with locating discontinuities in an image, such as lines and
edges.  Since an obvious approach to locating the pupil in an image is
to try to locate the transition in brightness between the pupil and
the iris, the field of edge detection is given separate treatment in
Section~\ref{image:edge}.

\subsubsection{References}

Most of the material in this chapter is from~\cite{digim}.  Another
main source is~\cite{digpic}.  The material in the section on template
matching (Section~\ref{image:segment:template}) is mainly
from~\cite{digpat} and~\cite{template}, the former having supplied
additional material to other sections of the chapter as well.  A last
source for additional material has been~\cite{digbild}.

%%%%%%%%%%%%%%%%%%%%%%%%%%%%%% -*- Mode: Latex -*- %%%%%%%%%%%%%%%%%%%%%%%%%%%%
%%% im_noise.tex --- 
%%% Author          : blehr
%%% Created On      : Thu Mar 11 05:38:23 1993
%%% Last Modified By: blehr
%%% Last Modified On: Sun Apr 11 13:16:48 1993
%%% RCS revision    : $Revision: 1.12 $ $Locker:  $
%%% Status          : In writing....
%%%%%%%%%%%%%%%%%%%%%%%%%%%%%%%%%%%%%%%%%%%%%%%%%%%%%%%%%%%%%%%%%%%%%%%%%%%%%%

\section{Noise Reduction}
\label{image:noise}

Noise reduction, or {\em smoothing\/}, is the process of reducing or,
if possible, removing unwanted spurious effects (noise) that may be
present in an image.  Noise in a digital image may stem from several
sources, such as a bad transmition channel or a poor sampling system.
In this section, noise reduction techniques both in the spatial and
frequency domains are considered.

Whether noise reduction measures have to be taken in the final
implementation of the complete eye-tracking system is not clear at
present.  By ensuring optimal light conditions and using ``clean''
transmission channels, relatively noiseless images ought to be
obtainable.  This section is meant as an indicator towards some
possible noise reduction schemes if it at a later stage of the project
should turn out that the noise present in the supplied images cannot
be ignored.

\subsection{Averaging}
\label{image:noise:averaging}

{\em Averaging\/} is a general technique with which the value at a
given pixel location in an image is replaced by the average value of a
set of pixels, either from the same image, or from a set of images.

\subsubsection{Neighbourhood Averaging}

In {\em neighbourhood averaging\/}, the value at a given pixel
location is replaced by the average value of the pixel in a defined
neighbourhood of the location.  The spatial mask proposed in
Section~\ref{image:spatial:mask} corresponds to neighbourhood
averaging where the neighbourhood is defined to be the 8-neighbours of
the given location plus the location itself.  The general formulation
of neighbourhood averaging of an $N\times N$ image $f$ is
\begin{equation}
\label{eq:averaging:neighbourhood}
  g(x,y)=\frac{1}{M}\sum_{(n,m)\in S}f(n,m)
\end{equation}
for $x,y=0,1,2,\ldots,N-1$, where $S$ is the set of coordinates
defining the neighbourhood of $(x,y)$ and $M$ is the number of points
in the neighbourhood.

Assuming that the noise values at each point in the image are
independent samples from a distribution whose mean is 0, it is clear
that the standard deviation of the noise present is reduced by
neighbourhood averaging (for a formal proof of this,
see~\cite{digpic}), thus indicating an improved signal-to-noise (S/N)
ratio.

An obvious side-effect of neighbourhood averaging is one of blurring
the image.  The degree of blurring is proportional to the size of the
averaging neighbourhood used.  If the noise is finer grained than the
smallest details of interest in the image, the blurring can be made
negligible by choosing a correspondingly small averaging
neighbourhood.  If this is not the case, or if it for some reason is
desirable to use a larger averaging neighbourhood, the averaging can
be ``switched off'' in regions with large variations in gray level
(lines, edges, etc.) by using the following criterion instead of
Eq.~(\ref{eq:averaging:neighbourhood}):
\begin{equation}
  g(x,y)=\left\{
    \begin{array}{ll}
      \frac{1}{M}\sum_{(m,n)\in S}f(m,n) &
        \mbox{if $\left|f(x,y)-\frac{1}{M}\sum_{(m,n)\in
            S}f(m,n)\right|<T$} \\
             & \\
      f(x,y) & \mbox{otherwise,}
    \end{array}\right.
\end{equation}
where $T$ is a specified nonnegative threshold.

\subsubsection{Averaging of Multiple Images}

When several images $g_{i}$ which have been formed by adding a noise
function $\eta$ to an original image $f$, and the noise $\eta$ at each
location $(x,y)$ is uncorrelated with mean 0, an image $\overline{g}$
with reduced noise level can be computed by assigning to each pixel
the average value of the gray levels of the corresponding pixels in
all the noisy images at hand:
\begin{equation}
  \overline{g}(x,y)=\frac{1}{M}\sum_{i=1}^{M}g_{i}(x,y)\mbox{,}
\end{equation}
$M$ being the number of noisy images.  Since the mean of the noise
function $\eta(x,y)$ is 0, the expected value of $\overline{g}$ at any
point is given by
\begin{equation}
  E\{\overline{g}(x,y)\}=f(x,y)\mbox{.}
\end{equation}
If $\sigma_{\eta(x,y)}$ denotes the standard deviation of the noise
function $\eta(x,y)$, the standard deviation
$\sigma_{\overline{g}(x,y)}$ at any point in the averaged image is 
given by the relation
\begin{equation}
  \sigma_{\overline{g}(x,y)}=\frac{1}{\sqrt{M}}\sigma_{\eta(x,y)}\mbox{,}
\end{equation}
thus indicating that as $M$ increases, $\overline{g}$ approaches the
original image $f$.

The obvious drawback of this method is the requirement of having $M$
images of the same scene at hand.  This may be the case with grainy
photographs or ``snowy'' TV-frames, but in the case of noisy images in
a real-time image application, this would hardly be the most
cost-efficient approach.

\subsection{Median Filtering}
\label{image:noise:median}

{\em Median filtering\/} is a technique with which the gray level of
each pixel in an image is replaced by the {\em median\/} of the gray
levels in a neighbourhood of that pixel instead of by the average, as
was the case in the previous section.  The median of a set of values
is the value that is such that half of the values in the set is
smaller than or equal to it, and the other half is greater than or
equal to it.  For example, suppose that a $3\times 3$ neighbourhood at
location $(x,y)$ (Fig.~\ref{fig:neighbour}) has the values
$(51,49,47,53,98,50,51,52,49)$, where $98$ is the value at $(x,y)$.
This value corresponds to a single-pixel spike, and can be assumed to
represent a noise point.  A {\em median filter\/} placed at $(x,y)$
sorts the pixel values in this neighbourhood and assigns to the pixel
at $(x,y)$ the median of this sorted set of values.  In the example,
the values are sorted as $(47,49,49,50,50,51,52,53,98)$.  The median
is seen to be $50$, and thus the pixel corresponding to the spike is
assigned the value 50 by the filter.

The effect of median filtering is one of forcing points with very
distinct gray levels to be more like their neighbours, thus
eliminating intensity spikes like the one in the example above.  The
superiority of median filtering over e.g.\ neighbourhood averaging in
noise removal is clearly demonstrated by the example on p.\ 163
in~\cite{digim}.

\subsection{Frequency Domain Methods}
\label{image:noise:frequency}

By observing that noise in an image mostly is high-frequency in
content, it is obvious that by properly manipulating the Fourier
transform of the image, noise reduction can be achieved.

\subsubsection{Lowpass Filtering}

The most general and also most obvious approach to noise reduction
in the frequency domain is lowpass filtering, as touched upon in
Section~\ref{image:frequency:image}.  Since not only the noise, but
also gray level transitions in the image are high-frequency in
content, lowpass filtering, as was the case also for neighbourhood
averaging, has the side-effect of blurring the image.  By using a
lowpass Butterworth filter (Fig.~\ref{fig:lowpass}(b)) instead of an
ideal lowpass filter (Fig.~\ref{fig:lowpass}(a)), the degree of
blurring is reduced.  This is because the ``tail'' of the Butterworth
filter lets a fairly high amount of high-frequency information
through.  In addition, an ideal lowpass filter tends to introduce what
is commonly referred to as {\em ringing\/} into the processed image,
thus further reducing its quality (for an explanation of ringing, see
for instance~\cite{digim}).

An approach equivalent to lowpass filtering is {\em low frequency
  emphasis\/}, emphasizing low frequencies without filtering out the
high frequency end of the Fourier spectrum completely (compare {\em
  high frequency emphasis\/}, Section~\ref{image:frequency:image}).
This technique also tends to blur the image, but the blurring is not
so predominant as for pure lowpass filtering.

\subsubsection{Selective Filtering}

If the noise constitutes a regular pattern which has been superimposed
on the image, this pattern tends to have most or all of its energy
concentrated at small spots in the frequency plane.  Thus, if the
Fourier transform of this superimposed pattern is known, it can be
removed entirely simply by frequency-domain subtraction of its Fourier
transform from the Fourier transform of the noisy image.  The loss of
information in the processed image will be negligible.

If the nature of the superimposed noise is only partly known, but one
is able to locate the spots in the frequency plane where the energy of
the noise seems to be concentrated, one can reduce the amount of noise
in the image without serious loss of information by suppressing these
spots.  This is called {\em selective filtering\/}.

\input{im_hance}
\input{im_pattn}
%%%%%%%%%%%%%%%%%%%%%%%%%%%%%% -*- Mode: Latex -*- %%%%%%%%%%%%%%%%%%%%%%%%%%%%
%%% im_edge.tex --- 
%%% Author          : blehr
%%% Created On      : Thu Mar 11 05:40:55 1993
%%% Last Modified By: blehr
%%% Last Modified On: Sun Apr 11 13:29:44 1993
%%% RCS revision    : $Revision: 1.8 $ $Locker:  $
%%% Status          : In writing....
%%%%%%%%%%%%%%%%%%%%%%%%%%%%%%%%%%%%%%%%%%%%%%%%%%%%%%%%%%%%%%%%%%%%%%%%%%%%%%

\section{Edge Detection}
\label{image:edge}

As mentioned in the previous section, {\em discontinuity detection\/}
is a branch of image segmentation.  To the area of discontinuity
detection belong the fields of {\em point detection\/}, {\em line
  detection\/}, and {\em edge detection\/}.  In this section, the
focus is on edge detection, as it constitutes by far the most common
approach for detecting discontinuities in gray level.  An {\em edge\/}
is defined as the boundary between two regions characterized by having
relatively distinct gray levels.

A difficulty with edge detection as an approach to image segmentation
is that the detected edges often have gaps in them, due to places
where transitions between regions are not abrupt enough that an edge
be detected.  In addition, if the image is noisy, edges may be
detected at points which are not parts of region boundaries, as
discussed below.  Thus the detected edges will not necessarily form
closed, connected curves which surround closed, connected regions.

\subsection{Thresholding}
\label{image:edge:threshold}

The thresholding technique introduced in
Section~\ref{image:segment:threshold} can, with slight modifications,
be applied to detect edges in an image, as described in~\cite{digim}.
The idea is to choose a threshold $T$, based on some criteria,
dividing the gray scale of the image into two distinct bands.  Then
the image is scanned in the $x$ and $y$ directions separately.  Each
time a change in gray level from one band to the other occurs, this
indicates the presence of a boundary point.  The procedure performs in
two passes as follows:

\paragraph{Pass 1:} For each row in $f(x,y)$, create a corresponding
row in an intermediate image $g_{1}(x,y)$, using the following
relation: 
\begin{equation}
  g_{1}(x,y)=\left\{
    \begin{array}{ll}
      L_{E} & \mbox{if the levels of $f(x,y)$ and $f(x,y-1)$ are in} \\
            & \mbox{different bands of the gray scale,} \\
      L_{B} & \mbox{otherwise,}
    \end{array}\right.
\end{equation}
where $L_{E}$ and $L_{B}$ are specified boundary and background
levels, respectively.

\paragraph{Pass 2:} For each column in $f(x,y)$, create a
corresponding column in an intermediate image $g_{2}(x,y)$ using the
following relation:
\begin{equation}
  g_{2}(x,y)=\left\{
    \begin{array}{ll}
      L_{E} & \mbox{if the levels of $f(x,y)$ and $f(x-1,y)$ are in} \\
            & \mbox{different bands of the gray scale,} \\
      L_{B} & \mbox{otherwise.}
    \end{array}\right.
\end{equation}
\vspace*{0.1cm}

\noindent The desired image, consisting of the boundary points of the
object of interest on a uniform background, is then given by the
relation:
\begin{equation}
  g(x,y)=\left\{
    \begin{array}{ll}
      L_{E} & \mbox{if $(g_{1}(x,y)=L_{E})\vee (g_{2}(x,y)=L_{E})$,} \\
      L_{B} & \mbox{otherwise.}
    \end{array}\right.
\end{equation}
As is the case with thresholding in general, care has to be taken when
choosing the threshold value $T$.  

As is evident from its formulation, this procedure assumes that the
edges in the given image are characterized by monotonously increasing
or decreasing functions.  This is seldom the case in real
applications.  If the image has been subjected to additive noise, so
that in regions with gray levels in the proximity of $T$ pixel values
are randomly distributed on both sides of $T$, which is normally the
case, this method will detect an edge point each time a transition
from one band to another occurs, thus making it difficult to
distinguish between an actual edge, and edges detected due to noise.
In the worst case an edge will be depicted as a ``bulb'' of layered
edge fragments.  If the edges of interest are relatively sharp,
however, that is, they approximate a step edge rather than a ramp
edge, this problem is less dominant.

\subsection{Gradient Operators}
\label{image:edge:gradient}

As edge points in an image are characterized as points where the gray
level of the image changes rapidly, an obvious approach to edge
detection is to apply some kind of differentiation on the image.  The
most common approach to differentiation in image processing
applications is the {\em gradient\/}.  The gradient of a
two-dimensional function $f(x,y)$ is defined as the vector
\begin{equation}
  \vec{G}[f(x,y)]=\left[
    \begin{array}{c}
      \underline{\partial f} \\
      \partial x \\ \\
      \underline{\partial f} \\
      \partial y
    \end{array}\right]=\left[
    \begin{array}{c}
      G_{x} \\ \\
      G_{y}
    \end{array}\right]\mbox{.}
\end{equation}
$\vec{G}[f(x,y)]$ always points in the direction of the maximum rate
of increase of $f$ at $(x,y)$, and the rate of increase per unit
distance in this direction, denoted $G[f(x,y)]$, is given by
\begin{equation}
\label{eq:gradient:sqrt}
  G[f(x,y)]=|\vec{G}|=\sqrt{G_{x}^{2}+G_{y}^{2}}\mbox{.}
\end{equation}
The magnitude function $G[f(x,y)]$ is generally referred to as the
{\em gradient\/} of $f$, to avoid constantly having to refer to it as
``the magnitude of the gradient''.  To reduce computational costs, the
gradient is commonly approximated using absolute values:
\begin{equation}
\label{eq:gradient:abs}
  G[f(x,y)]\approx|G_{x}|+|G_{y}|\mbox{.}
\end{equation}
For digital images, $G_{x}$ and $G_{y}$ are approximated using
differences.  One approach is to use first-order differences in a
$2\times 2$ neighbourhood.  A typical approximation in this respect is
\begin{equation}
  G_{x}\simeq f(x,y)-f(x+1,y)
\end{equation}
\begin{equation}
  G_{y}\simeq f(x,y)-f(x,y+1)\mbox{,}
\end{equation}
yielding the following expression for $G[f(x,y)]$:
\begin{equation}
\label{eq:gradient:simple}
  G[f(x,y)]\simeq|f(x,y)-f(x+1,y)|+|f(x,y)-f(x,y+1)|\mbox{.}
\end{equation}
This approximation uses first-order differences in the horizontal and
vertical directions about $(x,y)$ to compute $G[f(x,y)]$.  Another
approximation, commonly referred to as the {\em Roberts gradient\/},
uses differences symmetrical about the imaginary interpolation point
$(x+\frac{1}{2}, y+\frac{1}{2})$:
\begin{equation}
\label{eq:gradient:roberts}
  G[f(x,y)]\simeq|f(x,y)-f(x+1,y+1)|+|f(x+1,y)-f(x,y+1)|\mbox{.}
\end{equation}
The approximations in Eqs.~(\ref{eq:gradient:simple})
and~(\ref{eq:gradient:roberts}) use the differences between two pairs
of adjacent pixels to compute the gradient.  Thus, for about the same
reasons given in Section~\ref{image:edge:threshold}, these
approximations tend to be sensitive to noise.

\subsubsection{The Sobel Operators}

Using a $3\times 3$ neighbourhood to compute the gradient has the
advantage of increased smoothing over $2\times 2$ operators.  Consider
the following definitions of $G_{x}$ and $G_{y}$:
\begin{equation}
\label{eq:sobel:x}
  \begin{array}{lll}
    G_{x} & = & [f(x-1,y+1)+2f(x,y+1)+f(x+1,y+1)]\ - \\
          &   & [f(x-1,y-1)+2f(x,y-1)+f(x+1,y-1)]
  \end{array}
\end{equation}
\begin{equation}
\label{eq:sobel:y}
  \begin{array}{lll}
    G_{y} & = & [f(x+1,y-1)+2f(x+1,y)+f(x+1,y+1)]\ - \\
          &   & [f(x-1,y-1)+2f(x-1,y)+f(x-1,y+1)]\mbox{.}
  \end{array}
\end{equation}

\insertpdfwidth{sobelop}{\label{fig:sobelop}Spatial masks used to
  compute $G_{x}$ (a) and $G_{y}$ (b), as defined by
  Eqs.~(\protect\ref{eq:sobel:x})
  and~(\protect\ref{eq:sobel:y}).}{0.4}

The spatial masks shown in Figs.~\ref{fig:sobelop}(a) and (b)
implement $G_{x}$ and $G_{y}$, respectively.  These two masks are
commonly referred to as the {\em Sobel operators\/}.  The responses of
these two operators at any point $(x,y)$ in an image are combined
using Eq.~(\ref{eq:gradient:sqrt}) or Eq.~(\ref{eq:gradient:abs}) to
obtain the gradient at that point.  Convolving these masks with an
image yields the gradient at all points $(x,y)$ in the image, the
result often being referred to as a {\em gradient image\/}.

\subsubsection{The Laplacian}

Higher order derivative operators can also be used to detect edges in
an image.  The {\em Laplacian\/} operator $L[f(x,y)]$ is a
second-order derivative operator which is direction insensitive, as
opposed to the Sobel-operators.  In the continuous case, the Laplacian
is defined by the equation
\begin{equation}
  L[f(x,y)]=\frac{\partial^{2}f}{\partial x^{2}}+
  \frac{\partial^{2}f}{\partial y^{2}}\mbox{.}
\end{equation}
In the digital case, it is approximated by
\begin{equation}
\label{eq:laplacian}
  L[f(x,y)]\simeq f(x,y-1)+f(x-1,y)+f(x+1,y)+f(x,y+1)-4f(x,y)\mbox{.}
\end{equation}
A mask implementing Eq.~(\ref{eq:laplacian}) is shown in
Fig.~\ref{fig:laplacian}.  Note that Eq.~(\ref{eq:laplacian}) implies
that $L[f(x,y)]$ takes on both positive and negative values.  If only
positive values are desired, the absolute value can be used.  Note
also that the Laplacian is zero in constant areas and on the ramp
section of edges, as expected of a second-order derivative operator.

As a consequence of its being a second-order derivative operator,
however, the Laplacian tends to be unacceptably sensitive to noise.
It also does not respond as strongly to edges as do the other
operators described thus far.  On the other hand, it responds far more
strongly to single points, lines and line ends than these do.  It can
also be applied to detect whether a given pixel is on the dark or
bright side of an edge.

\insertpdfwidth{laplace}{\label{fig:laplacian}Spatial mask used to
  compute the Laplacian as defined by
  Eq.~(\protect\ref{eq:laplacian})}{0.15}

\subsection{Highpass Filtering}
\label{image:edge:highpass}

Convolving an image with any of the gradient operators described above
is equivalent to highpass filtering in the frequency domain.  Thus it
ought to be possible to detect edges in an image by highpass filtering
the Fourier transform of the image directly.  Highpass filtering the
Fourier transform of a given image indeed has effects similar to those
of applying the above operators.  The usefulness of direct highpass
filtering as an edge detection technique is, however, questionable, in
that the responses returned are relatively weak as compared to the
results obtained using the above techniques.  This is clearly
demonstrated in Figs.~9 and~10 on pp.~281--283 in~\cite{digpic}.

\subsection{The Hough Transform}
\label{image:edge:hough}

The last approach to edge detection discussed here, is the so-called
{\em Hough transform\/}.  The discussion below is directed towards
applying the Hough transform to detect straight lines and edges in an
image, but the principle can be extended to detecting any curve or
edge that can be described by the equation $g(\vec{x},\vec{c})=0$,
where $\vec{x}$ is a vector of coordinates and $\vec{c}$ is a vector
of coefficients.

Consider a point $(x_{i},y_{i})$ in an image.  An infinite number of
lines satisfying the equation $y_{i}=ax_{i}+b$ for varying values of
$a$ and $b$ pass through this point.  However, the equation
$b=-x_{i}a+y_{i}$, which is equivalent to the previous one, describes
a single line in the {\it ab\/} plane (often referred to as {\em
  parameter space\/}) for a fixed pair $(x_{i},y_{i})$.  Analogously,
a second image point $(x_{j},y_{j})$ also has a line in parameter
space associated with it.  This line intercepts the line associated
with $(x_{i},y_{i})$ at a point $(a',b')$, where $a'$ is the slope and
$b'$ the intercept of the line in the {\it xy\/} plane (the image)
containing both $(x_{i},y_{i})$ and $(x_{j},y_{j})$.  All points on
this line will have lines in parameter space which intercept at
$(a',b')$.

By associating with each point $(a_{i},b_{i})$ in parameter space an
{\em accumulator cell\/}, it is possible to keep track of how many
points in the {\it xy\/} plane lie on a given line.  This is done by
letting, for every point $(x_{k},y_{k})$ in the image, $a$ vary along
the entire $a$ axis in parameter space, and for each $a_{l}$ solve for
$b_{l}$ using the relation $b_{l}=-x_{k}a_{l}+y_{k}$.  The value of
the accumulator cell associated with $(a_{l},b_{l})$ is then
incremented by one, and at the end of the procedure, a value of $M$ in
the accumulator cell associated with $(a_{i},b_{j})$ corresponds to
$M$ points in the image lying on the line $y=a_{i}x+b_{j}$.

A problem with describing a line with the equation $y=ax+b$ is that
both slope and intercept approach infinity as the line approaches a
vertical position.  This problem is circumvented by using the {\em
  normal\/} description $x\cos\theta+y\sin\theta=\rho$ of the line
instead.  The only difference in using this representation when
constructing the set of accumulator cells is that the intercepts in
parameter space ($\rho\theta$ plane) are between sinusoidal curves
instead of straight lines.

As mentioned above, these principles can be applied to detect any
curve or edge which can be described by the relation
$g(\vec{x},\vec{c})=0$.  The equation
$(x-c_{1})^{2}+(y-c_{2})^{2}=c_{3}^{2}$, for example, describes a
circle and can thus be used analogously to the above procedure to
detect circles in an image.  One possible application would be to
detect the (circular) pupil in an image of the eye (cf.\
Section~\ref{eval:eval:edge}).

If the number of points in the image equals $N^{2}$ (i.e., the image
is of size $N\times N$) and the $a$ axis is divided into $K$ equal
increments, the number of computations needed for the above procedure
is $KN^{2}$, since for each image point $(x_{i},y_{i})$, $a$ is varied
from $a_{1}$ to $a_{K}$ to obtain the corresponding values for $b$.
Thus the Hough transform performs in $O(N^{2})$ time, assuming that
$K\ll N$.

\input{im_concl}
