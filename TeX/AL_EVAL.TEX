%%%%%%%%%%%%%%%%%%%%%%%%%%%%%% -*- Mode: Latex -*- %%%%%%%%%%%%%%%%%%%%%%%%%%%%
%%% al_eval.tex --- 
%%% Author          : blehr
%%% Created On      : Sun Mar 28 02:28:57 1993
%%% Last Modified By: blehr
%%% Last Modified On: Tue Apr  6 19:28:42 1993
%%% RCS revision    : $Revision$ $Locker$
%%% Status          : In writing....
%%%%%%%%%%%%%%%%%%%%%%%%%%%%%%%%%%%%%%%%%%%%%%%%%%%%%%%%%%%%%%%%%%%%%%%%%%%%%%

\section{Evaluation}
\label{algo:eval}

In this section, the {\octopus} eye-tracking algorithm described in
the previous sections is evaluated with respect to the algorithmic
problem definition given in Section~\ref{intro:problem}.  First, the
overall number of operations for the algorithm is derived in
Section~\ref{algo:eval:O}.  In Section~\ref{algo:eval:test}, an
implementation (previously referred to as the {\em current\/}
implementation) is run on the set of test images presented in
Fig.~\ref{fig:testimages} and the results returned are discussed, with
respect to both accuracy and time consumption.  Lastly in
Section~\ref{algo:eval:improve}, some suggestions as to how to improve
the performance and accuracy of the algorithm are presented.

\subsection{Overall Number of Operations}
\label{algo:eval:O}

In Section~\ref{algo:seek:O}, it was shown that the swimming octopus
search algorithm, described in Section~\ref{algo:seek} and
implementing Step 1 in the basic algorithm formulation given in
Section~\ref{eval:approach:algo}, performs in $O(N)$ time when applied
to an $N\times N$ image.  In Section~\ref{algo:pos:O} it was shown
that the position determining algorithm described in
Section~\ref{algo:pos}, implementing Step 2 of the basic algorithm
formulation, performs in $O(N)$ time.  Thus it can be concluded that
the {\octopus} eye-tracking algorithm, being composed of the given
implementations of Steps 1 and 2 of the basic algorithm formulation,
performs in $O(N)$ time.  Taking into account that the number of
locations in the image at which the edge detection operators of
Fig.~\ref{fig:masks} have to employed, assuming relatively good image
quality, is $O(1)$ (cf.\ Section~\ref{algo:pos:O}), the advantage of
{\octopus} over a ``traditional'' approach in terms of computational
cost is evident.

\subsection{Test Results}
\label{algo:eval:test}

\begin{figure}[tb]
  \epsfxsize 0.3\textwidth
  \paragraph{}
  \makebox[0.425\textwidth][r]{
    \epsfbox{c:/user/blehr/fag/diplom/rapport/figurer/eps/posbl01.eps}}
  \makebox[0.4\textwidth][r]{
    \epsfbox{c:/user/blehr/fag/diplom/rapport/figurer/eps/posbr01.eps}}\\
  \hspace*{0.28\textwidth}(a)\hspace*{0.38\textwidth}(b)
  \paragraph{}
  \makebox[0.425\textwidth][r]{
    \epsfbox{c:/user/blehr/fag/diplom/rapport/figurer/eps/posbl02.eps}}
  \makebox[0.4\textwidth][r]{
    \epsfbox{c:/user/blehr/fag/diplom/rapport/figurer/eps/posbr02.eps}}\\
  \hspace*{0.28\textwidth}(c)\hspace*{0.38\textwidth}(d)
  \paragraph{}
  \makebox[0.425\textwidth][r]{
    \epsfbox{c:/user/blehr/fag/diplom/rapport/figurer/eps/posbl05.eps}}
  \makebox[0.4\textwidth][r]{
    \epsfbox{c:/user/blehr/fag/diplom/rapport/figurer/eps/posbr03.eps}}\\
  \hspace*{0.28\textwidth}(e)\hspace*{0.38\textwidth}(f)
  \caption{\label{fig:results}Returned estimates of the pupil position
    in the images in Fig.~\protect\ref{fig:testimages}.  The
    coordinates of the estimates relative to the upper left image
    corner are: (a) $x=65$, $y=32$; (b) $x=59$, $y=44$; (c) $x=49$,
    $y=44$; (d) $x=71$, $y=42$; (e) $x=54$, $y=39$; (f) $x=54$,
    $y=44$.}
\end{figure}

The major amount of testing was done while developing and implementing
{\octopus}.  Six of the images with which I worked during that time
are given in Fig.~\ref{fig:testimages}; in addition I had another four
images at my disposal.  How these images were acquired is described in
Section~\ref{algo:intro:images}.  The tests performed at this stage
were a vital part of the development process, pointing at weaknesses
in the current algorithm formulation, suggesting solutions to
difficult problems and in general guiding the development process
towards sounder and more time-efficient algorithm formulations.
Figs.~\ref{fig:compare}, \ref{fig:landscape}, \ref{fig:lake},
\ref{fig:profile}, \ref{fig:mysobel} and~\ref{fig:myprofile} were all
obtained using software primarily designed to sustain this process.
In the following, the focus will be on the final formulation of
{\octopus} as given in the previous sections, and on its ability to
satisfy the requirements listed in the algorithmic problem definition
given in Section~\ref{intro:problem}.  Two points will be addressed,
corresponding to the two main aspects of the problem definition,
namely time consumption and accuracy.

\subsubsection{Accuracy}

In Fig.~\ref{fig:results}, the position estimates returned by
{\octopus} when supplied with the images in Fig.~\ref{fig:testimages}
are shown.  As is seen, the actual extent of the pupil in the images
can hardly be determined visually, which imply relatively weak pupil
edges.  Moreover, the edges of the bright reflexes, which are seen to
cover portions of the pupil in all images, evidently constitute the
sharpest and most clearly defined edges present.  As pointed out in
Section~\ref{algo:intro:images}, these reflexes are of a non-additive
nature, that is, the information contained in the regions they cover
is lost, and accordingly they cannot be removed.  Hence it is
unavoidable that the position estimates returned by {\octopus} are
influenced by these reflexes.  Taking this and the inaccuraccies due
to the chosen value for $T_{e}$ not being appropriate for all images
(cf.\ Section~\ref{algo:pos:operators}, pg.~\pageref{pg:TEproblems})
into account, it is evident that the difference between the actual
pupil centre and the position estimate returned by {\octopus} cannot
be used as an accuracy measure.  Actually, due to the poor image
quality, it would be practically impossible with some of the images to
manually determine the actual pupil centre.

\begin{table}[tb]
  \begin{center}
    \begin{tabular}{|c|l|}               \hline
      (a) & (65,31), (65,32)          \\ \hline
      (b) & (59,44), (60,44), (60,43) \\ \hline
      (c) & (49,43), (49,44), (50,44) \\ \hline
      (d) & (71,41), (72,41)          \\ \hline
      (e) & (54,39)                   \\ \hline
      (f) & (55,44)                   \\ \hline
    \end{tabular}
  \end{center}
  \caption{\label{tab:estimates}Position estimates obtained from 
    the figures in Fig.~\protect\ref{fig:results}.}
\end{table}

Consequently, another accuracy measure has to be employed.  Since the
stop criterion of the iterating position determining algorithm is that
the overall position estimate not change from one iteration to the
next, a natural choice is to use the variation in the estimates given
for one image as an accuracy measure.  Ideally and in compliance with
the requirement of accuracy given in the problem definition, this
variation should be zero both horizontally and vertically.  In other
words, {\octopus} should always return the same position estimate when
supplied with the same image.  This is, however, not the case with all
the given images.  The estimates obtained with the different images
are given in Table~\ref{tab:estimates}.  The letters refer to
Fig.~\ref{fig:results}.  As is seen, only for images (e) and (f) does
{\octopus} satisfy the requirement of accuracy.  For images (a) and
(d), two estimates are returned, and for images (b), (c) and (d),
three different estimates are returned.  It is noted, however, that
the maximum deviation from one estimate to another is not more than 1
pixel, either horizontally, vertically or both.  Still, with the low
resolution of the given images, 1 pixel corresponds to approximately
5\% of the pupil diameter, an error which hardly can be called
tolerable, even without the given accuracy requirement.

This error can be ascribed to two factors.  For one, the low
resolution of the images limits the number of different intermediate
estimates that can be obtained during the iteration process.
Consequently, relatively few iterations are necessary for a final
estimate to be obtained, thus reducing the ``weight'' of this
estimate.  Accordingly, depending on the location of the initial
origin of operation, different final estimates may be obtained which
would have converged if the number of iterations were higher.
Secondly, the technique of moving the origin of operation to the
current overall estimate every two iterations may cause the overall
estimate to converge at a location constituting the centre of the
pupil as seen along the detection lines from this location, but which
may not constitute the centre as seen from another location.  However,
once the overall estimate has landed at this possibly erroneous
location, it will remain there no matter how many consequent
iterations are performed.  In Section~\ref{algo:eval:improve} below, a
possible solution to this problem is suggested.

\subsubsection{Time Consumption}

The time measures presented here were obtained by letting {\octopus}
run 2000 times on the same image and then dividing the total time for
the two steps of the algorithm by this number in order to obtain an
estimate for the time required to perform the algorithm once.  The
results thus obtained are presented in Table~\ref{tab:time}.  $t_{s}$
designates the search time, corresponding to Step 1 of the basic
algorithm formulation, and $t_{p}$ the position determining time,
corresponding to Step 2 of the basic algorithm formulation.  $t$
designates the overall time for the algorithm.  All times are in ms.
The letters refer to Fig.~\ref{fig:results}.  Clearly, the requirement
of determining the position of the eye within 20 ms is satisfied.

From the table, rather large variations in the measured times are
observed.  The variations in $t_{s}$ can be ascribed to variations in
the number of times the octopus jumps into the pupil lake without
recognizing the location where it settles as a pupil point.  For image
(e), the particularly long time needed to locate a pupil point is
attributed to its having a relatively small pupil lake\footnote{With
  lake level $T_{l}=18$, that is, cf.\ Section~\ref{algo:seek:idea}.},
actually forcing the octopus to settle at a location where it hs one
of its arms in a pond on the shore of the pupil lake (cf.\ 
Section~\ref{algo:seek:octopus}).  For $t_{p}$, the variations can be
ascribed to the varying number of iterations necessary to arrive at a
final estimate.  The fact that the times constitute the averages of
2000 performances of the algorithm implies that some of the images
require more iterations for a final estimate to be arrived at than do
others.

\begin{table}[tb]
  \begin{center}
    \begin{tabular}{|l||c|c|c|c|c|c|}                      \hline
              & (a)  & (b)  & (c)  & (d)  & (e)  & (f)  \\ \hline\hline
      $t_{s}$ & 0.28 & 0.22 & 0.24 & 0.36 & 0.22 & 0.19 \\ \hline
      $t_{p}$ & 2.89 & 0.98 & 1.61 & 1.40 & 3.05 & 1.30 \\ \hline
      $t$     & 3.17 & 1.20 & 1.85 & 1.76 & 3.27 & 1.49 \\ \hline
    \end{tabular}
  \end{center}
  \caption{\label{tab:time}Time consumption of the {\octopus}
    eye-tracking algorithm.  All times are in ms.}
\end{table}

\subsection{Improving the Algorithm}
\label{algo:eval:improve}

