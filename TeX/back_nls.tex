%%%%%%%%%%%%%%%%%%%%%%%%%%%%%% -*- Mode: Latex -*- %%%%%%%%%%%%%%%%%%%%%%%%%%%%
%%% back_nls.tex --- 
%%% Author          : Anders Blehr
%%% Created On      : Sat Apr 25 19:58:14 1992
%%% Last Modified By: Anders Blehr
%%% Last Modified On: Mon May  4 06:32:44 1992
%%% RCS revision    : $Revision: 1.1 $ $Locker:  $
%%% Status          : In progress
%%%%%%%%%%%%%%%%%%%%%%%%%%%%%%%%%%%%%%%%%%%%%%%%%%%%%%%%%%%%%%%%%%%%%%%%%%%%%%

\section{Natural Language Systems}
\label{nls}

\subsection{What Constitutes a Natural Language?}
\label{nla}

Natural languages, as opposed to artificial languages (e.g.\
programming languages), are the languages human beings use in their
daily life to communicate all sorts of information about the world.
Thus Livian, Sorb, Basque, Norwegian, English and all other known
languages (as well as those not yet discovered) constitute natural
languages.

What characterizes natural languages is that they are never stable.
They are the result of a continued evolution, adapting to the ever
changing set of demands characterized by the cultural as well as
social setting in which each language is used, adopting words and
constructs from other languages, but also evolving in what may seem to
be a completely random manner.  This evolution has by no means come to
an end. On the contrary, the large amount of information available in
today's society combined with the extreme ease of communication has
caused an acceleration of this process that is unparalleled in
history. As a result of this evolution, natural languages have come to
be extremely powerful, both with respect to expressivity and
flexibility. On the other hand, the nondeterminism and inherent
ambiguity of natural languages, although being useful to humans in
their daily use of the language, cause serious problems for automated
natural language systems.

\subsubsection{Nondeterminism and Ambiguity}

An inherent feature of natural languages is that it is not possible
{\em a priori\/} to determine the set of well-formed sentences of
which a language consists. This can be summarized as follows
(from~\cite{gal}):

\begin{itemize}
  \item The vocabulary of a natural language is not completely known,
    in particular because of the existence of specialized vocabularies:
    technical, medical, local, etc.
  \item The set of constructions is itself not completely
    predetermined.
  \item The set of senses attributed to each word is also not
    completely predetermined, especially because a word often does not
    really have a precise sense outside a particular context.
\end{itemize}

In addition, identical sentences can contain different meanings
depending on the context in which they occur. Take for instance the
sentence

\begin{center}
  {\em He saw her shaking hands}.
\end{center}

Whether he (whoever {\em he\/} may be; this is an example of what is
called {\em anaphoric references\/}, i.e., the meaning of a sentence
depends on knowledge gained from previous sentences) saw her shaking
hands with someone, or saw that her hands were shaking, is not evident
from the sentence. (In speech this ambiguity is usually avoided due to
the fact that human beings use stress and intonation as a means of
conveying the intended meaning to the receiver(s) of their utterances.
This is an aspect of the act of communicating which is completely lost
in written language).

Another phenomenon is that to extract the correct meaning from
sentences, it is not always sufficient only to have knowledge about
the meanings of each word. Consider the sentences

\begin{itemize}
  \item {\em John saw the boy in the park with a telescope.}
  \item {\em John saw the boy in the park with a dog.}
  \item {\em John saw the boy in the park with a statue.}
\end{itemize}

In these three sentences, only the last word of the last prepositional
phrase differs, and from a merely syntactic viewpoint it is not
possible to deduce whether in each case this prepositional phrase is
to be attributed to John's seeing act, the boy or the park. In fact,
in each case, all attributions can be made to make sense if one
carefully defines the contexts in which they appear. On the other
hand, it is not very probable that John used a dog to see the boy in
the park, or that the boy was carrying a statue with him through the
park. Still it is important, although one knows that it is most likely
that John used a telescope to see the boy, not to exclude the
possibility that he in fact {\em did\/} use the dog (i.e., he was
blind and his guide dog was trained to give him a certain signal when
it saw a boy in the park).

In other words, to extract the intended meaning from a sentence, it is
necessary for the receiver, in addition to having semantic knowledge
about each word, also to have knowledge about the world in which the
language is used, and to be able to (unconsciously) assign probability
measures to each possible interpretation of the sentence in order to
arrive at a decision as to which one is the right one.

\subsection{Natural Language Processing}
\label{nlpr}

Natural Language Processing ({\nlp}) is the field of which the
objective is to make systems that are able to understand and reason on
the basis of natural languages. It is useful to divide the entire
{\nlp} problem into two tasks, of which the first is a subset of the
second (from~\cite{rich}):

\begin{itemize}
  \item Processing written text, using lexical, syntactic and semantic
    knowledge of the language as well as the required real world
    information. 
  \item Processing spoken language, using all the information needed
    above, plus additional knowledge about phonology as well as enough
    added information to handle the further ambiguities that occur in
    speech.
\end{itemize}

The field of speech recognition is mainly concerned with being able to
deduce from audio signals the sequence of words which make up an
utterance. Since this sequence is readily given in written text, it is
only natural that by far the most research has been done in the field
of processing written language, of which {\nlp} more or less has
become a synonym.

The principal areas of research in {\nlp} are (from~\cite{gal}):

\begin{itemize}
  \item Developing and modeling linguistic systems.
  \item Conceiving and implementing models and systems of {\nlp}.
  \item Evaluating such systems from the point of view of
    human-machine interfaces.
\end{itemize}

The necessity of developing new linguistic systems is due to the fact
that existing linguistic systems have usually been developed without
the needs of {\nlp} in mind. That is, to be able to develop a working
model of a linguistic system, it is necessary that the linguistic
system be suited for modeling.

\subsubsection{The Process of Understanding Natural Language}

The term {\em understand\/} in this context is to be understood as the
ability of an {\nlp}-system to respond to statements and queries given
to it in natural language as if these responses were the result of a
human-like reasoning process. The extent to which an {\nlp}-system
succeeds in seeming to understand and act according to its input is
closely connected with its ability to act in correspondence with the
principles of {\em cooperativity\/}, described briefly in
Section~\ref{coop}, and thoroughly treated in~\cite{torulf}.

The process of understanding natural language (written text) can be
divided into the following individual, and more or less independent,
steps:

\begin{description}
\item[Morphological Analysis] \begin{sloppypar} Individual words are
  decomposed into their components, inflected words into their stems,
  and derived words are traced back to their sources. \end{sloppypar}
  \item[Syntactic Analysis] The string of words is analyzed to see
    whether they represent correct sentences in the given language. If
    they do, they are transformed into structures showing how the
    individual words of each sentence relate to each other.
  \item[Semantic Analysis] The structures created during the
    syntactic analysis are analyzed to extract from them the meaning
    of the sentences. Two important issues must be addressed
    (from~\cite{rich}):

    \begin{itemize}
      \item Map individual words into appropriate objects in the
        knowledge base or database. If no such mapping is found, the
        sentence may or may not be refused, depending on the aims of
        the particular application.
      \item Create the correct substructures to correspond to the way
        the meanings of the individual words combine with each other.
    \end{itemize}
    
  \item[Pragmatic Analysis] The structure representing what was
    said is analyzed to determine which action the system should take.
    I.e., a statement like {\em ``It is cold in here''\/} could be
    interpreted as a simple declarative statement to the fact that it
    indeed {\em is\/} cold, but it could also be (and should be,
    according to the cooperativity principles) interpreted as a
    request to shut the window, given that there is an open window in
    the room where the statement is uttered.
\end{description}

The sequence in which these steps are performed is not necessarily the
same from one system to another, nor are the boundaries between them.
Also, if it is found that a given system would work according to its
specification without going through all the steps when analyzing its
input, it is no requirement that it do. Still, every full-fledged
{\nlp}-system has to have its input thoroughly analyzed to be certain
to behave correctly in all situations, and is thus subject to carrying
its input through all the steps.

\subsection{Cooperative Responses}
\label{coop}

In human to human conversation, a lot of knowledge is passed on
without being explicitly stated. In spoken conversation, the amount of
information conveyed by stress, intonation, facial expressions and
body language in general often by far exceeds the amount contained in
the few words actually spoken. Since we are only concerned here with
written language, all this information is lost to us, but even what
appears to be a short declarative statement in written language, can
convey a large body of hidden and implicit information, information
which has to be inferred in order for the system to be able to act
cooperatively and to the user's satisfaction.  From~\cite{gal}:

\begin{quote}
  {\em {\em [Human conversational exchanges]} are characteristically
  cooperative efforts and each participant recognizes in them a common
  purpose or a set of purposes or, at least, a mutually accepted
  direction. Unless exceptional circumstances prevail, each
  participant in a conversation is responsible for detecting and
  correcting misconceptions that might otherwise occur, and expects
  the other participants to do the same.}
\end{quote}

These implicit behaviour rules have been formalized by Grice, who
in~\cite{grice} suggests several maxims constituting what he calls the
{\em Cooperation Principle\/}. The maxims of quality, quantity,
relation and manner are the ones of concern here:

\paragraph{The maxim of quality} 
This maxim requires that one try to make a truthful contribution:

\begin{itemize}
  \item Do not say what you believe to be false.
  \item Do not say that for which you lack adequate evidence.
\end{itemize}

\paragraph{The maxim of quantity} 
This maxim relates to the amount of information contained in a
contribution:

\begin{itemize}
  \item Make your contribution as informative as is required.
  \item Do not make your contribution more informative than is
    required.
\end{itemize}

\paragraph{The maxim of relation} 
This maxim requires that a contribution be relevant to the
conversation.

\paragraph{The maxim of manner} 
This maxim relates to questions of presentation:

\begin{itemize}
  \item Avoid obscurity of expression.
  \item Avoid ambiguity.
  \item Be brief.
\end{itemize}

The other maxims suggested by Grice address aesthetic, social and
moral questions, but in general the four maxims above are sufficient
to guide a meaningful discourse. The Cooperation Principle was
intended merely as a description of how humans behave in a
conversation, but it has come to be widely accepted also as a model on
which to base intelligent interfaces to {\nlp}-systems.

By {\em cooperative responses\/} we mean responses that correspond as
closely as possible to those we would expect to get from a person
trying to understand and reply to the queries of another person. In
connection with {\nlp}-systems, the aim is, in other words, to make
the system behave as much in accordance with the Cooperation Principle
as possible.
