%%%%%%%%%%%%%%%%%%%%%%%%%%%%%% -*- Mode: Latex -*- %%%%%%%%%%%%%%%%%%%%%%%%%%%%
%%% im_aprch.tex --- 
%%% Author          : blehr
%%% Created On      : Thu Mar 25 04:25:48 1993
%%% Last Modified By: blehr
%%% Last Modified On: Thu Mar 25 05:36:06 1993
%%% RCS revision    : $Revision$ $Locker$
%%% Status          : In writing....
%%%%%%%%%%%%%%%%%%%%%%%%%%%%%%%%%%%%%%%%%%%%%%%%%%%%%%%%%%%%%%%%%%%%%%%%%%%%%%

\section{Chosen Approach}
\label{image:approach}

In this section it is assumed that the images supplied by the camera
are relatively free from noise.  In other words, they will not be
subjected to a noise removing scheme prior to being input to the
algorithm.  Still, it will be aimed at making the algorithm as
insensitive as possible to the noise present in the images, mainly by
using operators on the image that introduce some kind of smoothing in
addition to performing their designated tasks.  Should it, however, at
a later stage of the project, turn out that noise removing
preprocessing of the images is necessary, it follows from the previous
section that median filtering is the computationally cheapest and also
most promising of the presented schemes.

\subsection{Rejected Approaches}
\label{image:approach:reject}

In the previous sections, a number of techniques were discarded as
unappliccable to the given problem.  Most of the techniques were
discarded due to being to computationally expensive.  This applies to
frequency domain methods in general, being at least
$O(N^{2}\log_{2}N)$ in time; to template matching, being $O(N^{4})$ in
time; and lastly, to the Hough transform, being $O(N^{k})$ in time,
$k\geq 3$.  The other incentive to rejection was failure to serve the
putpose.  The techniques thus rejected were simple thresholding, both
as a means of segmenting out the pupil, as a criterion for inclusion
in region growing, and as a means of detecting the pupil-iris contour;
and the Laplacian operator, responding to weakly to the edges of
interest.

\subsection{Promising Aspects}
\label{image:approach:aspect}

Many of the presented techniques displayed promising aspects which in
some way may be combined to yield a satisfactory foundation for the
algorithm to be developed.  The most promising of these aspects are
listed below:

\begin{itemize}
\item Thresholding was, as elaborated upon in
  Section~\ref{image:eval}, discarded, both as as a means of
  segmenting out the pupil, as a criterion for inclusion in region
  growing, and as a means of detecting the pupil-iris contour.
  However, as was pointed out in Section~\ref{image:eval:segment}, it
  can be applied to implement a {\em pixel filter\/} which can be used
  to unambigously classify a given pixel as either a pixel point or a
  non-pixel point.
\item The region growing technique, supplied with a seed point from a
  search algorithm employing some sort of pixel filter as stop
  criterion, would be able to detect the extent of the pupil, and thus
  also its centre.
\item The most promising inclusion criterion for the region growing
  approach depicted in Section~\ref{image:eval:segment} was deemed to
  be one accepting for inclusion only those pixels {\em not\/}
  recognized as edge points.
\item By letting an edge detection procedure start from a point
  positively classified as a pupil point and then proceed in
  all directions until, in each direction, an edge is detected, the
  extent of the pupil, and thus its centre can be computed.
\item Of the presented edge detecting schemes, gradiant operators,
  examplified by the Sobel operators, turned out to give the strongest
  response to the relatively weak pupil contour in the test image in
  Fig.~\ref{fig:compare}(a).
\end{itemize}

\subsection{Satisfying the Requirement of Speed}
\label{image:approach:speed}

As the minimum error in an eye-tracking system based on digital image
analysis is proportional to the pixel size in the digitized image, it
is desirable to employ images with as high a resolution as possible.
As pointed out in Section~\ref{intro:problem}, a resolution of
$1024\times 1024$ pixels with 256 gray levels would have been
preferred.  For reasons elaborated upon there, however, a lower
resolution of $512\times 512\times 256$ was decided upon.  Still, the
total number of pixels that the algorithm has to cope with is fairly
high: 262.144.  In contrast to this number is the maximum time of 20
ms for the algorithm to detect and locate the pupil in a given image.
From these more or less incompatible requirements it is evident that
every effort has to be put into making the algorithm as fast as
possible, without compromising the requirement of accuracy.

To solve this problem it would be beneficial if it were possible to
detect and locate the pupil without having to consider every single
pixel in the image.  The smaller the fraction of the total number of
pixels that have to be taken into consideration, the faster the
algorithm.  

\subsubsection{Region Growing?}

From the above, it is clear that one approach to reducing the number
of pixels having to be considered is to use region growing and only
include non-edge points during the growing process.  This, however,
would have to be implemented recursively, which implementation by
nature is rather expensive.  In addition, to keep track of the pixels
that have been visited and those that have not, it would be necessary
to maintain a large data structure of size similar to that of the
image.  Each time a pixel is to be tested for inclusion, this
structure has to be consulted as to whether it already {\em has\/}
been tested, which implies a large number of unnecessary comparisons.
And, perhaps most importantly, this approach requires that {\em all\/}
pixels in the pupil be tested as to whether they are to be included in
the growing region or not.  The number of pixels in the pupil is not
large compared to the total number of pixels in the image, but an even
better solution would be one with which only a subset of the pixels in
the pupil has to be considered.

\subsection{Pixel Oriented Edge Detection}
\label{image:approach:chosen}

An algorithm of which the above region growing procedure is but an
implementation is the edge detection based algorithm referred to in
the list in Section~\ref{image:approach:aspect}, and described in
Section~\ref{image:eval:edge}.  The basis of this algorithm is that
the pupil-iris edge can be unambigously recognized as such by letting
it be the first edge that an edge detecting algorithm encounters.  The
algorithm operates by proceeding, from a point positively classified
as a pupil point (the {\em origin of operation\/}), in different
directions, applying some edge detecting operator successively on
every pixel encountered in each direction, and, if an edge point is
recognized, registering the coordinates of this point.  Since the
pupil-iris edge can be assumed to be the first edge encountered when
moving in any direction from a point within the pupil, the points thus
detected represent the extent of the pupil.  To compute the position
of the centre of the pupil, it is necessary to know the coordinates of
the oppositely positioned edge points of two perpendicular lines
intercepting at the origin of operation, as shown in
Fig.~\ref{fig:compute}. 

\inserteps{compute}{\label{fig:compute}Illustration of how the centre
  of the pupil can be computed.}
