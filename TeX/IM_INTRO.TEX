%%%%%%%%%%%%%%%%%%%%%%%%%%%%%% -*- Mode: Latex -*- %%%%%%%%%%%%%%%%%%%%%%%%%%%%
%%% im_intro.tex --- 
%%% Author          : blehr
%%% Created On      : Thu Mar 11 05:37:29 1993
%%% Last Modified By: blehr
%%% Last Modified On: Thu Mar 18 03:50:24 1993
%%% RCS revision    : $Revision$ $Locker$
%%% Status          : In writing....
%%%%%%%%%%%%%%%%%%%%%%%%%%%%%%%%%%%%%%%%%%%%%%%%%%%%%%%%%%%%%%%%%%%%%%%%%%%%%%

\section{Introduction}
\label{image:intro}

The general field of automated image processing may be divided into
four main categories~(\cite{digim}):

\begin{description}
\item[Image digitization:] Converting continous (real) images into
  discrete (digitized) form.
\item[Image enhancement and restoration:] Improving degraded (blurred,
  noisy) images.
\item[Image encoding:] Techniques for representing an image, or the
  information contained in the image, with fewer bits than was needed
  for the ``raw'' digitized image.
\item[Image segmentation, representation, and description:] Image
  segmentation is the process of subdividing an image into its
  constituents regions or objects.
\end{description}

{\em Digital\/} image processing deals with the latter three of these
categories, in that it already digitized images are required as input.
In this chapter, an introduction is given to the fields of {\em noise
  reduction\/}, and {\em image segmentation\/}, with an inclination
towards the problem as defined in Section~\ref{back:problem}.

The structure of the chapter is as follows:
Sections~\ref{image:spatial} and~\ref{image:frequency} give an
introduction to the two main categories of image processing
techniques: {\em spatial domain\/} techniques and {\em frequency
  domain\/} techniques; Section~\ref{image:noise} addresses the
problem of noise reduction; in Section~\ref{image:segment}, some
approaches to image segmentation are discussed, and in
Section~\ref{image:edge}, some edge detection approaches are
presented.  In Section~\ref{image:concl}, a conclusion is drawn as to
which technique to choose for the problem at hand.
