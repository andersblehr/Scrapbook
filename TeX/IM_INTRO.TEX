%%%%%%%%%%%%%%%%%%%%%%%%%%%%%% -*- Mode: Latex -*- %%%%%%%%%%%%%%%%%%%%%%%%%%%%
%%% im_intro.tex --- 
%%% Author          : blehr
%%% Created On      : Thu Mar 11 05:37:29 1993
%%% Last Modified By: blehr
%%% Last Modified On: Sun Mar 21 05:57:25 1993
%%% RCS revision    : $Revision$ $Locker$
%%% Status          : In writing....
%%%%%%%%%%%%%%%%%%%%%%%%%%%%%%%%%%%%%%%%%%%%%%%%%%%%%%%%%%%%%%%%%%%%%%%%%%%%%%

\section{Introduction}
\label{image:intro}

\subsection{Intent}
\label{image:intro:intent}

The intent of this chapter is to give a relatively broad presentation
of different techniques that may be applied to solve the problem at
hand, as defined in Section~\ref{intro:problem}.  The presentation is
intended to indicate some possible approaches to the problem, and thus
to form a theoretical foundation on which to base the development of a
solution as complete and satisfactory as possible.

Because of the requirement of speed, the techniques presented have
been selected on the basis of being relatively cheap computationally.
Most of them are also well-established, in that they are thoroughly
described in the literature.  Many reliable, but, due to relatively
complex mathematics, computational expensive methods, such as the
Canny/Deriche approach to edge
detection~(\cite{canny},~\cite{deriche}), are not discussed.

In order to maintain an overall view, the presentation is kept on a
relatively fundamental level.  Still, it has been aimed at making it
thorough enough to suffice as a basis on which to draw a conclusion
as to a suitable approach to the actual problem.

\subsection{Categories in Image Processing}
\label{image:intro:categories}

The general field of automated image processing may be divided into
four main categories.  These categories are~(\cite{digim}):

\begin{description}
\item[Image digitization:] In order to be in a form suitable for
  processing in a computer, an image has to be discretized (digitized)
  both spatially and in amplitude.  {\em Image digitation\/} is the
  process of converting continuous (real) images into digitized form.
\item[Image enhancement and restoration:] Often an image is unsuited
  for machine and/or human perception, due to its being blurred,
  noisy, or in some other way degraded.  The field of improving
  degraded images for machine or human perception is generally
  referred to as {\em image enhancement and restoration\/}.
\item[Image encoding:] Techniques for representing an image, or the
  information contained in the image, with fewer bits than the ``raw''
  digitized image needs are categorized as {\em image encoding\/}
  techniques.
\item[Image segmentation, representation, and description:] Normally,
  an image consists of contiguous regions, representing the objects
  that make up the image.  Often, one is interested in segmenting out
  one or several objects in an image.  This can be done by applying
  techniques classified as {\em image segmentation\/} techniques.  In
  addition, one may be interested in representing the constituent
  parts of an image in forms suitable for further computer processing
  and to describe them in terms of parts and properties.  This is
  referred to as {\em image representation\/} and {\em description\/},
  respectively.
\end{description}

The field of {\em digital\/} image processing is concerned with the
latter three of these categories, in that already digitized images are
required as input to a computer.  In this chapter, the focus is on the
fields of {\em noise reduction\/}, and {\em image segmentation\/}, as
described in the next section.

\subsection{Structure of the Chapter}
\label{image:intro:structure}

The focus in this chapter is on spatial domain techniques, these being
computationally cheaper as compared to frequency domain methods.  A
theoretical introduction to the spatial domain and terms pertaining to
it is given in Section~\ref{image:spatial}.  Still, due to its wide
scope of application in image processing problems, the theoretical
basis of frequency domain techniques, the Fourier transform, is given
a relatively extensive presentation in Section~\ref{image:frequency},
along with a basic description of its use in image processing.

A usual problem when processing digital images, is that they often
have been subjected to some kind of noise.  In
Section~\ref{image:noise}, some approaches to the problem of reducing
the amount of noise present in a given image are presented.  

Since the problem at hand is one of recognizing and locating a given
object in an image (the pupil), the problem of decomposing an image
into constituent regions representing different objects is elaborated
upon in Section~\ref{image:segment}.  

A branch of the field of image segmentation is the field that is
concerned with locating discontinuties in an image, such as lines and
edges.  Since an obvious approach to locating the pupil in an image is
to try and locate the transistion in brightness between the pupil and
the iris, the field of edge detection is given separate treatment in
Section~\ref{image:edge}.  

Lastly, in Section~\ref{image:concl}, the applicability of the
techniques presented in the foregoing sections with respect to the
given problem is discussed, and a conclusion is drawn as to which
approach to take in developing an algorithm to solve the problem.  The
proposed algorithm itself is presented in Chapter~\ref{algo}.

\subsubsection{References}

Most of the material in this chapter is from~\cite{digim}.  Another
main source is~\cite{digpic}.  The material in the section on template
matching (Section~\ref{image:segment:template}) is mainly
from~\cite{digpat} and~\cite{template}, the former having supplied
additional material to other sections of the chapter as well.  A last
source for additional material has been~\cite{digbild}.
