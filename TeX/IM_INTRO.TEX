%%%%%%%%%%%%%%%%%%%%%%%%%%%%%% -*- Mode: Latex -*- %%%%%%%%%%%%%%%%%%%%%%%%%%%%
%%% im_intro.tex --- 
%%% Author          : blehr
%%% Created On      : Thu Mar 11 05:37:29 1993
%%% Last Modified By: blehr
%%% Last Modified On: Sun Mar 14 01:40:45 1993
%%% RCS revision    : $Revision$ $Locker$
%%% Status          : In writing....
%%%%%%%%%%%%%%%%%%%%%%%%%%%%%%%%%%%%%%%%%%%%%%%%%%%%%%%%%%%%%%%%%%%%%%%%%%%%%%

\section{Introduction}
\label{image:intro}

The general field of automated image processing may be divided into
four main categories~(\cite{digim}):

\begin{description}
\item[Image digitization:] Converting continous (real) images into
  discrete (digitized) form.
\item[Image enhancement and restoration:] Improving degraded (blurred,
  noisy) images.
\item[Image encoding:] Techniques for representing an image, or the
  information contained in the image, with fewer bits than was needed
  for the ``raw'' image.
\item[Image segmentation, representation, and description:] Image
  segmentation is the process that subdivides an image into its
  constituents regions or objects.  Image representation and
  description are important processes in the implementation of
  autonomous image processing and analysis systems.
\end{description}

{\em Digital\/} image processing deals with the latter three of these
categories, in that it requires already digitized images as input.  In
this chapter, an introduction is given to the fields of {\em image
  enhancement and restoration\/}, and {\em image segmentation\/}, with
an inclination towards the problem at hand, namely recognition and
location of the pupil within a digital image of the eye.  For a
thorough problem definition, see Section~\ref{algo:problem}.

The structure of the chapter is as follows:
Sections~\ref{image:spatial} and~\ref{image:frequency} give an
introduction to the two main categories of image processing methods:
spatial domain methods and frequency domain methods;
Section~\ref{image:noise} addresses the problem of noise reduction;
Section~\ref{image:enhance} discusses image enhancement techniques; in
Section~\ref{image:pattern}, the problem of pattern recognition is
presented, and in Section~\ref{image:edge}, some edge detection
approaches are discussed.  In Section~\ref{image:concl}, a conclusion
is drawn as to which technique to choose for the problem at hand.
