%%%%%%%%%%%%%%%%%%%%%%%%%%%%%% -*- Mode: Latex -*- %%%%%%%%%%%%%%%%%%%%%%%%%%%%
%%% bk_eye.tex --- 
%%% Author          : blehr
%%% Created On      : Tue Mar  9 03:58:22 1993
%%% Last Modified By: blehr
%%% Last Modified On: Thu Mar 11 04:06:36 1993
%%% RCS revision    : $Revision$ $Locker$
%%% Status          : In writing....
%%%%%%%%%%%%%%%%%%%%%%%%%%%%%%%%%%%%%%%%%%%%%%%%%%%%%%%%%%%%%%%%%%%%%%%%%%%%%%

\section{The Eye}
\label{back:eye}

In this section, an introduction is given to some of the terms
relating to the eye, as well as to its structure.  Since the subjects
may be human as well as monkeys, both kinds of eyes are described.
The human eye is given the most thorough presentation in
Section~\ref{back:eye:human}, and in Section~\ref{back:eye:monkey},
properties of the monkey's eye demanding special concern are
described.  Section~\ref{back:eye:visual} describe som useful visual
properties of the eye, from an image processing point of view.

\subsection{The Structure of the Human Eye}
\label{back:eye:human}

In Fig.~\ref{fig:back:eye:human}, a horizontal cross section of the
human eye is shown (from~\cite{digim}).  The shape of the eye is
nearly spherical, and its average diameter is approximately 20 mm.
The eye ball ({\em optical globe\/}) is built up from three enclosing
membranes: the {\em cornea\/} and {\em sclera\/}, which together form
the outer cover of the eye, the middle {\em choroid\/}, and the inner
{\em retina\/}.  

The cornea covers the {\em iris\/} and the {\em pupil\/}, and consists
of a tough, transparent tissue, allowing light to pass through.  The
rest of the eye is covered by the white sclera, which is continuous
with the cornea.

The choroid lies directly below the sclera.  This membrane contains a
network of blood vessels that serves as a major source of nutrition to
the eye.  Its coat is heavily pigmented to reduce the amount of
extraneous light entering the eye.  The part of the choroid belonging
to the foremost portion of the eye (its {\em anterior extreme\/}) is
divided into the iris and the {\em ciliary body\/}.  The {\em ciliary
  muscles\/} allows the iris to contract or expand to control the
amount of light entering the eye through its central opening, the
pupil.  The pupil is variable in diameter from approximately 2 mm up
to 8 mm.  The front of the iris contains the visible pigment of the
eye, whereas the back contains a black pigment.

\insertepswidth{eye}{\label{fig:back:eye:human}A simplified diagram of
  a cross section of the human eye}{0.6}

The {\em lens\/}, which lies behind the iris, is made up of concentric
layers of fibrous cells and is suspended by {\em ciliary fibres\/},
attaching it to the ciliary body.  It contains between 60 and 70 per
cent water, about 6 per cent fat, and more protein than any other
tissue in the eye.  The shape of the lens is controlled by the tension
in the ciliary fibres.  To focus on distant objects, the controlling
muscles cause the lens to be relatively flattened, whereas they cause
the lens to thicken to focus on objects near the eye.  Both
ultraviolet and infrared light are absorbed appreciably by the lens,
and, in large amounts, may cause damage to the eye.

The innermost membrane of the eye is the retina, which covers the
inside of the backmost portion of the eye.  The surface of the retina
is covered by light receptors, on which an image of the outside world
(an object, a scene, etc.) is focused by the lens.  There are two
classes of light receptors: {\em cones\/} and {\em rods\/}.  The
cones, which number between 6 and 7 million and are highly sensitive
to colour, are primariliy located in the central portion of the
retina, the {\em fovea\/}.  The eye is rotated until the image of the
object of interest falls on the fovea.  Each cone is connected to its
own nerve end, allowing very fine visual resolution.  Cone vision is
known as {\em photopic\/} or bright-light vision.

The rods, which are distributed over the entire retinal surface,
number between 75 and 150 million, and serve to give a general,
overall picture of the field of view.  The larger area of distribution
and the fact that several rods are connected to a single nerve end
reduce the amount of detail discernable by them.  They are insensitive
to colour, but much more sensitive to low levels of illumination than
the cones.  Rod vision is known as {\em scotopic\/} or dim-light
vision.

\subsection{The Monkey's Eye}
\label{back:eye:monkey}

\subsection{Visual Properties of the Eye}
\label{back:eye:visual}

The most obvious visual property of the eye, both in man and monkey,
is the contrast between the relatively bright iris and the dark pupil.
