%%%%%%%%%%%%%%%%%%%%%%%%%%%%%% -*- Mode: Latex -*- %%%%%%%%%%%%%%%%%%%%%%%%%%%%
%%% im_edge.tex --- 
%%% Author          : blehr
%%% Created On      : Thu Mar 11 05:40:55 1993
%%% Last Modified By: blehr
%%% Last Modified On: Sun Apr 11 13:29:44 1993
%%% RCS revision    : $Revision$ $Locker$
%%% Status          : In writing....
%%%%%%%%%%%%%%%%%%%%%%%%%%%%%%%%%%%%%%%%%%%%%%%%%%%%%%%%%%%%%%%%%%%%%%%%%%%%%%

\section{Edge Detection}
\label{image:edge}

As mentioned in the previous section, {\em discontinuty detection\/}
is a branch of image segmentation.  To the area of discontinuty
detection belong the fields of {\em point detection\/}, {\em line
  detection\/}, and {\em edge detection\/}.  In this section, the
focus is on edge detection, as it constitutes by far the most common
approach for detecting discontinuties in gray level.  An {\em edge\/}
is defined as the boundary between two regions characterized by having
relatively distinct gray levels.

A difficulty with edge detection as an approach to image segmentation
is that the detected edges often have gaps in them, due to places
where transistions between regions are not abrupt enough that an edge
be detected.  In addition, if the image is noisy, edges may be
detected at points which are not parts of region boundaries, as
discussed below.  Thus the detected edges will not necessarily form
closed, connected curves which surround closed, connected regions.

\subsection{Thresholding}
\label{image:edge:threshold}

The tresholding technique introduced in
Section~\ref{image:segment:threshold} can, with slight modifications,
be applied to detect edges in an image, as described in~\cite{digim}.
The idea is to choose a threshold $T$, based on some criteria,
dividing the gray scale of the image into two distinct bands.  Then
the image is scanned in the $x$ and $y$ directions separately.  Each
time a change in gray level from one band to the other occurs, this
indicates the presence of a boundary point.  The procedure performs in
two passes as follows:

\paragraph{Pass 1:} For each row in $f(x,y)$, create a corresponding
row in an intermediate image $g_{1}(x,y)$, using the following
relation: 
\begin{equation}
  g_{1}(x,y)=\left\{
    \begin{array}{ll}
      L_{E} & \mbox{if the levels of $f(x,y)$ and $f(x,y-1)$ are in} \\
            & \mbox{different bands of the gray scale,} \\
      L_{B} & \mbox{otherwise,}
    \end{array}\right.
\end{equation}
where $L_{E}$ and $L_{B}$ are specified boundary and background
levels, respectively.

\paragraph{Pass 2:} For each column in $f(x,y)$, create a
corresponding column in an intermediate image $g_{2}(x,y)$ using the
following relation:
\begin{equation}
  g_{2}(x,y)=\left\{
    \begin{array}{ll}
      L_{E} & \mbox{if the levels of $f(x,y)$ and $f(x-1,y)$ are in} \\
            & \mbox{different bands of the gray scale,} \\
      L_{B} & \mbox{otherwise.}
    \end{array}\right.
\end{equation}
\vspace*{0.1cm}

\noindent The desired image, consisting of the boundary points of the
object of interest on a uniform background, is then given by the
relation:
\begin{equation}
  g(x,y)=\left\{
    \begin{array}{ll}
      L_{E} & \mbox{if $(g_{1}(x,y)=L_{E})\vee (g_{2}(x,y)=L_{E})$,} \\
      L_{B} & \mbox{otherwise.}
    \end{array}\right.
\end{equation}
As is the case with thresholding in general, care has to be taken when
choosing the threshold value $T$.  

As is evident from its formulation, this procedure assumes that the
edges in the given image are characterized by monotonously increasing
or decreasing functions.  This is seldom the case in real
applications.  If the image has been subjected to additive noise, so
that in regions with gray levels in the proximity of $T$ pixel values
are randomly distributed on both sides of $T$, which is normally the
case, this method will detect an edge point each time a transition
from one band to another occurs, thus making it difficult to
distinguish between an actual edge, and edges detected due to noise.
In the worst case an edge will be depicted as a ``bulb'' of layered
edge fragments.  If the edges of interest are relatively sharp,
however, that is, they approximate a step edge rather than a ramp
edge, this problem is less dominant.

\subsection{Gradient Operators}
\label{image:edge:gradient}

As edge points in an image are characterized as points where the gray
level of the image changes rapidly, an obvious approach to edge
detection is to apply some kind of differentiation on the image.  The
most common approach to differentiation in image processing
applications is the {\em gradient\/}.  The gradient of a
two-dimensional function $f(x,y)$ is defined as the vector
\begin{equation}
  \vec{G}[f(x,y)]=\left[
    \begin{array}{c}
      \underline{\partial f} \\
      \partial x \\ \\
      \underline{\partial f} \\
      \partial y
    \end{array}\right]=\left[
    \begin{array}{c}
      G_{x} \\ \\
      G_{y}
    \end{array}\right]\mbox{.}
\end{equation}
$\vec{G}[f(x,y)]$ always points in the direction of the maximum rate
of increase of $f$ at $(x,y)$, and the rate of increase per unit
distance in this direction, denoted $G[f(x,y)]$, is given by
\begin{equation}
\label{eq:gradient:sqrt}
  G[f(x,y)]=|\vec{G}|=\sqrt{G_{x}^{2}+G_{y}^{2}}\mbox{.}
\end{equation}
The magnitude function $G[f(x,y)]$ is generally referred to as the
{\em gradient\/} of $f$, to avoid constantly having to refer to it as
``the magnitude of the gradient''.  To reduce computational costs, the
gradient is commonly approximated using absolute values:
\begin{equation}
\label{eq:gradient:abs}
  G[f(x,y)]\approx|G_{x}|+|G_{y}|\mbox{.}
\end{equation}
For digital images, $G_{x}$ and $G_{y}$ are approximated using
differences.  One approach is to use first-order differences in a
$2\times 2$ neighbourhood.  A typical approximation in this respect is
\begin{equation}
  G_{x}\simeq f(x,y)-f(x+1,y)
\end{equation}
\begin{equation}
  G_{y}\simeq f(x,y)-f(x,y+1)\mbox{,}
\end{equation}
yielding the following expression for $G[f(x,y)]$:
\begin{equation}
\label{eq:gradient:simple}
  G[f(x,y)]\simeq|f(x,y)-f(x+1,y)|+|f(x,y)-f(x,y+1)|\mbox{.}
\end{equation}
This approximation uses first-order differences in the horizontal and
vertical directions about $(x,y)$ to compute $G[f(x,y)]$.  Another
approximation, commonly referred to as the {\em Roberts gradient\/},
uses differences symmetrical about the imaginary interpolation point
$(x+\frac{1}{2}, y+\frac{1}{2})$:
\begin{equation}
\label{eq:gradient:roberts}
  G[f(x,y)]\simeq|f(x,y)-f(x+1,y+1)|+|f(x+1,y)-f(x,y+1)|\mbox{.}
\end{equation}
The approximations in Eqs.~(\ref{eq:gradient:simple})
and~(\ref{eq:gradient:roberts}) use the differences between two pairs
of adjacent pixels to compute the gradient.  Thus, for about the same
reasons given in Section~\ref{image:edge:threshold}, these
approximations tend to be sensitive to noise.

\subsubsection{The Sobel Operators}

Using a $3\times 3$ neighbourhood to compute the gradient has the
advantage of increased smoothing over $2\times 2$ operators.  Consider
the following definitions of $G_{x}$ and $G_{y}$:
\begin{equation}
\label{eq:sobel:x}
  \begin{array}{lll}
    G_{x} & = & [f(x-1,y+1)+2f(x,y+1)+f(x+1,y+1)]\ - \\
          &   & [f(x-1,y-1)+2f(x,y-1)+f(x+1,y-1)]
  \end{array}
\end{equation}
\begin{equation}
\label{eq:sobel:y}
  \begin{array}{lll}
    G_{y} & = & [f(x+1,y-1)+2f(x+1,y)+f(x+1,y+1)]\ - \\
          &   & [f(x-1,y-1)+2f(x-1,y)+f(x-1,y+1)]\mbox{.}
  \end{array}
\end{equation}

\insertepswidth{sobelop}{\label{fig:sobelop}Spatial masks used to
  compute $G_{x}$ (a) and $G_{y}$ (b), as defined by
  Eqs.~(\protect\ref{eq:sobel:x})
  and~(\protect\ref{eq:sobel:y}).}{0.4}

The spatial masks shown in Figs.~\ref{fig:sobelop}(a) and (b)
implement $G_{x}$ and $G_{y}$, respectively.  These two masks are
commonly referred to as the {\em Sobel operators\/}.  The responses of
these two operators at any point $(x,y)$ in an image are combined
using Eq.~(\ref{eq:gradient:sqrt}) or Eq.~(\ref{eq:gradient:abs}) to
obtain the gradient at that point.  Convolving these masks with an
image yields the gradient at all points $(x,y)$ in the image, the
result often being referred to as a {\em gradient image\/}.

\subsubsection{The Laplacian}

Higher order derivative operators can also be used to detect edges in
an image.  The {\em Laplacian\/} operator $L[f(x,y)]$ is a
second-order derivative operator which is direction insensitive, as
opposed to the Sobel-operators.  In the continuous case, the Laplacian
is defined by the equation
\begin{equation}
  L[f(x,y)]=\frac{\partial^{2}f}{\partial x^{2}}+
  \frac{\partial^{2}f}{\partial y^{2}}\mbox{.}
\end{equation}
In the digital case, it is approximated by
\begin{equation}
\label{eq:laplacian}
  L[f(x,y)]\simeq f(x,y-1)+f(x-1,y)+f(x+1,y)+f(x,y+1)-4f(x,y)\mbox{.}
\end{equation}
A mask implementing Eq.~(\ref{eq:laplacian}) is shown in
Fig.~\ref{fig:laplacian}.  Note that Eq.~(\ref{eq:laplacian}) implies
that $L[f(x,y)]$ takes on both positive and negative values.  If only
positive values are desired, the absolute value can be used.  Note
also that the Laplacian is zero in constant areas and on the ramp
section of edges, as expected of a second-order derivative operator.

As a consequence of its being a second-order derivative operator,
however, the Laplacian tends to be unacceptably sensitive to noise.
It also does not respond as strongly to edges as do the other
operators described thus far.  On the other hand, it responds far more
strongly to single points, lines and line ends than these do.  It can
also be applied to detect whether a given pixel is on the dark or
bright side of an edge.

\insertepswidth{laplace}{\label{fig:laplacian}Spatial mask used to
  compute the Laplacian as defined by
  Eq.~(\protect\ref{eq:laplacian})}{0.15}

\subsection{Highpass Filtering}
\label{image:edge:highpass}

Convolving an image with any of the gradient operators described above
is equivalent to highpass filtering in the frequency domain.  Thus it
ought to be possible to detect edges in an image by highpass filtering
the Fourier transform of the image directly.  Highpass filtering the
Fourier transform of a given image indeed has effects similar to those
of applying the above operators.  The usefulness of direct highpass
filtering as an edge detection technique is, however, questionable, in
that the responses returned are relatively weak as compared to the
results obtained using the above techniques.  This is clearly
demonstrated in Figs.~9 and~10 on pp.~281--283 in~\cite{digpic}.

\subsection{The Hough Transform}
\label{image:edge:hough}

The last approach to edge detection discussed here, is the so-called
{\em Hough transform\/}.  The discussion below is directed towards
applying the Hough transform to detect straight lines and edges in an
image, but the principle can be extended to detecting any curve or
edge that can be described by the equation $g(\vec{x},\vec{c})=0$,
where $\vec{x}$ is a vector of coordinates and $\vec{c}$ is a vector
of coefficients.

Consider a point $(x_{i},y_{i})$ in an image.  An infinite number of
lines satisfying the equation $y_{i}=ax_{i}+b$ for varying values of
$a$ and $b$ pass through this point.  However, the equation
$b=-x_{i}a+y_{i}$, which is equivalent to the previous one, describes
a single line in the {\it ab\/} plane (often referred to as {\em
  parameter space\/}) for a fixed pair $(x_{i},y_{i})$.  Analogously,
a second image point $(x_{j},y_{j})$ also has a line in parameter
space associated with it.  This line intercepts the line associated
with $(x_{i},y_{i})$ at a point $(a',b')$, where $a'$ is the slope and
$b'$ the intercept of the line in the {\it xy\/} plane (the image)
containing both $(x_{i},y_{i})$ and $(x_{j},y_{j})$.  All points on
this line will have lines in parameter space which intercept at
$(a',b')$.

By associating with each point $(a_{i},b_{i})$ in parameter space an
{\em accumulator cell\/}, it is possible to keep track of how many
points in the {\it xy\/} plane lie on a given line.  This is done by
letting, for every point $(x_{k},y_{k})$ in the image, $a$ vary along
the entire $a$ axis in parameter space, and for each $a_{l}$ solve for
$b_{l}$ using the relation $b_{l}=-x_{k}a_{l}+y_{k}$.  The value of
the accumulator cell associated with $(a_{l},b_{l})$ is then
incremented by one, and at the end of the procedure, a value of $M$ in
the accumulator cell associated with $(a_{i},b_{j})$ corresponds to
$M$ points in the image lying on the line $y=a_{i}x+b_{j}$.

A problem with describing a line with the equation $y=ax+b$ is that
both slope and intercept approach infinity as the line approaches a
vertical position.  This problem is circumvented by using the {\em
  normal\/} description $x\cos\theta+y\sin\theta=\rho$ of the line
instead.  The only difference in using this representation when
constructing the set of accumulator cells is that the intercepts in
parameter space ($\rho\theta$ plane) are between sinusoidal curves
instead of straight lines.

As mentioned above, these principles can be applied to detect any
curve or edge which can be described by the relation
$g(\vec{x},\vec{c})=0$.  The equation
$(x-c_{1})^{2}+(y-c_{2})^{2}=c_{3}^{2}$, for example, describes a
circle and can thus be used analogously to the above procedure to
detect circles in an image.  One possible application would be to
detect the (circular) pupil in an image of the eye (cf.\
Section~\ref{eval:eval:edge}).

If the number of points in the image equals $N^{2}$ (i.e., the image
is of size $N\times N$) and the $a$ axis is divided into $K$ equal
increments, the number of computations needed for the above procedure
is $KN^{2}$, since for each image point $(x_{i},y_{i})$, $a$ is varied
from $a_{1}$ to $a_{K}$ to obtain the corresponding values for $b$.
Thus the Hough transform performs in $O(N^{2})$ time, assuming that
$K\ll N$.
