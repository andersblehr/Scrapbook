%%%%%%%%%%%%%%%%%%%%%%%%%%%%%% -*- Mode: Latex -*- %%%%%%%%%%%%%%%%%%%%%%%%%%%%
%%% im_edge.tex --- 
%%% Author          : blehr
%%% Created On      : Thu Mar 11 05:40:55 1993
%%% Last Modified By: blehr
%%% Last Modified On: Thu Mar 18 08:41:32 1993
%%% RCS revision    : $Revision$ $Locker$
%%% Status          : In writing....
%%%%%%%%%%%%%%%%%%%%%%%%%%%%%%%%%%%%%%%%%%%%%%%%%%%%%%%%%%%%%%%%%%%%%%%%%%%%%%

\section{Edge Detection}
\label{image:edge}

As mentioned in the previous section, {\em discontinuty detection\/}
is a branch of image segmentation.  To the area of discontinuty
detection belong the fields of {\em point detection\/}, {\em line
  detection\/}, and {\em edge detection\/}.  In this section, the
focus is on edge detection, as it constitutes by far the most common
approach for detecting discontinuties in gray level.  An {\em edge\/}
is defined as the boundary between two regions characterized by having
relatively distinct gray levels.

\subsection{Thresholding}
\label{image:edge:threshold}

The tresholding technique introduced in
Section~\ref{image:segment:threshold} can, with slight modifications,
be applied to detect edges in an image, as described in~\cite{digim}.
The idea is to choose a threshold $T$, based on some criteria,
dividing the gray scale of the image into two distinct bands.  Then
the image is scanned in the $x$- and $y$-directions separately.  Each
time a change in gray level from one band to the other occurs, this
indicates the presence of a boundary point.  The procedure performs in
two passes as follows:

\paragraph{Pass 1:} For each row in $f(x,y)$, create a corresponding
row in an intermediate image $g_{1}(x,y)$, using the following
relation: 
\begin{equation}
  g_{1}(x,y)=\left\{
    \begin{array}{ll}
      L_{E} & \mbox{if the levels of $f(x,y)$ and $f(x,y-1)$ are in} \\
            & \mbox{different bands of the gray scale,} \\
      L_{B} & \mbox{otherwise,}
    \end{array}\right.
\end{equation}
where $L_{E}$ and $L_{B}$ are specified boundary and background
levels, respectively.

\paragraph{Pass 2:} For each column in $f(x,y)$, create a
corresponding column in an intermediate image $g_{2}(x,y)$ using the
following relation:
\begin{equation}
  g_{2}(x,y)=\left\{
    \begin{array}{ll}
      L_{E} & \mbox{if the levels of $f(x,y)$ and $f(x-1,y)$ are in} \\
            & \mbox{different bands of the gray scale,} \\
      L_{B} & \mbox{otherwise.}
    \end{array}\right.
\end{equation}
The desired image, consisting of the boundary points of the object of
interest on a uniform background, is then given by the relation:
\begin{equation}
  g(x,y)=\left\{
    \begin{array}{ll}
      L_{E} & \mbox{if $(g_{1}(x,y)=L_{E}\vee g_{2}(x,y)=L_{E})$,} \\
      L_{B} & \mbox{otherwise.}
    \end{array}\right.
\end{equation}
As is the case with thresholding in general, care has to be taken when
choosing the threshold $T$.  

If the image has been subjected to additive noise, so that in regions
with gray levels in the proximity of $T$, pixel values are randomly
distributed on both sides of $T$, this method will detect an edge
point each time a transition from one band to another occurs, thus
making it difficult to distinguish between the actual edge, and edges
detected due to noise.  This is less of a problem if the edges of
interest are relatively sharp, that is, they approximate a step edge
rather than a ramp edge.  Sharp edges are present in ``sharp'' images
where the gray levels on the two sides of an edge differ by a
relatively large amount.

\subsection{Gradient Operators}
\label{image:edge:gradient}

As edge points in an image are characterized as points where the gray
level of the image changes rapidly, an obvious approach to edge
detection is to apply some kind of {\em differentiation\/} on the
image.  The most common approach to differentiation in image
processing applications is the {\em gradient\/}.  The gradient of a
two-dimensional function $f(x,y)$ is defined as the vector
\begin{equation}
  \vec{G}[f(x,y)]=\left[
    \begin{array}{c}
      \underline{\partial f} \\
      \partial x \\ \\
      \underline{\partial f} \\
      \partial y
    \end{array}\right]\mbox{.}
\end{equation}
$\vec{G}[f(x,y)]$ always points in the direction of the maximum rate
of increase of $f(x,y)$ at $(x,y)$, and the rate of increase per unit
distance in this direction, denoted $G[f(x,y)]$, is given by
\begin{equation}
  G[f(x,y)]=|\vec{G}|=\sqrt{(\frac{\partial f}{\partial x})^{2}+
    (\frac{\partial f}{\partial y})^{2}}\mbox{.}
\end{equation}
The magnitude function $G[f(x,y)]$ is generally referred to as the
{\em gradient\/} of $f$, to avoid constantly having to refer to it as
``the magnitude of the gradient''.

\subsection{The Hough Transform}
\label{image:edge:hough}

\subsection{Highpass Filtering}
\label{image:edge:highpass}

