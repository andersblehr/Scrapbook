%%%%%%%%%%%%%%%%%%%%%%%%%%%%%% -*- Mode: Latex -*- %%%%%%%%%%%%%%%%%%%%%%%%%%%%
%%% intro.tex --- 
%%% Author          : blehr
%%% Created On      : Mon Jan 18 10:34:12 1993
%%% Last Modified By: blehr
%%% Last Modified On: Sun Mar 14 23:24:12 1993
%%% RCS revision    : $Revision$ $Locker$
%%% Status          : In writing
%%%%%%%%%%%%%%%%%%%%%%%%%%%%%%%%%%%%%%%%%%%%%%%%%%%%%%%%%%%%%%%%%%%%%%%%%%%%%%

\chapter{Introduction}
\label{intro}

\section{Synchronization and Cognition}
\label{intro:synch}

{\em Synchronization and Cognition\/} ({\em SC\/}) is a cooperation
project initiated by R.\ Eckhorn and R.\ Bauer of the Biophysics Group
and F.\ R\"{o}sler of the Physiologic Psychology Group at the
Philipps-Universit\"{a}t Marburg, Germany.

The basis of the current work is observations done by R.\ Eckhorn and
R.\ Bauer~(\cite{cat}), as well as by the group of W.\ Singer
(e.g.,~\cite{singer2} and~\cite{singer1}), that when stimulated,
retinally distributed and rhythmically correlated neural assemblies
organize in peripheral visual areas in the brain of drugged cats.  The
spatial distribution of these assemblies, but not their frequencies,
depend on the supplied visual stimulus.

The goal of the {\em SC\/} project is, through the use of
microelectrodes (monkeys) and EEG recordings (humans), to show that
coherent neurological activations ({\em synchronizations\/}) in the
visual cortex, relating to coherent visual stimuli, is a requirement
for the coherent perception of these stimuli.  In addition, the
time-space {\em synchronization dynamics\/} in peripheral and central,
as well as in intermediate cortex areas, are to be investigated as to
how they relate to the cognitive tasks being performed.

For a further description of the {\em SC\/} project,
see~\cite{antrag}.  Some of the publications by R.\ Eckhorn {\em et
  al.\/} concerning the discovery and implications of synchronizations
in the visual cortex of cats and monkeys
are~\cite{eckhorn1},~\cite{eckhorn2},~\cite{cat} and~\cite{eckhorn3}.

\section{Thesis Motivation}
\label{intro:motivation}

Since the direction of gaze and the orientation of the visual stimuli
have to correspond as exactly as possible, in order to be certain that
exactly those neural assemblies are stimulated whose activity is being
registered, it is of vital importance for the validity of the
experimental results that the exact eye position of the subject be
known at a given time.  Otherwise the danger exists of registering
neurological responses whose variations are not the result of the
assigned cognitive task, but of the subject's incorrect gaze
direction.

To gain absolute certainty about where the subject at a given time
focuses, it is necessary to have a system which continuously tracks
the movements of the subject's eyes.  For our purpose it was decided
that a system complying to the following requirements would suffice:
\begin{itemize}
\item The position-analyzing method of the system may in no way
  interfere with recordings being made from the visual cortex of a
  subject, be it man or monkey.
\item A position measuring rate of 50 Hz.  Thus every 20 ms a
  coordinate pair corresponding as closesly as possible to the actual
  gaze direction has to be made available.  When the subject is {\em
    drifting\/}, that is, it does not focus where it should, a warning
  signal has to be given.
\item The error in the position returned by the system should be less
  than or equal to $\pm 0.25^{\circ}$.
\item The measuring range of the system should be at least $\pm
  20^{\circ}$.
\item The system should be relatively easy to configure to comply with
  the specific needs of an experiment.
\item The price of the complete system may not exceed DM 12.000.
\end{itemize}

There exist several commercially available eye tracking systems, some
of which are described in Section~\ref{back:track}.  For reasons
elaborated upon in Section~\ref{back:track:summary}, it was decided to
develop an entirely new eye tracking system, based on real-time image
processing of video input.  And this is, in short, what has been the
subject of my thesis; to develop, and, as far as possible, test a
suitable {\em software platform\/} for a real-time eye position
measuring system complying to the requirements of precision and speed
mentioned above.
