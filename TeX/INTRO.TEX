%%%%%%%%%%%%%%%%%%%%%%%%%%%%%% -*- Mode: Latex -*- %%%%%%%%%%%%%%%%%%%%%%%%%%%%
%%% intro.tex --- 
%%% Author          : blehr
%%% Created On      : Mon Jan 18 10:34:12 1993
%%% Last Modified By: blehr
%%% Last Modified On: Tue Mar  2 05:18:03 1993
%%% RCS revision    : $Revision$ $Locker$
%%% Status          : In writing
%%%%%%%%%%%%%%%%%%%%%%%%%%%%%%%%%%%%%%%%%%%%%%%%%%%%%%%%%%%%%%%%%%%%%%%%%%%%%%

\chapter{Introduction}
\label{intro}

In the following a brief overview is given of the project within whose
framework my thesis took place.  In Section~\ref{intro:synch}, the
need of having an eye position measuring system as part of the
experimental setup of the work group is stated, and in
Section~\ref{intro:motivation}, the reason why it was decided to
develop an entirely new system instead of investing in an already
existing one is touched upon.

In experiments, registration of brain activity in human subjects is
done through EEG-recordings, whereas recordings from cats and monkeys
are performed with microelectrodes being inserted into the brain
tissue.

\section{Synchronization and Cognition}
\label{intro:synch}

{\em Synchronization and Cognition\/} ({\em SC\/}) is a project
initiated by R.\ Eckhorn, R.\ Bauer (Biophysics Group) and F.\
R\"{o}sler (\typeout{--> Roeslers gruppe?}Psychology Group) at the
Philipps-Universit\"{a}t Marburg, Germany.

The basis of the current work is observations done by R.\ Eckhorn and
R.\ Bauer~(\cite{cat1}), as well as by the group of W.\ Singer
(e.g.,~\cite{cat3} and~\cite{cat2}), that in drugged cats,
retinally\typeout{--> retinotop = retinal??} distributed and
rhythmically correlated neural assemblies organize in peripheral
visual areas of the brain.  The spacial distribution of these
assemblies, but not their frequencies, depend on the supplied visual
stimulation.

The goal of the {\em SC\/} project is to show that coherent
neurological activations ({\em synchronizations\/}) in the visual
cortex of monkeys and humans, relating to coherent visual stimulation,
is a requirement for the coherent perception of this stimulation.  In
addition, the time-space {\em synchronization dynamics\/} in
peripheral and central, as well as in intermediate cortex areas, are
to be investigated as to how they relate to the cognitive tasks being
performed.

Since the direction of gaze and the orientation of the visual
stimulation have to correspond as exactly as possible, in order to be
certain that exactly those neural assemblies are stimulated whose
activity is being registered, it is of vital importance for the
validity of the experimental results that the exact eye position of
the subject be known at a given time.  Otherwise the danger would
exist of registering neurological responses whose variations are not
the result of the assigned cognitive task, but of the subject's
incorrect gaze direction.  For a further description of the {\em SC\/}
project, see~\cite{antrag}.

To gain absolute certainty about where the subject at a given time
directs its gaze, it is necessary to have a system which continuously
and in real-time ($\sim$50 Hz) tracks the movements of the subject's
eyes, returning the coordinate pair corresponding to the momentary 
gaze direction, and giving a warning when the subject is {\em
  drifting\/}, i.e., does not focus where it should.

\section{Thesis Motivation}
\label{intro:motivation}

The task of my thesis has, in short, been to develop, and, as far as
possible, test a suitable software platform for a real-time eye
position measuring system complying to the requirements of precision
and speed mentioned above.

One of the main reasons for choosing to develop an entirely new eye
tracking system instead of investing in an already existing one, has
simply to do with most of the existing systems being too expensive for
the work group to be considered as alternatives.  Another reason is
that many systems, like the {\em search coil\/}~(\cite{coil}) system,
for example, require physical implantations (i.e., the search coil)
into the eye of the subject, what makes the system not suited for
human subjects.  The {\em infrared oculometer\/}~(\cite{infrared})
does not require implantations of any kind, but requires very precise
and time consuming calibration, and in addition it is realized
entirely in hardware, which makes it impossible to configure it for
specific needs.


