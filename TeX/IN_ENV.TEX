%%%%%%%%%%%%%%%%%%%%%%%%%%%%%% -*- Mode: Latex -*- %%%%%%%%%%%%%%%%%%%%%%%%%%%%
%%% in_env.tex --- 
%%% Author          : blehr
%%% Created On      : Mon Mar 22 00:59:31 1993
%%% Last Modified By: blehr
%%% Last Modified On: Sun Apr 11 12:14:49 1993
%%% RCS revision    : $Revision$ $Locker$
%%% Status          : In writing....
%%%%%%%%%%%%%%%%%%%%%%%%%%%%%%%%%%%%%%%%%%%%%%%%%%%%%%%%%%%%%%%%%%%%%%%%%%%%%%

\section{Thesis Environment}
\label{intro:env}

All computational work was carried out under MS-DOS 5.0 on an i386 IBM
compatible PC running at 40 MHz.  All time measures given throughout
the paper were achieved on this computer.  The eye-tracking algorithm
{\octopus} presented in Chapter~\ref{algo} was implemented solely in
Pascal using Borland Turbo-Pascal 6.0.  Versions were administered
using a DOS-port of the GNU versioning system RCS.  The camera I had
at my disposal during the work, and with which all test images were
obtained, is an ordinary CCD Sony Handicam, type F550E.  The circuit
board for the frame-grabber used to digitize the images supplied by
the camera was purchased from the German computer magazine {\sf
  c't\/}, whereas the components had to be purchased from various
vendors.

The editor used to write this paper was Demacs, a DOS-port of GNU
Emacs, and the paper was prepared using em\TeX, a DOS-port of \LaTeX,
Leslie Lamport's user-friendly interface to Donald E.\ Knuth's
text-formatting system \TeX\ (\cite{lamport}).  To convert the {\tt
  .dvi}-output from \TeX\ to printable PostScript, I used a DOS-port
of Radical Eye Software's {\tt dvips}, and to view the PostScript
files I used a DOS-port of Ghostscript, the GNU PostScript previewer.
Most of the figures appearing throughout the paper were made using
Corel Corporation's CorelDRAW!\ 3.0; other figures were made using my
own tools.  Lastly, I employed a row of different UNIX-like tools to
make the DOS environment as bearable as possible.  It is as the saying
goes: ``{\em MS-DOS---can't live with it, can't live without it\/}''.
