%%%%%%%%%%%%%%%%%%%%%%%%%%%%%% -*- Mode: Latex -*- %%%%%%%%%%%%%%%%%%%%%%%%%%%%
%%% cn_eval.tex --- 
%%% Author          : blehr
%%% Created On      : Fri Apr  9 13:05:21 1993
%%% Last Modified By: blehr
%%% Last Modified On: Sun Apr 11 16:04:13 1993
%%% RCS revision    : $Revision$ $Locker$
%%% Status          : In writing....
%%%%%%%%%%%%%%%%%%%%%%%%%%%%%%%%%%%%%%%%%%%%%%%%%%%%%%%%%%%%%%%%%%%%%%%%%%%%%%

\section{Algorithmic Evaluation}
\label{concl:algo}

In Section~\ref{algo:eval}, the proposed eye-tracking algorithm
{\octopus} was given a relatively thorough evaluation.  Evidently, the
accuracy of the algorithm has to be improved to comply with the
accuracy requirement given in the algorithmic problem definition if
{\octopus} is to be employed as the core of a complete eye-tracking
system satisfying the requirements given in
Section~\ref{intro:motivation}.  The technique suggested in
Section~\ref{algo:eval:improve} ought to increase the accuracy, but
since it has not been implemented and consequently not tested, nothing
can be said about whether or not an implementation incorporating this
technique would be sufficiently accurate.

As pointed out in Section~\ref{eval:approach:edge}, {\octopus} assumes
the pupil to form a circle in a given image.  However, as stated in
Section~\ref{back:eye:visual}, this is only the case when the subject
looks straight into the camera.  Still, this assumption was chosen as
a basis on which to formulate the algorithm.  The incentive for doing
this was that the primary point of interest as seen by the
Eckhorn-Bauer group is to be able to determine whether or not the
monkey is focusing on the reference point on the inducing monitor
(cf.\ Section~\ref{intro:motivation}).  Since the experimental setup
is to be designed so that the subject focusing on the reference point
corresponds to its looking straight into the camera, the assumption of
having a circular pupil is sound.  However, if {\octopus} is to be of
interest as a general-purpose eye-tracking algorithm, this problem has
to be addressed.

The most promising aspect of {\octopus} is doubtlessly its
computational efficiency.  The fact is that the current implementation
of {\octopus} running on a 40 MHz i386 PC needs approximately 2 ms to
return an estimate of the pupil's location in the supplied image.
Incorporating the suggested technique for increasing the accuracy of
the returned estimates should have minimal influence on the time
consumption.  However, if measures are to be taken to correct for the
error introduced by the pupil not forming a circle when the subject is
drifting, the computational overhead introduced may, depending on
their nature and implementation, cause the time needed for one
estimate to surpass the limit of 20 ms.  Apparently there is a
trade-off between accuracy and speed.
