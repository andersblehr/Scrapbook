%%%%%%%%%%%%%%%%%%%%%%%%%%%%%% -*- Mode: Latex -*- %%%%%%%%%%%%%%%%%%%%%%%%%%%%
%%% cn_sumry.tex --- 
%%% Author          : blehr
%%% Created On      : Fri Apr  9 13:06:01 1993
%%% Last Modified By: blehr
%%% Last Modified On: Mon Apr 12 16:58:17 1993
%%% RCS revision    : $Revision$ $Locker$
%%% Status          : In writing....
%%%%%%%%%%%%%%%%%%%%%%%%%%%%%%%%%%%%%%%%%%%%%%%%%%%%%%%%%%%%%%%%%%%%%%%%%%%%%%

\section{Summary}
\label{concl:summary}

All in all, I am very pleased that I got the opportunity of doing my
thesis work here in Marburg.  Some of the experiences I have obtained
during my stay will doubtlessly be of value later.  In particular I
believe that my experiences during the first couple of months of my
thesis work (cf.\ Section~\ref{intro:problem}) will come in handy, as
they were of a very ``this is the real world''-nature; having to phone
one vendor after the other in order to obtain information material,
going to fairs in Wiesbaden and Cologne, discussing with various
``experts'' how their solution would fit the needs of the group, going
through immeasurable numbers of brochures, folders and technical notes
to try to extract the essential information, etc.  I also had to work
with tasks not related to my professional field, such as soldering and
debugging the frame-grabber used to digitize the test images, all this
contributing to a varied and challenging sojourn in Germany.

On a more personal note, I consider it of immense value to have had
the opportunity of living and working in a ``foreign'' country for
almost a year (including my pre-thesis stay in Marburg during the
summer of 1992).  It is indisputably the best way to get to know the
culture, atmosphere, people and language of a country to stay and live
there for a relatively long period of time.  In fact, a pleasant
``side-effect'' of my stay in Germany is that I now consider myself
more or less fluent in German!
