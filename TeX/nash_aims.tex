%%%%%%%%%%%%%%%%%%%%%%%%%%%%%% -*- Mode: Latex -*- %%%%%%%%%%%%%%%%%%%%%%%%%%%%
%%% nash_aims.tex --- 
%%% Author          : Anders Blehr
%%% Created On      : Mon Apr 27 17:23:57 1992
%%% Last Modified By: Anders Blehr
%%% Last Modified On: Mon May  4 06:06:03 1992
%%% RCS revision    : $Revision: 1.1 $ $Locker:  $
%%% Status          : Unknown, Use with caution!
%%%%%%%%%%%%%%%%%%%%%%%%%%%%%%%%%%%%%%%%%%%%%%%%%%%%%%%%%%%%%%%%%%%%%%%%%%%%%%

\section{Aims and Goals}
\label{aims}

As mentioned in Section~\ref{tucc}, and further elaborated upon
in~\cite{solon}, {\tuc} aims at investigating 

\begin{quote}
  {\em how a naturally readable language can be a unifying framework
  for knowledge representation and natural language interfaces, {\em
  [that is, whether]} natural language is able to express knowledge in
  a way that, apart from its understandability, also is sufficient for
  an automated reasoning system to draw the right conclusions.}
\end{quote}

My task in this project has been to define a semi-natural language
({\nal} - {\em Semi-Natural Language\/}) and on the basis of this
develop and implement a prototype of a natural language based expert
system shell ({\nash} - {\em Natural Language Shell\/}).

It has been my intention neither to make {\nal} nor {\nash} out to be
anything more than prototypes. Whether they contain features of
interest to Tore Amble in his work with {\tuc} - as I hope and believe
they do - remains to be seen.

\subsection{{\nal} - The Language}

The term {\nal} will be used throughout the rest of this paper to
interchangeably refer to the {\em language\/} and to the {\em
analyzer\/} which constructs semantic representations from sentences
of the language. In working with {\nal}, I aimed at making it

\begin{itemize}
  \item appear, from the user's point of view, as close to standard
    English as practically tractable. That is, it should never require
    that the user employ constructs which are illegal in English.
  \item sufficiently formal. That is, it should be able to
    represent and convey knowledge and information in a manner which
    is unambiguous and suited for treatment by an automated reasoning
    system. In order to avoid the ambiguities of standard English, its
    vocabulary and set of legal constructs should be finite and well
    defined.
\end{itemize}

It is important to keep in mind that, no matter how ``natural'' {\nal}
may seem at first sight, it still is a {\em formal\/} language. No
effort has been put into making it more than a very restricted subset
of English.

\subsection{{\nash} - The Expert System Shell}

The aim of {\nash} has been to make a working prototype of a system
containing the basic features of an expert system shell, and of which
the dialogue handler is entirely based on {\nal}. That is, all input
to the system (defining rules, supplying knowledge, querying, etc.) as
well as all output from the system (answers to queries, explanations,
queries to the user, etc) should be represented in {\nal}.

To support the reasoning process, an internal knowledge representation
scheme had to be chosen. I had the option of using Tore Amble's
{\solon} language, which indeed is a powerful tool, but considering
that it (or rather its specification) is still under development and
thus changing from day to day, I chose to use first order predicate
calculus as a platform on which to build the scheme.
