%%%%%%%%%%%%%%%%%%%%%%%%%%%%%% -*- Mode: Latex -*- %%%%%%%%%%%%%%%%%%%%%%%%%%%%
%%% abstract.tex --- 
%%% Author          : blehr
%%% Created On      : Wed Mar  3 21:02:05 1993
%%% Last Modified By: blehr
%%% Last Modified On: Fri Apr  9 09:29:49 1993
%%% RCS revision    : $Revision: 1.2 $ $Locker:  $
%%% Status          : In writing....
%%%%%%%%%%%%%%%%%%%%%%%%%%%%%%%%%%%%%%%%%%%%%%%%%%%%%%%%%%%%%%%%%%%%%%%%%%%%%%

\begin{abstract}

  This paper constitutes my diploma thesis at the Norwegian Institute
  of Technology, Division of Computer Science \& Telematics.  On the
  basis of a relatively broad discussion of general digital image
  processing techniques, an $O(N)$ algorithm for determining the
  location of the pupil in an $N\times N$ image of the eye has been
  developed.  A prototype of the algorithm has been implemented and
  tested, and the results obtained were discussed in terms of a set of
  given requirements.  Lastly, some suggestions were made as to how to
  employ the algorithm in a real-time eye-tracking system.

\end{abstract}
