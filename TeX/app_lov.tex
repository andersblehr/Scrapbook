%%%%%%%%%%%%%%%%%%%%%%%%%%%%%% -*- Mode: Latex -*- %%%%%%%%%%%%%%%%%%%%%%%%%%%%
%%% lov.tex --- 
%%% Author          : Anders Blehr
%%% Created On      : Thu Apr 23 14:49:54 1992
%%% Last Modified By: Anders Blehr
%%% Last Modified On: Thu Apr 30 11:13:15 1992
%%% RCS revision    : $Revision: 1.1 $ $Locker:  $
%%% Status          : OK!
%%%%%%%%%%%%%%%%%%%%%%%%%%%%%%%%%%%%%%%%%%%%%%%%%%%%%%%%%%%%%%%%%%%%%%%%%%%%%%

\chapter{Extract from the Norwegian Nationality Act}
\label{app_lov}

Below the first three paragraphs of the Norwegian Nationality Act
(Dec.\ 8., 1950) are quoted, in Norwegian (footnotes are omitted):

\begin{center}
\subsubsection*{Kapitel 1. Korleis folk f\aa r borgarretten}
\end{center}

\begin{quote}
\begin{description}

\item[\S 1.]\ Barn f\aa r norsk borgarrett n\aa r det kjem til:
\begin{itemize}
\item[a)]{dersom mora er norsk borgar,}
\item[b)]{dersom faren er norsk borgar og foreldra er gifte,}
\item[c)]{dersom faren er d\o d, men var norsk borgar og gift med mor
  til barnet d\aa\ han d\o ydde.}
\end{itemize}
Hittebarn som er funne her i riket, vert rekna for norsk borgar til
dess noko anna vert opplyst.

\item[\S 2.]\ Har norsk mann og utenlandsk kvinne barn i lag fr\aa\ f\o
  r dei gifter seg med kvarandre, f\aa r barnet norsk borgarrett n\aa
  r foreldra vert vigde, s\aa\ framt det er ugift og under 18 \aa r.

\item[\S 3.]\ Utlending som har budd i riket fr\aa\ fylte 16 \aa r og
  tidlegare samanlagt i minst 5 \aa r, f\aa r norsk borgarrett n\aa r
  han etterat han fylte 21 \aa r, men f\o r han fyller 23 \aa r, gjev
  inn skriftleg melding til fylkesmannen om at han vil vera norsk
  borgar. Har han ikkje borgarrett i noko land, kan han gjeve inn slik
  melding s\aa\ snart han har fylt 18 \aa r, dersom han n\aa r han
  gjev inn meldinga har hatt bustad i riket dei siste 5 \aa ra og
  tidlegare har budd her i minst 5 \aa r til; det same gjeld s\aa
  framt han etterviser at han misser den framande borgarretten n\aa r
  han f\aa r norsk borgarrett.

  Er Noreg i krig, kan ingen borgar i fiendestat f\aa\ norsk
  borgarrett etter denne paragrafen. Det same gjeld den som er
  borgarrettslaus, men som seinast hadde borgarrett i fiendestat.

\end{description}
\end{quote}
