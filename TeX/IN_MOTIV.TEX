%%%%%%%%%%%%%%%%%%%%%%%%%%%%%% -*- Mode: Latex -*- %%%%%%%%%%%%%%%%%%%%%%%%%%%%
%%% in_motiv.tex --- 
%%% Author          : blehr
%%% Created On      : Mon Mar 22 00:57:11 1993
%%% Last Modified By: blehr
%%% Last Modified On: Sun Apr 11 12:05:35 1993
%%% RCS revision    : $Revision$ $Locker$
%%% Status          : In writing....
%%%%%%%%%%%%%%%%%%%%%%%%%%%%%%%%%%%%%%%%%%%%%%%%%%%%%%%%%%%%%%%%%%%%%%%%%%%%%%

\section{Thesis Motivation}
\label{intro:motivation}

Since the direction of gaze and the orientation of the visual stimuli
have to correspond as exactly as possible in order to be certain that
exactly those neural assemblies are stimulated whose activity is being
registered, it is of vital importance for the validity of the
experimental results that the exact eye position of the subject be
known at a given time.  Otherwise the danger exists of registering
neurological responses whose variations are not the result of the
assigned cognitive task, but of the subject's incorrect gaze
direction.  

In the part of the SC project carried out by the Eckhorn group, the
visual stimuli are generated on a computer monitor placed in front of
the monkey, sometimes referred to as the {\em inducing monitor\/}
(this term stems from the fact that the neural activity in the visual
cortex of the monkey is {\em induced\/} by the visual patterns on the
monitor).  The monkey is trained to focus on a bright point, called
the reference point, on the monitor, across which various geometrical
shapes move in different directions.  When the monkey does not focus
on the reference point, it is said to be {\em drifting\/}.

To gain absolute certainty about where the monkey at a given time
focuses, it is necessary to have a system which continuously tracks
the movements of the monkey's eyes.  This system has to comply to the
following requirements:
\begin{enumerate}
\item The position-analyzing method of the system may in no way
  interfere with recordings being made from the visual cortex of the
  monkey.
\item\label{req:first}A position measuring rate of 50 Hz.  Thus every
  20 ms, hereafter referred to as a {\em pass\/}, a coordinate pair
  relative to a reference point on the inducing monitor, corresponding
  to the point on the monitor on which the monkey had its focus at the
  beginning of the pass, has to be made available.  When the monkey is
  drifting, a warning signal has to be given with each pass as long as
  it continues to drift.
\item\label{req:2}The error in the position returned by the system
  should be less than or equal to $\pm 0.25^{\circ}$.
\item\label{req:3}The measuring range of the system should be at least
  $\pm 20^{\circ}$.
\item\label{req:4}The system should be relatively easy to configure to
  comply with the specific needs of an experiment.
\item\label{req:last}The price of the complete system may not exceed
  DM 12,000.
\end{enumerate}

There exist several commercially available eye tracking systems, some
of which are described in Section~\ref{back:track}.  For reasons
elaborated upon in Section~\ref{back:track:summary}, it was decided to
develop an entirely new eye tracking system, based on real-time image
processing of video input.

